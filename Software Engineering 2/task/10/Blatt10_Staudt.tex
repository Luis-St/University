\documentclass[
	fontsize=12pt,          % default font size 12pt
	paper=a4,               % DIN A4 page format
	numbers=noenddot,       % remove dots behind chapter numbers (e.g. 1.5 not 1.5.)
	listof=totoc,           % add list of figures, tables, etc. to ToC
	listof=entryprefix,     % add entry name to figures, tables, etc.
	listof=nochaptergap,    % no chapter gap for figures, tables, etc.
	bibliography=totoc,     % add bibliography to ToC but without a chapter number
	parskip=half            % half line spacing between paragraphs
	openany                 % chapters can start on any page
]{scrbook}
\usepackage{tikz}

% ##################################################
% ENCODING
% ##################################################
\usepackage{cmap}               % PDF character encoding

%\usepackage[T1]{fontenc}        % 8-bit font encoding
%\usepackage[utf8]{inputenc}     % UTF-8 input encoding
%\usepackage[german]{babel} %%%%% Sprache festlegen

\usepackage[utf8]{inputenc}
\usepackage[T1]{fontenc}
\usepackage{lmodern}
\usepackage{ngerman}


% ##################################################
% GENERAL
% ##################################################
\usepackage{scrhack}            % better KOMA adaptions
\usepackage[table]{xcolor}      % color support
\usepackage{chngcntr}           % for renumbering stuff


\usepackage{amsthm}


\usepackage{lipsum}             % lorem ipsum generator (used for example content)


\usepackage{mathtools}


% ##################################################
% PDF SETTINGS
% ##################################################
\usepackage[
	colorlinks=false,
	linkcolor=black,
	citecolor=black,
	filecolor=black,
	urlcolor=black,
	bookmarks=true,
	bookmarksopen=true,
	bookmarksopenlevel=3,
	bookmarksnumbered,
	plainpages=false,
	pdfpagelabels=true,
	hyperfootnotes
]{hyperref}


% ##################################################
% FONTS AND SPACING
% ##################################################
\renewcommand{\familydefault}{\sfdefault}   % default font
\usepackage[onehalfspacing]{setspace}       % default 1.5 line spacing
\raggedbottom   % don't stretch spacing to fit page length

\usepackage{anyfontsize} % use any font size

% font sizes and styles
\addtokomafont{chapter}{\sffamily\large\bfseries}  % chapter heading 
\addtokomafont{section}{\sffamily\normalsize\bfseries}
\addtokomafont{subsection}{\sffamily\normalsize\mdseries}
\addtokomafont{caption}{\sffamily\normalsize\mdseries}

% url font style
\usepackage{relsize}
\renewcommand*{\UrlFont}{\ttfamily\smaller\relax}


% ##################################################
% PAGE FORMATTING
% ##################################################
% Page layout / Seitenränder
\usepackage[
	bindingoffset=0cm,
	inner=2.5cm,
	outer=2.5cm,
	top=3cm,
	bottom=2cm
]{geometry}

% Page header
\usepackage[
	headsepline,        % seperator line beneath page header on normal pages
	plainheadsepline    % seperator line beneath page header on pages like ToC
]{scrlayer-scrpage}
\clearpairofpagestyles                  % clear default settings
\addtokomafont{pagehead}{\normalfont}   % use normal font for page header
\ohead*{\thepage}                       % page number
\ihead*{\leftmark}                      % chapter name


% ##################################################
% IMAGES AND FIGURES
% ##################################################
\usepackage{graphicx}       % support for including images
\graphicspath{{pictures/}}  % default path
\usepackage{float}          % better control over float positions

% simple numbering without chapter
\renewcommand{\thefigure}{\arabic{figure}}
\counterwithout{figure}{chapter}

% wrap text around figures
\usepackage{wrapfig}


% ##################################################
% TABLES
% ##################################################
% multi row and multi column table functionality
\usepackage{booktabs}   % beautiful table style
\usepackage{multirow}

% simple numbering without chapter
\renewcommand{\thetable}{\arabic{table}}
\counterwithout{table}{chapter}


% ##################################################
% SOURCE CODE LISTINGS
% ##################################################
\usepackage{listings}
\usepackage{beramono}   % use a typewriter font which supports bold characters

\renewcommand{\lstlistlistingname}{List of Code Listings}   % 
\renewcommand{\lstlistingname}{Code Listing}
\newcommand{\listoflolentryname}{\lstlistingname}   % prefix for List of Code Listings

% define colors for source code highlighting
\definecolor{codegreen}{rgb}{0,0.6,0}
\definecolor{codegray}{rgb}{0.5,0.5,0.5}
\definecolor{codepurple}{rgb}{0.5,0,0.33}
\definecolor{codepurblue}{rgb}{0.16,0.0,1.0}
\definecolor{backcolour}{rgb}{0.95,0.95,0.92}

% ##################################################
% TABLE OF CONTENTS
% ##################################################
\KOMAoptions{toc=chapterentrydotfill}       % dotted lines for chapters
\addtokomafont{chapterentry}{\normalfont}   % use normal font for chapter entries
\setuptoc{toc}{totoc}                       % add ToC to ToC

% spacing
\DeclareTOCStyleEntry[beforeskip=0cm]{chapter}{chapter}
\DeclareTOCStyleEntry[beforeskip=0cm]{section}{section}
\DeclareTOCStyleEntry[beforeskip=0cm]{default}{subsection}

% colons after entry names
\BeforeStartingTOC[lof]{\def\autodot{:}}
\BeforeStartingTOC[lot]{\def\autodot{:}}
\BeforeStartingTOC[lol]{\def\autodot{:}}


% ##################################################
% BIBLIOGRAPHY
% ##################################################
\iffalse
\usepackage{csquotes} % context sensitive quotation
\setlength\bibitemsep{.5\baselineskip} % increase spacing between entries
\setcounter{biburlnumpenalty}{9000} % break URLs on numbers
\setcounter{biburllcpenalty}{9000}  % break URLs on lower case letters
\setcounter{biburlucpenalty}{9000}  % break URLs on upper case letters

\fi

% ##################################################
% ABBREVIATIONS
% ##################################################
\usepackage[printonlyused]{acronym}


% ##################################################
% APPENDIX
% ##################################################
\usepackage[title,titletoc]{appendix}

% appendix chapter
\newcommand{\appendixchapter}[1]{
\cleardoublepage
\pagenumbering{arabic}
\renewcommand{\thepage}{\thechapter-\arabic{page}}
\chapter{#1}
}

% insert monthly report pdf as picture in order to keep page header
\newcommand{\monthlyreport}[2]{
\section{#1}
\centering
\includegraphics[trim=55 35 55 35,clip,width=1\textwidth]{#2}
\clearpage
}


% ##################################################
% Theoreme
% ##################################################

% Umgebung fuer Beispiele
\newtheorem{beispiel}{Beispiel}

% Umgebung fuer These
\newtheorem{these}{These}

% Umgebung fuer Definitionen
\newtheorem{definition}{Definition}


% ##################################################
% MISC
% ##################################################
% better referencing of images, tables, etc.
\usepackage[nameinlink, noabbrev]{cleveref}


% ############################################################################
% WIE MACH ICH DAS HIER?
% 
% 1. Schreibe im "titlepage" Abschnitt den Titel in das element mit "\fontsize{22}{22}"
% 2. Aus obsidian raus kopieren mit "Copy to LaTeX"
% 3. Zwischen "CONTENT STARTS HERE" und "CONTENT ENDS HERE" den Inhalt einfügen
% 4. Sections anpassen. Damit man ne gescheite Chapter > Section > Subsection Struktur hat
% 5. Leere Seite nach Titelseite am Anfang mit {\let\cleardoublepage\relax \chapter{Erstes Kapitel}} verhindern
%
% TABELLEN
% Müssen extra gemacht werden, da obsidian das nicht unterstützt
% Kann ich nur empfehlen: https://tableconvert.com/markdown-to-latex
%
% BILDER
% Muss man vermutlich auch extra machen, hab ich aber noch nicht probiert
% ############################################################################

\begin{document}

\begin{titlepage}
	\pagestyle{empty}
	
	% HFU Logo
	\begin{flushright}
		\begin{figure}[ht]
			\flushright
			\includegraphics[height=2cm]{../../for_latex/hfu}
		\end{figure}
	\end{flushright}
	
	\begin{center}
		\vspace{3cm}
		
		{\fontsize{22}{22} \selectfont \textbf{Blatt 10}}\\[5mm]
		{\fontsize{18}{18} \selectfont Software Engineering 2}
		
		\vspace{12cm}
		
		{\fontsize{14}{14} \selectfont Luis Staudt}
	\end{center}
\end{titlepage}

% ############################################################################
% CONTENT STARTS HERE
% ############################################################################

{\let\cleardoublepage\relax \chapter*{Aufgabe 1}}

\section*{Testfall für User Story 1}

\subsection*{User Story 1:}
Als Dozent möchte ich Aufgaben mit allen notwendigen Attributen und Kategorisierungen in der Datenbank speichern können, damit ich eine umfangreiche Sammlung an Klausuraufgaben erstellen kann.

\subsection*{Testfall 1.1: Vollständige Aufgabe erfolgreich speichern}
\begin{enumerate}
	\item Anmeldung im System als Dozent mit gültigen Anmeldedaten
	\item Navigation zur Funktion ``Neue Aufgabe erstellen''
	\item Eingabe aller erforderlichen Attribute:
	\begin{itemize}
		\item Name: ``Relationale Algebra Grundlagen''
		\item Aufgabentext: ``Erklären Sie die grundlegenden Operatoren der relationalen Algebra und geben Sie je ein Beispiel an.''
		\item Modulzugehörigkeit: ``Datenbanken I''
		\item Geschätzte Bearbeitungszeit: ``15 Minuten''
		\item Bloom'sche Taxonomie: ``Level 2 - Verstehen''
		\item Aufgabenformat: ``Offen''
		\item Musterlösung: ``Die grundlegenden Operatoren sind Selektion, Projektion, Vereinigung, Differenz, Kartesisches Produkt und Verbund...''
	\end{itemize}
	\item Bestätigung der Eingabe durch Klick auf ``Speichern''
	\item Überprüfung der Erfolgsmeldung: ``Aufgabe erfolgreich gespeichert''
	\item Verifikation in der Aufgabenübersicht: Die gespeicherte Aufgabe erscheint in der Liste mit allen eingegebenen Attributen
	\item Filtertest: Aufgabe kann über alle definierten Filterkriterien gefunden werden
\end{enumerate}

\subsection*{Erwartetes Ergebnis:}
Die Aufgabe wird vollständig mit allen Attributen in der Datenbank gespeichert und ist über die Suchfunktionen auffindbar.


\section*{Testfall für User Story 2}
\subsection*{User Story 2:}
Als Dozent möchte ich Klausuren aus vorhandenen Aufgaben nach definierten Kriterien zusammenstellen können, damit ich effizient Prüfungen erstellen kann.

\subsection*{Testfall 2.1: Klausur nach Filterkriterien zusammenstellen}
\begin{enumerate}
	\item Anmeldung im System als Dozent
	\item Vorbereitung: Mindestens 5 verschiedene Aufgaben mit unterschiedlichen Attributen sind in der Datenbank vorhanden
	\item Navigation zur Funktion ``Klausur zusammenstellen''
	\item Anwendung der Filterkriterien:
	\begin{itemize}
		\item Modul: ``Datenbanken I''
		\item Bloom'sche Taxonomie: Level 1-3 (Grundlagen bis Anwendung)
		\item Format: Gemischt (offen und geschlossen)
	\end{itemize}
	\item Überprüfung der Filterergebnisse: Nur Aufgaben, die den Kriterien entsprechen, werden angezeigt
	\item Auswahl von 4 Aufgaben durch Markierung
	\item Überprüfung der automatischen Zeitberechnung: Gesamtzeit wird korrekt summiert angezeigt
	\item Speicherung der Klausurzusammenstellung unter dem Namen ``Datenbanken I - Zwischenprüfung''
	\item Bestätigung der Speicherung durch Erfolgsmeldung
	\item Verifikation: Klausur erscheint in der Übersicht gespeicherter Klausuren
	\item Exporttest: Klausur kann als PDF-Datei exportiert werden
\end{enumerate}

\subsection*{Erwartetes Ergebnis:}
Eine Klausur wird erfolgreich aus gefilterten Aufgaben zusammengestellt, die Gesamtzeit wird korrekt berechnet, und die Klausur kann gespeichert und exportiert werden.


{\let\cleardoublepage\relax \chapter*{Aufgabe 2}}

Für die Kaffeemaschine werden folgende Unit-Test-Klassen erstellt, um alle wichtigen Methoden und Zustandsübergänge abzudecken:

\section*{Testabdeckung}
\subsection*{Zu testende Klassen und Methoden:}
\begin{enumerate}
	\item \textbf{CoffeeMachine Klasse:}
	\begin{itemize}
		\item Konstruktor und Initialisierung
		\item setState() - Zustandswechsel
		\item Getter/Setter für Zutaten und Münzguthaben
		\item hasEnoughIngredients() - Zutatenprüfung
	\end{itemize}
	
	\item \textbf{State-Klassen:} Alle Zustandsübergänge für jeden State
	\begin{itemize}
		\item IdleState
		\item CoinAcceptedState
		\item CheckingIngredientsState
		\item PreparingCoffeeState
		\item CoffeeReadyState
		\item ErrorState
	\end{itemize}
\end{enumerate}

\subsection*{Testszenarien:}
\begin{itemize}
	\item \textbf{Normale Abläufe:} Vollständiger Kaffeezubereitungsprozess
	\item \textbf{Fehlerfälle:} Fehlende Zutaten, mehrfache Münzeinwürfe
	\item \textbf{Grenzfälle:} Zustandsübergänge bei verschiedenen Bedingungen
	\item \textbf{Zutatenverwaltung:} Auffüllen und Verbrauch von Ressourcen
\end{itemize}

{\let\cleardoublepage\relax \chapter*{Aufgabe 3}}

\section*{Identifikation der Äquivalenzklassen}
Für die Bestellmenge mit dem gültigen Bereich 1--1000 Stück ergeben sich folgende Äquivalenzklassen:
\begin{table}[h]
	\centering
	\begin{tabular}{|l|l|l|l|}
		\hline
		\textbf{Klasse} & \textbf{Beschreibung} & \textbf{Bereich} & \textbf{Erwartetes Verhalten} \\
		\hline
		ÄK1 & Ungültig (zu klein) & $< 1$ & Ablehnung \\
		\hline
		ÄK2 & Gültig & $1 \leq x \leq 1000$ & Annahme \\
		\hline
		ÄK3 & Ungültig (zu groß) & $> 1000$ & Ablehnung \\
		\hline
	\end{tabular}
	\caption{Äquivalenzklassen für Bestellmenge}
\end{table}

\section*{Grenzwertanalyse}
Die kritischen Grenzwerte sind:
\begin{table}[h]
	\centering
	\begin{tabular}{|l|l|l|}
		\hline
		\textbf{Grenzwert} & \textbf{Wert} & \textbf{Erwartetes Verhalten} \\
		\hline
		Untere Grenze - 1 & 0 & Ablehnung \\
		\hline
		Untere Grenze & 1 & Annahme \\
		\hline
		Untere Grenze + 1 & 2 & Annahme \\
		\hline
		Obere Grenze - 1 & 999 & Annahme \\
		\hline
		Obere Grenze & 1000 & Annahme \\
		\hline
		Obere Grenze + 1 & 1001 & Ablehnung \\
		\hline
	\end{tabular}
	\caption{Grenzwertanalyse für Bestellmenge}
\end{table}
\newpage

\section*{Testfälle}
\begin{table}[h]
	\centering
	\begin{tabular}{|p{2cm}|p{3cm}|p{2cm}|p{4cm}|p{3cm}|}
		\hline
		\textbf{Test-ID} & \textbf{Beschreibung} & \textbf{Eingabe} & \textbf{Erwartetes Ergebnis} & \textbf{Äquivalenz-klasse} \\
		\hline
		T1 & Negative Bestellmenge & -5 & Fehlermeldung: ``Bestellung abgelehnt'' & ÄK1 \\
		\hline
		T2 & Bestellmenge Null & 0 & Fehlermeldung: ``Bestellung abgelehnt'' & ÄK1 \\
		\hline
		T3 & Minimum gültige Menge & 1 & Bestellung akzeptiert & ÄK2 \\
		\hline
		T4 & Menge knapp über Minimum & 2 & Bestellung akzeptiert & ÄK2 \\
		\hline
		T5 & Mittlere gültige Menge & 500 & Bestellung akzeptiert & ÄK2 \\
		\hline
		T6 & Menge knapp unter Maximum & 999 & Bestellung akzeptiert & ÄK2 \\
		\hline
		T7 & Maximum gültige Menge & 1000 & Bestellung akzeptiert & ÄK2 \\
		\hline
		T8 & Menge über Maximum & 1001 & Fehlermeldung: ``Bestellung abgelehnt'' & ÄK3 \\
		\hline
		T9 & Sehr große Menge & 5000 & Fehlermeldung: ``Bestellung abgelehnt'' & ÄK3 \\
		\hline
	\end{tabular}
	\caption{Testfälle für Online-Bestellung}
\end{table}

{\let\cleardoublepage\relax \chapter*{Aufgabe 4}}

\section*{Identifikation der Äquivalenzklassen}
Für ein Programm zur Dreieckerkennung mit drei ganzzahligen positiven Werten ergeben sich folgende Äquivalenzklassen:
\begin{table}[h]
	\centering
	\begin{tabular}{|l|l|l|}
		\hline
		\textbf{Klasse} & \textbf{Beschreibung} & \textbf{Erwartetes Ergebnis} \\
		\hline
		ÄK1 & Ungültige Eingabe (nicht-positive Werte) & Fehlermeldung \\
		\hline
		ÄK2 & Keine Dreiecksungleichung erfüllt & Fehlermeldung \\
		\hline
		ÄK3 & Gleichseitiges Dreieck ($a = b = c$) & ``Gleichseitiges Dreieck'' \\
		\hline
		ÄK4 & Gleichschenkliges Dreieck (zwei Seiten gleich) & ``Gleichschenkliges Dreieck'' \\
		\hline
		ÄK5 & Ungleichseitiges Dreieck (alle Seiten verschieden) & ``Ungleichseitiges Dreieck'' \\
		\hline
	\end{tabular}
	\caption{Äquivalenzklassen für Dreieckerkennung}
\end{table}

\section*{Detaillierte Äquivalenzklassen}
\subsection*{Mathematische Bedingungen:}
\begin{itemize}
	\item \textbf{Gültige Dreiecke:} $a + b > c$ und $a + c > b$ und $b + c > a$
	\item \textbf{Gleichseitig:} $a = b = c$
	\item \textbf{Gleichschenklig:} $a = b \neq c$ oder $a = c \neq b$ oder $b = c \neq a$
	\item \textbf{Ungleichseitig:} $a \neq b \neq c \neq a$
\end{itemize}

\subsection*{Testfälle}
\begin{table}[h]
	\centering
	\begin{tabular}{|p{1.5cm}|p{4cm}|p{2.5cm}|p{4cm}|p{2cm}|}
		\hline
		\textbf{Test-ID} & \textbf{Beschreibung} & \textbf{Eingabe (a,b,c)} & \textbf{Erwartetes Ergebnis} & \textbf{ÄK} \\
		\hline
		T1 & Negative Werte & (-1, 5, 3) & Fehlermeldung & ÄK1 \\
		\hline
		T2 & Null-Werte & (0, 4, 5) & Fehlermeldung & ÄK1 \\
		\hline
		T3 & Dreiecksungleichung verletzt (1) & (1, 2, 5) & Fehlermeldung & ÄK2 \\
		\hline
		T4 & Dreiecksungleichung verletzt (2) & (1, 1, 3) & Fehlermeldung & ÄK2 \\
		\hline
		T5 & Gleichseitiges Dreieck & (5, 5, 5) & ``Gleichseitiges Dreieck'' & ÄK3 \\
		\hline
		T6 & Gleichseitiges Dreieck (klein) & (1, 1, 1) & ``Gleichseitiges Dreieck'' & ÄK3 \\
		\hline
		T7 & Gleichschenklig (a=b) & (5, 5, 3) & ``Gleichschenkliges Dreieck'' & ÄK4 \\
		\hline
		T8 & Gleichschenklig (a=c) & (4, 6, 4) & ``Gleichschenkliges Dreieck'' & ÄK4 \\
		\hline
		T9 & Gleichschenklig (b=c) & (3, 5, 5) & ``Gleichschenkliges Dreieck'' & ÄK4 \\
		\hline
		T10 & Ungleichseitig & (3, 4, 5) & ``Ungleichseitiges Dreieck'' & ÄK5 \\
		\hline
		T11 & Ungleichseitig (groß) & (13, 14, 15) & ``Ungleichseitiges Dreieck'' & ÄK5 \\
		\hline
		T12 & Grenzfall Dreiecksungleichung & (1, 2, 3) & Fehlermeldung & ÄK2 \\
		\hline
	\end{tabular}
	\caption{Testfälle für Dreieckerkennung}
\end{table}
\newpage

\section*{Zusätzliche Grenzfälle:}
\begin{itemize}
	\item \textbf{Minimale gültige Werte:} (1, 1, 1) - Gleichseitig
	\item \textbf{Grenzfall Gleichschenklig:} (2, 2, 3) - Gerade noch gültiges gleichschenkliges Dreieck
	\item \textbf{Grenzfall Ungleichung:} (5, 5, 9) - Gerade noch gültiges gleichschenkliges Dreieck
	\item \textbf{Große Werte:} (100, 150, 200) - Test mit größeren Zahlen
\end{itemize}

% ############################################################################
% CONTENT ENDS HERE
% ############################################################################

\end{document}
