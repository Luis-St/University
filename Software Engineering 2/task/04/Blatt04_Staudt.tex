\documentclass[
	fontsize=12pt,          % default font size 12pt
	paper=a4,               % DIN A4 page format
	numbers=noenddot,       % remove dots behind chapter numbers (e.g. 1.5 not 1.5.)
	listof=totoc,           % add list of figures, tables, etc. to ToC
	listof=entryprefix,     % add entry name to figures, tables, etc.
	listof=nochaptergap,    % no chapter gap for figures, tables, etc.
	bibliography=totoc,     % add bibliography to ToC but without a chapter number
	parskip=half            % half line spacing between paragraphs
	openany                 % chapters can start on any page
]{scrbook}
\usepackage{tikz}

% ##################################################
% ENCODING
% ##################################################
\usepackage{cmap}               % PDF character encoding

%\usepackage[T1]{fontenc}        % 8-bit font encoding
%\usepackage[utf8]{inputenc}     % UTF-8 input encoding
%\usepackage[german]{babel} %%%%% Sprache festlegen

\usepackage[utf8]{inputenc}
\usepackage[T1]{fontenc}
\usepackage{lmodern}
\usepackage{ngerman}


% ##################################################
% GENERAL
% ##################################################
\usepackage{scrhack}            % better KOMA adaptions
\usepackage[table]{xcolor}      % color support
\usepackage{chngcntr}           % for renumbering stuff


\usepackage{amsthm}


\usepackage{lipsum}             % lorem ipsum generator (used for example content)


\usepackage{mathtools}


% ##################################################
% PDF SETTINGS
% ##################################################
\usepackage[
	colorlinks=false,
	linkcolor=black,
	citecolor=black,
	filecolor=black,
	urlcolor=black,
	bookmarks=true,
	bookmarksopen=true,
	bookmarksopenlevel=3,
	bookmarksnumbered,
	plainpages=false,
	pdfpagelabels=true,
	hyperfootnotes
]{hyperref}


% ##################################################
% FONTS AND SPACING
% ##################################################
\renewcommand{\familydefault}{\sfdefault}   % default font
\usepackage[onehalfspacing]{setspace}       % default 1.5 line spacing
\raggedbottom   % don't stretch spacing to fit page length

\usepackage{anyfontsize} % use any font size

% font sizes and styles
\addtokomafont{chapter}{\sffamily\large\bfseries}  % chapter heading 
\addtokomafont{section}{\sffamily\normalsize\bfseries}
\addtokomafont{subsection}{\sffamily\normalsize\mdseries}
\addtokomafont{caption}{\sffamily\normalsize\mdseries}

% url font style
\usepackage{relsize}
\renewcommand*{\UrlFont}{\ttfamily\smaller\relax}


% ##################################################
% PAGE FORMATTING
% ##################################################
% Page layout / Seitenränder
\usepackage[
	bindingoffset=0cm,
	inner=2.5cm,
	outer=2.5cm,
	top=3cm,
	bottom=2cm
]{geometry}

% Page header
\usepackage[
	headsepline,        % seperator line beneath page header on normal pages
	plainheadsepline    % seperator line beneath page header on pages like ToC
]{scrlayer-scrpage}
\clearpairofpagestyles                  % clear default settings
\addtokomafont{pagehead}{\normalfont}   % use normal font for page header
\ohead*{\thepage}                       % page number
\ihead*{\leftmark}                      % chapter name


% ##################################################
% IMAGES AND FIGURES
% ##################################################
\usepackage{graphicx}       % support for including images
\graphicspath{{pictures/}}  % default path
\usepackage{float}          % better control over float positions

% simple numbering without chapter
\renewcommand{\thefigure}{\arabic{figure}}
\counterwithout{figure}{chapter}

% wrap text around figures
\usepackage{wrapfig}


% ##################################################
% TABLES
% ##################################################
% multi row and multi column table functionality
\usepackage{booktabs}   % beautiful table style
\usepackage{multirow}

% simple numbering without chapter
\renewcommand{\thetable}{\arabic{table}}
\counterwithout{table}{chapter}


% ##################################################
% SOURCE CODE LISTINGS
% ##################################################
\usepackage{listings}
\usepackage{beramono}   % use a typewriter font which supports bold characters

\renewcommand{\lstlistlistingname}{List of Code Listings}   % 
\renewcommand{\lstlistingname}{Code Listing}
\newcommand{\listoflolentryname}{\lstlistingname}   % prefix for List of Code Listings

% define colors for source code highlighting
\definecolor{codegreen}{rgb}{0,0.6,0}
\definecolor{codegray}{rgb}{0.5,0.5,0.5}
\definecolor{codepurple}{rgb}{0.5,0,0.33}
\definecolor{codepurblue}{rgb}{0.16,0.0,1.0}
\definecolor{backcolour}{rgb}{0.95,0.95,0.92}

% ##################################################
% TABLE OF CONTENTS
% ##################################################
\KOMAoptions{toc=chapterentrydotfill}       % dotted lines for chapters
\addtokomafont{chapterentry}{\normalfont}   % use normal font for chapter entries
\setuptoc{toc}{totoc}                       % add ToC to ToC

% spacing
\DeclareTOCStyleEntry[beforeskip=0cm]{chapter}{chapter}
\DeclareTOCStyleEntry[beforeskip=0cm]{section}{section}
\DeclareTOCStyleEntry[beforeskip=0cm]{default}{subsection}

% colons after entry names
\BeforeStartingTOC[lof]{\def\autodot{:}}
\BeforeStartingTOC[lot]{\def\autodot{:}}
\BeforeStartingTOC[lol]{\def\autodot{:}}


% ##################################################
% BIBLIOGRAPHY
% ##################################################
\iffalse
\usepackage{csquotes} % context sensitive quotation
\setlength\bibitemsep{.5\baselineskip} % increase spacing between entries
\setcounter{biburlnumpenalty}{9000} % break URLs on numbers
\setcounter{biburllcpenalty}{9000}  % break URLs on lower case letters
\setcounter{biburlucpenalty}{9000}  % break URLs on upper case letters

\fi

% ##################################################
% ABBREVIATIONS
% ##################################################
\usepackage[printonlyused]{acronym}


% ##################################################
% APPENDIX
% ##################################################
\usepackage[title,titletoc]{appendix}

% appendix chapter
\newcommand{\appendixchapter}[1]{
\cleardoublepage
\pagenumbering{arabic}
\renewcommand{\thepage}{\thechapter-\arabic{page}}
\chapter{#1}
}

% insert monthly report pdf as picture in order to keep page header
\newcommand{\monthlyreport}[2]{
\section{#1}
\centering
\includegraphics[trim=55 35 55 35,clip,width=1\textwidth]{#2}
\clearpage
}


% ##################################################
% Theoreme
% ##################################################

% Umgebung fuer Beispiele
\newtheorem{beispiel}{Beispiel}

% Umgebung fuer These
\newtheorem{these}{These}

% Umgebung fuer Definitionen
\newtheorem{definition}{Definition}


% ##################################################
% MISC
% ##################################################
% better referencing of images, tables, etc.
\usepackage[nameinlink, noabbrev]{cleveref}


% ############################################################################
% WIE MACH ICH DAS HIER?
% 
% 1. Schreibe im "titlepage" Abschnitt den Titel in das element mit "\fontsize{22}{22}"
% 2. Aus obsidian raus kopieren mit "Copy to LaTeX"
% 3. Zwischen "CONTENT STARTS HERE" und "CONTENT ENDS HERE" den Inhalt einfügen
% 4. Sections anpassen. Damit man ne gescheite Chapter > Section > Subsection Struktur hat
% 5. Leere Seite nach Titelseite am Anfang mit {\let\cleardoublepage\relax \chapter{Erstes Kapitel}} verhindern
%
% TABELLEN
% Müssen extra gemacht werden, da obsidian das nicht unterstützt
% Kann ich nur empfehlen: https://tableconvert.com/markdown-to-latex
%
% BILDER
% Muss man vermutlich auch extra machen, hab ich aber noch nicht probiert
% ############################################################################

\begin{document}

\begin{titlepage}
	\pagestyle{empty}
	
	% HFU Logo
	\begin{flushright}
		\begin{figure}[ht]
			\flushright
			\includegraphics[height=2cm]{../../for_latex/hfu}
		\end{figure}
	\end{flushright}
	
	\begin{center}
		\vspace{3cm}
		
		{\fontsize{22}{22} \selectfont \textbf{Blatt 4}}\\[5mm]
		{\fontsize{18}{18} \selectfont Software Engineering 2}
		
		\vspace{12cm}
		
		{\fontsize{14}{14} \selectfont Luis Staudt}
	\end{center}
\end{titlepage}

% ############################################################################
% CONTENT STARTS HERE
% ############################################################################

{\let\cleardoublepage\relax \chapter*{Aufgabe 1}}

\section*{1. Ziele der Spezifikationen:}
\begin{itemize}
	\item \textbf{SyRS (System Requirements Specification)}: Identifiziert technische Anforderungen für das System-of-Interest und beschreibt, was das System aus technischer Sicht tun soll.
	Dient als Brücke zwischen Auftraggeber und technischer Community, indem sie alle Eingaben, Ausgaben und erforderlichen Beziehungen zwischen diesen vollständig beschreibt.
	\item \textbf{SRS (Software Requirements Specification)}: Spezifiziert detailliert die Anforderungen für ein bestimmtes Softwareprodukt oder Programm in einer spezifischen Umgebung.
	Definiert alle erforderlichen Fähigkeiten, dokumentiert Bedingungen und Einschränkungen unter denen die Software arbeiten muss.
\end{itemize}

\section*{2. Notwendigkeit:}
\begin{itemize}
	\item \textbf{SyRS}: Wird benötigt, wenn ein System entwickelt wird, das mehrere Komponenten umfasst und eine übergeordnete technische Beschreibung erforderlich ist.
	\item \textbf{SRS}: Wird benötigt, wenn spezifisch ein Softwareprodukt entwickelt wird.
	Bei größeren Systemen erweitert die SRS die im SyRS definierten softwarebezogenen Anforderungen und stimmt mit diesen überein.
\end{itemize}

\newpage
\section*{2. Vergleich zwischen SyRS und SRS}

\begin{table}[htbp]
	\centering
	\begin{tabular}{>{\raggedright\arraybackslash}p{2.5cm}>{\raggedright\arraybackslash}p{6cm}>{\raggedright\arraybackslash}p{6cm}}
		\toprule
		\textbf{Aspekt} & \textbf{System Requirements Specification (SyRS)} & \textbf{Software Requirements Specification (SRS)} \\
		\midrule
		Fokus & Gesamtsystem (Hardware, Software, Mensch) & Speziell Softwarekomponenten \\
		\midrule
		Umfang & Systemebene, kann mehrere Komponenten umfassen & Fokussiert auf Softwareprogramm oder -set \\
		\midrule
		Detaillierungsgrad & Übergeordnete Systemanforderungen & Detaillierte Softwareanforderungen \\
		\midrule
		Spezifische Inhalte & Physikalische Eigenschaften, Umgebungsbedingungen, Systemsicherheit & Logische Datenbankanforderungen, Softwareattribute \\
		\midrule
		Gemeinsamkeiten & \multicolumn{2}{p{12cm}}{Beide enthalten: funktionale Anforderungen, Schnittstellen, Leistungsanforderungen, Benutzeranforderungen und Verifizierungsansätze} \\
		\midrule
		Verantwortung & Anforderer und Entwicklerteam & Lieferant, Auftraggeber oder beide \\
		\bottomrule
	\end{tabular}
	\caption{Vergleich zwischen SyRS und SRS}
	\label{tab:comparison}
\end{table}

\newpage
{\let\cleardoublepage\relax \chapter*{Aufgabe 2}}
\section*{Revisionshistorie}
\begin{center}
	\begin{tabular}{|l|l|l|p{8cm}|}
		\hline
		\textbf{Datum} & \textbf{Version} & \textbf{Autor} & \textbf{Beschreibung} \\
		\hline
		&  &  &  \\
		\hline
		&  &  &  \\
		\hline
	\end{tabular}
\end{center}

\section*{Definitionen}
\textcolor[gray]{0.7}{
	Definitionen für Wörter oder Ausdrücke, die eine besondere Bedeutung über normale Wörterbuchdefinitionen hinaus haben.
}
\begin{center}
	\begin{tabular}{|l|l|l|p{8cm}|}
		\hline
		\textbf{Begriff} & \textbf{Definition} \\
		\hline
		&  \\
		\hline
		&  \\
		\hline
	\end{tabular}
\end{center}

\section*{Akronyme und Abkürzungen}
\textcolor[gray]{0.7}{
	Ausschreiben oder Definition aller im Dokument verwendeten Akronyme und Abkürzungen
}
\begin{center}
	\begin{tabular}{|l|l|l|p{8cm}|}
		\hline
		\textbf{Akronym/Abkürzung} & \textbf{Definition} \\
		\hline
		&  \\
		\hline
		&  \\
		\hline
	\end{tabular}
\end{center}

\section*{Einleitung}

\subsection*{Zweck}
\textcolor[gray]{0.7}{
	Beschreibung des Zwecks der zu spezifizierenden Software.
	Erläuterung, wofür diese Spezifikation verwendet wird und an wen sie sich richtet.
}

\subsection*{Umfang}
\textcolor[gray]{0.7}{
	Beschreibung des Umfangs der betreffenden Software.
	Identifikation der Software-Produkte nach Namen.
	Erklärung, was die Software-Produkte tun werden.
	Beschreibung der Anwendung der spezifizierten Software.
}

\subsection*{Produktperspektive}
\textcolor[gray]{0.7}{
	Definition der Beziehung des Systems zu anderen verwandten Produkten.
	Wenn das Produkt Teil eines größeren Systems ist, Beschreibung der Beziehung.
	Identifikation der Schnittstellen zwischen dem Produkt und dem größeren System.
}

\subsubsection*{Systemschnittstellen}
\textcolor[gray]{0.7}{
	Auflistung jeder Systemschnittstelle und Identifikation der Funktionalität.
}

\subsubsection*{Benutzerschnittstellen}
\textcolor[gray]{0.7}{
	Spezifikation der logischen Merkmale jeder Schnittstelle zwischen Software und Benutzern.
}

\subsubsection*{Betriebsabläufe}
\textcolor[gray]{0.7}{
	Spezifikation der normalen und speziellen Operationen, die vom Benutzer benötigt werden.
}


\subsection*{Produktfunktionen}
\textcolor[gray]{0.7}{
	Zusammenfassung der Hauptfunktionen, die die Software ausführen wird.
	Organisation der Funktionen, um die Liste für den Auftraggeber verständlich zu machen.
	Verwendung textueller oder grafischer Methoden, um verschiedene Funktionen und Beziehungen darzustellen.
}

\subsection*{Benutzermerkmale}
\textcolor[gray]{0.7}{
	Beschreibung der allgemeinen Merkmale der beabsichtigten Benutzer.
	Bildungsniveau, Erfahrung, technisches Fachwissen, mögliche Behinderungen und andere Faktoren.
}

\subsection*{Einschränkungen}
\textcolor[gray]{0.7}{
	Allgemeine Beschreibung von Faktoren, die die Optionen des Entwicklers einschränken.
	Regulatorische Richtlinien, Hardware-Beschränkungen, Schnittstellen zu anderen Anwendungen,
	Parallelbetriebs-Anforderungen, Sicherheits- und Schutzanforderungen.
}

\subsection*{Annahmen und Abhängigkeiten}
\textcolor[gray]{0.7}{
	Auflistung von Faktoren, die die in der SRS angegebenen Anforderungen beeinflussen.
	Annahmen über die Verfügbarkeit von Ressourcen, Komponenten usw.
	Abhängigkeiten von externen Faktoren.
}

\section*{Spezifische Anforderungen}

\subsection*{Externe Schnittstellen}
\textcolor[gray]{0.7}{
	Definition aller Ein- und Ausgaben des Softwaresystems.
	Name des Elements, Beschreibung des Zwecks, Quelle der Eingabe oder Ziel der Ausgabe,
	Gültigkeitsbereich, Genauigkeit oder Toleranz, Maßeinheiten, Timing, Datenformate.
}

\subsection*{Funktionen}
\textcolor[gray]{0.7}{
	Definition der grundlegenden Aktionen der Software.
	Gültigkeitsprüfungen der Eingaben, genaue Abfolge von Operationen,
	Reaktionen auf abnormale Situationen, Auswirkung von Parametern,
	Beziehung von Ausgaben zu Eingaben.
}

\subsection*{Qualitätsanforderungen}
\textcolor[gray]{0.7}{
	Definition von Gebrauchstauglichkeits- und Qualitätsanforderungen.
	Messbare Kriterien für Effektivität, Effizienz, Zufriedenheit und Vermeidung von Schäden.
}

\subsection*{Leistungsanforderungen}
\textcolor[gray]{0.7}{
	Spezifikation sowohl statischer als auch dynamischer numerischer Anforderungen.
	Statische Anforderungen (z.B. Anzahl von Terminals, Benutzern, zu verarbeitende Daten).
	Dynamische Anforderungen (z.B. Transaktionen/Aufgaben pro Zeiteinheit).
}

\subsection*{Designeinschränkungen}
\textcolor[gray]{0.7}{
	Spezifikation von Einschränkungen für das Systemdesign, die durch externe Standards auferlegt werden.
	Einhaltung von Standards, Hardware- und Softwarebeschränkungen.
}

\subsection*{Standardkonformität}
\textcolor[gray]{0.7}{
	Spezifikation von Anforderungen, die sich aus bestehenden Standards oder Vorschriften ergeben.
	Berichtsformat, Datenbenennung, Buchführungsverfahren, Prüfpfadverfolgung.
}

\subsection*{Softwaresystemattribute}
\textcolor[gray]{0.7}{
	Spezifikation erforderlicher Attribute des Softwareprodukts.
}

\subsubsection*{Zuverlässigkeit}
\textcolor[gray]{0.7}{
	Spezifikation von Faktoren, die erforderlich sind, um die erforderliche Zuverlässigkeit zu gewährleisten.
}

\subsubsection*{Verfügbarkeit}
\textcolor[gray]{0.7}{
	Spezifikation von Faktoren, die erforderlich sind, um ein definiertes Verfügbarkeitsniveau zu garantieren.
}

\subsubsection*{Sicherheit}
\textcolor[gray]{0.7}{
	Spezifikation von Anforderungen zum Schutz der Software vor versehentlichem oder böswilligem Zugriff,
	Nutzung, Modifikation, Zerstörung oder Offenlegung.
}

\subsubsection*{Wartbarkeit}
\textcolor[gray]{0.7}{
	Spezifikation von Attributen der Software, die sich auf die Wartungsfreundlichkeit beziehen.
}

\subsubsection*{Portabilität}
\textcolor[gray]{0.7}{
	Spezifikation von Attributen, die sich auf die Übertragbarkeit der Software in andere Umgebungen beziehen.
}

\subsection*{Verifizierung}
\textcolor[gray]{0.7}{
	Bereitstellung der Verifizierungsansätze und -methoden, die zur Qualifizierung der Software geplant sind.
	Verifizierungsansätze für jede Anforderung, Testmethoden, Validierungskriterien.
}

\newpage

{\let\cleardoublepage\relax \chapter*{Aufgabe 3}}
\section*{Revisionshistorie}
\begin{center}
	\begin{tabular}{|l|l|l|p{8cm}|}
		\hline
		\textbf{Datum} & \textbf{Version} & \textbf{Autor} & \textbf{Beschreibung} \\
		\hline
		\today & 1.0 & Luis Staudt & Initiale Version \\
		\hline
		&  &  &  \\
		\hline
	\end{tabular}
\end{center}

\section*{Definitionen}
\begin{tabular}{p{4cm}p{11cm}}
	\textbf{Begriff} & \textbf{Definition} \\
	\hline
	Bloom'sche Taxonomie & Hierarchisches System zur Klassifizierung von Lernzielen in sechs Ebenen: erinnern, verstehen, anwenden, analysieren, bewerten, erschaffen \\
	Geschlossene Aufgabe & Aufgabentyp mit vorgegebenen Antwortmöglichkeiten \\
	Offene Aufgabe & Aufgabentyp ohne vorgegebene Antwortmöglichkeiten \\
	Modul & Lehrveranstaltungseinheit im akademischen Kontext \\
\end{tabular}

\section*{Referenzen}
\begin{enumerate}
	\item Fallstudie: Generierung von Klausuren aus bestehender Klausuraufgabensammlung
\end{enumerate}

\section*{Akronyme und Abkürzungen}
\begin{tabular}{p{4cm}p{11cm}}
	\textbf{Akronym/Abkürzung} & \textbf{Definition} \\
	\hline
	SRS & Software Requirements Specification \\
	DB & Datenbank \\
\end{tabular}

\section*{Einleitung}

\subsection*{Zweck}
Diese Softwareanforderungsspezifikation (SRS) beschreibt die funktionalen und nicht-funktionalen Anforderungen an den Klausurgenerator, der zur Erstellung und Verwaltung von Klausuraufgaben und zur Zusammenstellung von Klausuren dient. Dieses Dokument richtet sich an das Entwicklungsteam sowie an die zukünftigen Nutzer des Systems.

\subsection*{Umfang}
Der Klausurgenerator soll Dozenten ermöglichen, Klausuraufgaben zu erstellen, zu verwalten und zu kategorisieren, sowie aus diesen Aufgaben Klausuren zusammenzustellen. Die Software soll folgende Hauptfunktionen bieten:
\begin{itemize}
	\item Eingabe und Speicherung von Klausuraufgaben mit definierten Attributen in einer Datenbank
	\item Speicherung von Musterlösungen oder richtigen Lösungen
	\item Zuordnung von Aufgaben zu Lehrveranstaltungen/Modulen
	\item Zusammenstellung von Klausuren nach definierten Kriterien
	\item Druck von einzelnen Aufgaben und vollständigen Klausuren
\end{itemize}

\subsection*{Produktperspektive}
Der Klausurgenerator ist ein eigenständiges System, das von Dozenten zur Klausurerstellung und -verwaltung genutzt wird. Es ist nicht in bestehende Systeme integriert.

\subsubsection*{Systemschnittstellen}
Der Klausurgenerator benötigt keine externen Systemschnittstellen in der ersten Version.

\subsubsection*{Benutzerschnittstellen}
Die Benutzerschnittstelle soll intuitiv und benutzerfreundlich sein und folgende Bereiche umfassen:
\begin{itemize}
	\item Eingabeformulare für Klausuraufgaben mit allen Attributen
	\item Verwaltungsbereich für bestehende Aufgaben
	\item Filterbereich zur Auswahl von Aufgaben nach verschiedenen Kriterien
	\item Klausurzusammenstellungsbereich
	\item Druck- und Exportfunktionen
\end{itemize}

\subsection*{Produktfunktionen}
Die Hauptfunktionen des Klausurgenerators sind:
\begin{itemize}
	\item Verwaltung von Klausuraufgaben (Erstellen, Bearbeiten, Löschen)
	\item Kategorisierung von Aufgaben nach Bloom'scher Taxonomie, Format und Typ
	\item Zusammenstellung von Klausuren aus vorhandenen Aufgaben
	\item Druck von Aufgaben und Klausuren
\end{itemize}

\subsection*{Benutzermerkmale}
Die primären Nutzer sind Dozenten an Bildungseinrichtungen, die regelmäßig Klausuren erstellen müssen. Es wird erwartet, dass die Nutzer grundlegende Computerkenntnisse besitzen und mit pädagogischen Konzepten wie der Bloom'schen Taxonomie vertraut sind.

\subsection*{Einschränkungen}
\begin{itemize}
	\item Die Software soll auf gängigen Betriebssystemen lauffähig sein
	\item Die Benutzerschnittstelle soll in deutscher Sprache sein
\end{itemize}

\subsection*{Annahmen und Abhängigkeiten}
\begin{itemize}
	\item Es wird angenommen, dass die Nutzer über ein geeignetes Computersystem verfügen
	\item Es wird angenommen, dass die Nutzer Grundkenntnisse über Klausurerstellung haben
\end{itemize}

\section*{Spezifische Anforderungen}

\subsection*{Externe Schnittstellen}
\subsubsection*{Benutzerschnittstellen}
\begin{itemize}
	\item Die Benutzerschnittstelle soll übersichtlich und intuitiv bedienbar sein
	\item Formulare zur Eingabe von Aufgaben sollen alle erforderlichen Felder enthalten
	\item Die Zusammenstellung von Klausuren soll durch einen Assistenten unterstützt werden
\end{itemize}

\subsection*{Funktionen}
\subsubsection*{Aufgabenverwaltung}
\begin{itemize}
	\item Die Software soll das Erstellen, Bearbeiten und Löschen von Klausuraufgaben ermöglichen
	\item Jede Aufgabe soll folgende Attribute haben: Name, Aufgabentext, Antwortmöglichkeiten (bei geschlossenen Aufgaben), geschätzte Zeit, Modulzugehörigkeit
	\item Jede Aufgabe soll nach folgenden Kriterien kategorisiert werden:
	\begin{itemize}
		\item Bloom'sche Taxonomie (Level 1-6)
		\item Aufgabenformat (offen/geschlossen)
		\item Typ der geschlossenen Aufgabe
	\end{itemize}
	\item Zu jeder Aufgabe soll eine Musterlösung oder richtige Lösung gespeichert werden
\end{itemize}

\subsubsection*{Klausurzusammenstellung}
\begin{itemize}
	\item Die Software soll die Zusammenstellung von Klausuren aus vorhandenen Aufgaben ermöglichen
	\item Aufgaben sollen nach verschiedenen Kriterien filterbar sein
	\item Die Zusammenstellung soll gespeichert werden können
\end{itemize}

\subsubsection*{Druck und Export}
\begin{itemize}
	\item Die Software soll den Druck einzelner Aufgaben ermöglichen
	\item Die Software soll den Druck vollständiger Klausuren ermöglichen
\end{itemize}

\subsection*{Leistungsanforderungen}
\begin{itemize}
	\item Die Software soll auch bei einer großen Anzahl von Aufgaben performant bleiben
	\item Die Reaktionszeit bei Filteroperationen soll angemessen sein
\end{itemize}

\subsection{*Softwaresystemattribute}

\subsubsection*{Zuverlässigkeit}
Die Software soll stabil laufen und keine Datenverluste verursachen.

\subsubsection*{Verfügbarkeit}
Die Software soll als lokale Anwendung jederzeit verfügbar sein.

\subsubsection*{Sicherheit}
Der Zugriff auf die Software soll durch ein Anmeldesystem geschützt sein.

\subsubsection*{Wartbarkeit}
Die Software soll über Import- und Exportfunktionen für die Datensicherung verfügen.

\subsection*{Verifizierung}
Für jede Anforderung sollen Testfälle definiert werden, um die korrekte Implementierung zu überprüfen.

\section*{Anhänge}

\subsection*{Konzeptionelle Modellierung}

\subsubsection*{User Stories mit Akzeptanzkriterien}

\subsubsubsection*{User Story 1}
\textbf{Als} Dozent \textbf{möchte ich} Aufgaben mit allen notwendigen Attributen und Kategorisierungen in der Datenbank speichern können, \textbf{damit} ich eine umfangreiche Sammlung an Klausuraufgaben erstellen kann.

\textbf{Akzeptanzkriterien:}
\begin{itemize}
	\item Alle definierten Attribute können eingegeben werden
	\item Bloom'sche Taxonomie Level kann ausgewählt werden
	\item Aufgabenformat kann festgelegt werden
	\item Bei geschlossenen Aufgaben kann der Typ spezifiziert werden
	\item Die Lösung oder Musterlösung kann gespeichert werden
\end{itemize}

\subsubsubsection*{User Story 2}
\textbf{Als} Dozent \textbf{möchte ich} Klausuren aus vorhandenen Aufgaben nach definierten Kriterien zusammenstellen können, \textbf{damit} ich effizient Prüfungen erstellen kann.

\textbf{Akzeptanzkriterien:}
\begin{itemize}
	\item Aufgaben können nach Modul gefiltert werden
	\item Aufgaben können nach Bloom'scher Taxonomie gefiltert werden
	\item Aufgaben können nach Format und Typ gefiltert werden
	\item Die Gesamtzeit der Klausur wird angezeigt
	\item Die ausgewählten Aufgaben können als Klausur gespeichert werden
\end{itemize}

\subsubsection*{Anwendungsfälle}

\subsubsubsection*{Anwendungsfall 1: Aufgabe erstellen und speichern}

\begin{table}[h]
	\begin{tabularx}{\textwidth}{|l|X|}
		\hline
		\textbf{Name:} & Aufgabe erstellen und speichern \\
		\hline
		\textbf{Akteur:} & Dozent \\
		\hline
		\textbf{Beschreibung:} & Der Dozent erstellt eine neue Klausuraufgabe und speichert sie mit allen erforderlichen Attributen und Kategorisierungen in der Datenbank. \\
		\hline
		\textbf{Vorbedingung:} & Der Dozent ist im System angemeldet. \\
		\hline
		\textbf{Standardablauf:} &
		1. Der Dozent wählt die Option \("\)Neue Aufgabe erstellen\("\). \\
		& 2. Das System zeigt ein Formular zur Eingabe aller Attribute. \\
		& 3. Der Dozent gibt den Namen, Aufgabentext und ggf. Antwortmöglichkeiten ein. \\
		& 4. Der Dozent wählt die Zugehörigkeit zu einem Modul. \\
		& 5. Der Dozent schätzt die Bearbeitungszeit. \\
		& 6. Der Dozent ordnet die Aufgabe einem kognitiven Level der Bloom'schen Taxonomie zu. \\
		& 7. Der Dozent legt das Aufgabenformat fest (offen/geschlossen). \\
		& 8. Bei geschlossenen Aufgaben wählt der Dozent den Typ. \\
		& 9. Der Dozent gibt die Lösung oder Musterlösung ein. \\
		& 10. Der Dozent speichert die Aufgabe. \\
		& 11. Das System bestätigt die erfolgreiche Speicherung. \\
		\hline
		\textbf{Alternative Abläufe:} & Wenn der Dozent nicht alle erforderlichen Felder ausfüllt, zeigt das System eine Fehlermeldung. \\
		\hline
		\textbf{Nachbedingung:} & Die neue Aufgabe ist in der Datenbank gespeichert. \\
		\hline
	\end{tabularx}
\end{table}

\subsubsubsection*{Anwendungsfall 2: Klausur zusammenstellen}

\begin{table}[h]
	\begin{tabularx}{\textwidth}{|l|X|}
		\hline
		\textbf{Name:} & Klausur zusammenstellen \\
		\hline
		\textbf{Akteur:} & Dozent \\
		\hline
		\textbf{Beschreibung:} & Der Dozent stellt aus vorhandenen Aufgaben eine Klausur zusammen. \\
		\hline
		\textbf{Vorbedingung:} & Der Dozent ist im System angemeldet und es sind Aufgaben in der Datenbank vorhanden. \\
		\hline
		\textbf{Standardablauf:} &
		1. Der Dozent wählt die Option \("\)Klausur erstellen\("\). \\
		& 2. Das System zeigt die Filtermöglichkeiten für Aufgaben. \\
		& 3. Der Dozent wählt das gewünschte Modul. \\
		& 4. Der Dozent filtert nach weiteren Kriterien (Bloom'sche Taxonomie, Format, Typ). \\
		& 5. Das System zeigt die passenden Aufgaben. \\
		& 6. Der Dozent wählt Aufgaben aus und fügt sie zur Klausur hinzu. \\
		& 7. Das System berechnet die Gesamtzeit der Klausur. \\
		& 8. Der Dozent speichert die zusammengestellte Klausur. \\
		& 9. Der Dozent kann die Klausur ausdrucken oder als Datei exportieren. \\
		\hline
		\textbf{Alternative Abläufe:} & Wenn keine passenden Aufgaben vorhanden sind, erhält der Dozent eine entsprechende Meldung. \\
		\hline
		\textbf{Nachbedingung:} & Die Klausur ist zusammengestellt und kann verwendet werden. \\
		\hline
	\end{tabularx}
\end{table}

\subsubsection*{Architekturdesign auf konzeptioneller Ebene}

\begin{itemize}
	\item \textbf{Präsentationsschicht:}
	\begin{itemize}
		\item Benutzeroberfläche für Dozenten
		\item Formulare zur Eingabe von Aufgaben
		\item Übersichten und Filtermöglichkeiten
		\item Druckvorschau
	\end{itemize}
	\item \textbf{Anwendungsschicht:}
	\begin{itemize}
		\item Aufgabenverwaltung
		\item Klausurzusammenstellung
		\item Filterlogik
		\item Druckaufbereitung
	\end{itemize}
	\item \textbf{Datenhaltungsschicht:}
	\begin{itemize}
		\item Datenbank für Aufgaben und deren Attribute
		\item Datenbank für erstellte Klausuren
		\item Import/Export-Funktionalität
	\end{itemize}
\end{itemize}


% ############################################################################
% CONTENT ENDS HERE
% ############################################################################

\end{document}
