\section*{Revisionshistorie}
\begin{center}
	\begin{tabular}{|l|l|l|p{8cm}|}
		\hline
		\textbf{Datum} & \textbf{Version} & \textbf{Autor} & \textbf{Beschreibung} \\
		\hline
		&  &  &  \\
		\hline
		&  &  &  \\
		\hline
	\end{tabular}
\end{center}

\section*{Definitionen}
\textcolor[gray]{0.7}{
	Definitionen für Wörter oder Ausdrücke, die eine besondere Bedeutung über normale Wörterbuchdefinitionen hinaus haben.
}
\begin{center}
	\begin{tabular}{|l|l|l|p{8cm}|}
		\hline
		\textbf{Begriff} & \textbf{Definition} \\
		\hline
		&  \\
		\hline
		&  \\
		\hline
	\end{tabular}
\end{center}

\section*{Akronyme und Abkürzungen}
\textcolor[gray]{0.7}{
	Ausschreiben oder Definition aller im Dokument verwendeten Akronyme und Abkürzungen
}
\begin{center}
	\begin{tabular}{|l|l|l|p{8cm}|}
		\hline
		\textbf{Akronym/Abkürzung} & \textbf{Definition} \\
		\hline
		&  \\
		\hline
		&  \\
		\hline
	\end{tabular}
\end{center}

\section*{Einleitung}

\subsection*{Zweck}
\textcolor[gray]{0.7}{
	Beschreibung des Zwecks der zu spezifizierenden Software.
	Erläuterung, wofür diese Spezifikation verwendet wird und an wen sie sich richtet.
}

\subsection*{Umfang}
\textcolor[gray]{0.7}{
	Beschreibung des Umfangs der betreffenden Software.
	Identifikation der Software-Produkte nach Namen.
	Erklärung, was die Software-Produkte tun werden.
	Beschreibung der Anwendung der spezifizierten Software.
}

\subsection*{Produktperspektive}
\textcolor[gray]{0.7}{
	Definition der Beziehung des Systems zu anderen verwandten Produkten.
	Wenn das Produkt Teil eines größeren Systems ist, Beschreibung der Beziehung.
	Identifikation der Schnittstellen zwischen dem Produkt und dem größeren System.
}

\subsubsection*{Systemschnittstellen}
\textcolor[gray]{0.7}{
	Auflistung jeder Systemschnittstelle und Identifikation der Funktionalität.
}

\subsubsection*{Benutzerschnittstellen}
\textcolor[gray]{0.7}{
	Spezifikation der logischen Merkmale jeder Schnittstelle zwischen Software und Benutzern.
}

\subsubsection*{Betriebsabläufe}
\textcolor[gray]{0.7}{
	Spezifikation der normalen und speziellen Operationen, die vom Benutzer benötigt werden.
}


\subsection*{Produktfunktionen}
\textcolor[gray]{0.7}{
	Zusammenfassung der Hauptfunktionen, die die Software ausführen wird.
	Organisation der Funktionen, um die Liste für den Auftraggeber verständlich zu machen.
	Verwendung textueller oder grafischer Methoden, um verschiedene Funktionen und Beziehungen darzustellen.
}

\subsection*{Benutzermerkmale}
\textcolor[gray]{0.7}{
	Beschreibung der allgemeinen Merkmale der beabsichtigten Benutzer.
	Bildungsniveau, Erfahrung, technisches Fachwissen, mögliche Behinderungen und andere Faktoren.
}

\subsection*{Einschränkungen}
\textcolor[gray]{0.7}{
	Allgemeine Beschreibung von Faktoren, die die Optionen des Entwicklers einschränken.
	Regulatorische Richtlinien, Hardware-Beschränkungen, Schnittstellen zu anderen Anwendungen,
	Parallelbetriebs-Anforderungen, Sicherheits- und Schutzanforderungen.
}

\subsection*{Annahmen und Abhängigkeiten}
\textcolor[gray]{0.7}{
	Auflistung von Faktoren, die die in der SRS angegebenen Anforderungen beeinflussen.
	Annahmen über die Verfügbarkeit von Ressourcen, Komponenten usw.
	Abhängigkeiten von externen Faktoren.
}

\section*{Spezifische Anforderungen}

\subsection*{Externe Schnittstellen}
\textcolor[gray]{0.7}{
	Definition aller Ein- und Ausgaben des Softwaresystems.
	Name des Elements, Beschreibung des Zwecks, Quelle der Eingabe oder Ziel der Ausgabe,
	Gültigkeitsbereich, Genauigkeit oder Toleranz, Maßeinheiten, Timing, Datenformate.
}

\subsection*{Funktionen}
\textcolor[gray]{0.7}{
	Definition der grundlegenden Aktionen der Software.
	Gültigkeitsprüfungen der Eingaben, genaue Abfolge von Operationen,
	Reaktionen auf abnormale Situationen, Auswirkung von Parametern,
	Beziehung von Ausgaben zu Eingaben.
}

\subsection*{Qualitätsanforderungen}
\textcolor[gray]{0.7}{
	Definition von Gebrauchstauglichkeits- und Qualitätsanforderungen.
	Messbare Kriterien für Effektivität, Effizienz, Zufriedenheit und Vermeidung von Schäden.
}

\subsection*{Leistungsanforderungen}
\textcolor[gray]{0.7}{
	Spezifikation sowohl statischer als auch dynamischer numerischer Anforderungen.
	Statische Anforderungen (z.B. Anzahl von Terminals, Benutzern, zu verarbeitende Daten).
	Dynamische Anforderungen (z.B. Transaktionen/Aufgaben pro Zeiteinheit).
}

\subsection*{Designeinschränkungen}
\textcolor[gray]{0.7}{
	Spezifikation von Einschränkungen für das Systemdesign, die durch externe Standards auferlegt werden.
	Einhaltung von Standards, Hardware- und Softwarebeschränkungen.
}

\subsection*{Standardkonformität}
\textcolor[gray]{0.7}{
	Spezifikation von Anforderungen, die sich aus bestehenden Standards oder Vorschriften ergeben.
	Berichtsformat, Datenbenennung, Buchführungsverfahren, Prüfpfadverfolgung.
}

\subsection*{Softwaresystemattribute}
\textcolor[gray]{0.7}{
	Spezifikation erforderlicher Attribute des Softwareprodukts.
}

\subsubsection*{Zuverlässigkeit}
\textcolor[gray]{0.7}{
	Spezifikation von Faktoren, die erforderlich sind, um die erforderliche Zuverlässigkeit zu gewährleisten.
}

\subsubsection*{Verfügbarkeit}
\textcolor[gray]{0.7}{
	Spezifikation von Faktoren, die erforderlich sind, um ein definiertes Verfügbarkeitsniveau zu garantieren.
}

\subsubsection*{Sicherheit}
\textcolor[gray]{0.7}{
	Spezifikation von Anforderungen zum Schutz der Software vor versehentlichem oder böswilligem Zugriff,
	Nutzung, Modifikation, Zerstörung oder Offenlegung.
}

\subsubsection*{Wartbarkeit}
\textcolor[gray]{0.7}{
	Spezifikation von Attributen der Software, die sich auf die Wartungsfreundlichkeit beziehen.
}

\subsubsection*{Portabilität}
\textcolor[gray]{0.7}{
	Spezifikation von Attributen, die sich auf die Übertragbarkeit der Software in andere Umgebungen beziehen.
}

\subsection*{Verifizierung}
\textcolor[gray]{0.7}{
	Bereitstellung der Verifizierungsansätze und -methoden, die zur Qualifizierung der Software geplant sind.
	Verifizierungsansätze für jede Anforderung, Testmethoden, Validierungskriterien.
}
