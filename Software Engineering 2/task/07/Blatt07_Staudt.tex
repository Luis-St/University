\documentclass[
	fontsize=12pt,          % default font size 12pt
	paper=a4,               % DIN A4 page format
	numbers=noenddot,       % remove dots behind chapter numbers (e.g. 1.5 not 1.5.)
	listof=totoc,           % add list of figures, tables, etc. to ToC
	listof=entryprefix,     % add entry name to figures, tables, etc.
	listof=nochaptergap,    % no chapter gap for figures, tables, etc.
	bibliography=totoc,     % add bibliography to ToC but without a chapter number
	parskip=half            % half line spacing between paragraphs
	openany                 % chapters can start on any page
]{scrbook}
\usepackage{tikz}

% ##################################################
% ENCODING
% ##################################################
\usepackage{cmap}               % PDF character encoding

%\usepackage[T1]{fontenc}        % 8-bit font encoding
%\usepackage[utf8]{inputenc}     % UTF-8 input encoding
%\usepackage[german]{babel} %%%%% Sprache festlegen

\usepackage[utf8]{inputenc}
\usepackage[T1]{fontenc}
\usepackage{lmodern}
\usepackage{ngerman}


% ##################################################
% GENERAL
% ##################################################
\usepackage{scrhack}            % better KOMA adaptions
\usepackage[table]{xcolor}      % color support
\usepackage{chngcntr}           % for renumbering stuff


\usepackage{amsthm}


\usepackage{lipsum}             % lorem ipsum generator (used for example content)


\usepackage{mathtools}


% ##################################################
% PDF SETTINGS
% ##################################################
\usepackage[
	colorlinks=false,
	linkcolor=black,
	citecolor=black,
	filecolor=black,
	urlcolor=black,
	bookmarks=true,
	bookmarksopen=true,
	bookmarksopenlevel=3,
	bookmarksnumbered,
	plainpages=false,
	pdfpagelabels=true,
	hyperfootnotes
]{hyperref}


% ##################################################
% FONTS AND SPACING
% ##################################################
\renewcommand{\familydefault}{\sfdefault}   % default font
\usepackage[onehalfspacing]{setspace}       % default 1.5 line spacing
\raggedbottom   % don't stretch spacing to fit page length

\usepackage{anyfontsize} % use any font size

% font sizes and styles
\addtokomafont{chapter}{\sffamily\large\bfseries}  % chapter heading 
\addtokomafont{section}{\sffamily\normalsize\bfseries}
\addtokomafont{subsection}{\sffamily\normalsize\mdseries}
\addtokomafont{caption}{\sffamily\normalsize\mdseries}

% url font style
\usepackage{relsize}
\renewcommand*{\UrlFont}{\ttfamily\smaller\relax}


% ##################################################
% PAGE FORMATTING
% ##################################################
% Page layout / Seitenränder
\usepackage[
	bindingoffset=0cm,
	inner=2.5cm,
	outer=2.5cm,
	top=3cm,
	bottom=2cm
]{geometry}

% Page header
\usepackage[
	headsepline,        % seperator line beneath page header on normal pages
	plainheadsepline    % seperator line beneath page header on pages like ToC
]{scrlayer-scrpage}
\clearpairofpagestyles                  % clear default settings
\addtokomafont{pagehead}{\normalfont}   % use normal font for page header
\ohead*{\thepage}                       % page number
\ihead*{\leftmark}                      % chapter name


% ##################################################
% IMAGES AND FIGURES
% ##################################################
\usepackage{graphicx}       % support for including images
\graphicspath{{pictures/}}  % default path
\usepackage{float}          % better control over float positions

% simple numbering without chapter
\renewcommand{\thefigure}{\arabic{figure}}
\counterwithout{figure}{chapter}

% wrap text around figures
\usepackage{wrapfig}


% ##################################################
% TABLES
% ##################################################
% multi row and multi column table functionality
\usepackage{booktabs}   % beautiful table style
\usepackage{multirow}

% simple numbering without chapter
\renewcommand{\thetable}{\arabic{table}}
\counterwithout{table}{chapter}


% ##################################################
% SOURCE CODE LISTINGS
% ##################################################
\usepackage{listings}
\usepackage{beramono}   % use a typewriter font which supports bold characters

\renewcommand{\lstlistlistingname}{List of Code Listings}   % 
\renewcommand{\lstlistingname}{Code Listing}
\newcommand{\listoflolentryname}{\lstlistingname}   % prefix for List of Code Listings

% define colors for source code highlighting
\definecolor{codegreen}{rgb}{0,0.6,0}
\definecolor{codegray}{rgb}{0.5,0.5,0.5}
\definecolor{codepurple}{rgb}{0.5,0,0.33}
\definecolor{codepurblue}{rgb}{0.16,0.0,1.0}
\definecolor{backcolour}{rgb}{0.95,0.95,0.92}

% ##################################################
% TABLE OF CONTENTS
% ##################################################
\KOMAoptions{toc=chapterentrydotfill}       % dotted lines for chapters
\addtokomafont{chapterentry}{\normalfont}   % use normal font for chapter entries
\setuptoc{toc}{totoc}                       % add ToC to ToC

% spacing
\DeclareTOCStyleEntry[beforeskip=0cm]{chapter}{chapter}
\DeclareTOCStyleEntry[beforeskip=0cm]{section}{section}
\DeclareTOCStyleEntry[beforeskip=0cm]{default}{subsection}

% colons after entry names
\BeforeStartingTOC[lof]{\def\autodot{:}}
\BeforeStartingTOC[lot]{\def\autodot{:}}
\BeforeStartingTOC[lol]{\def\autodot{:}}


% ##################################################
% BIBLIOGRAPHY
% ##################################################
\iffalse
\usepackage{csquotes} % context sensitive quotation
\setlength\bibitemsep{.5\baselineskip} % increase spacing between entries
\setcounter{biburlnumpenalty}{9000} % break URLs on numbers
\setcounter{biburllcpenalty}{9000}  % break URLs on lower case letters
\setcounter{biburlucpenalty}{9000}  % break URLs on upper case letters

\fi

% ##################################################
% ABBREVIATIONS
% ##################################################
\usepackage[printonlyused]{acronym}


% ##################################################
% APPENDIX
% ##################################################
\usepackage[title,titletoc]{appendix}

% appendix chapter
\newcommand{\appendixchapter}[1]{
\cleardoublepage
\pagenumbering{arabic}
\renewcommand{\thepage}{\thechapter-\arabic{page}}
\chapter{#1}
}

% insert monthly report pdf as picture in order to keep page header
\newcommand{\monthlyreport}[2]{
\section{#1}
\centering
\includegraphics[trim=55 35 55 35,clip,width=1\textwidth]{#2}
\clearpage
}


% ##################################################
% Theoreme
% ##################################################

% Umgebung fuer Beispiele
\newtheorem{beispiel}{Beispiel}

% Umgebung fuer These
\newtheorem{these}{These}

% Umgebung fuer Definitionen
\newtheorem{definition}{Definition}


% ##################################################
% MISC
% ##################################################
% better referencing of images, tables, etc.
\usepackage[nameinlink, noabbrev]{cleveref}


% ############################################################################
% WIE MACH ICH DAS HIER?
% 
% 1. Schreibe im "titlepage" Abschnitt den Titel in das element mit "\fontsize{22}{22}"
% 2. Aus obsidian raus kopieren mit "Copy to LaTeX"
% 3. Zwischen "CONTENT STARTS HERE" und "CONTENT ENDS HERE" den Inhalt einfügen
% 4. Sections anpassen. Damit man ne gescheite Chapter > Section > Subsection Struktur hat
% 5. Leere Seite nach Titelseite am Anfang mit {\let\cleardoublepage\relax \chapter{Erstes Kapitel}} verhindern
%
% TABELLEN
% Müssen extra gemacht werden, da obsidian das nicht unterstützt
% Kann ich nur empfehlen: https://tableconvert.com/markdown-to-latex
%
% BILDER
% Muss man vermutlich auch extra machen, hab ich aber noch nicht probiert
% ############################################################################

\begin{document}

\begin{titlepage}
	\pagestyle{empty}
	
	% HFU Logo
	\begin{flushright}
		\begin{figure}[ht]
			\flushright
			\includegraphics[height=2cm]{../../for_latex/hfu}
		\end{figure}
	\end{flushright}
	
	\begin{center}
		\vspace{3cm}
		
		{\fontsize{22}{22} \selectfont \textbf{Blatt 7}}\\[5mm]
		{\fontsize{18}{18} \selectfont Software Engineering 2}
		
		\vspace{12cm}
		
		{\fontsize{14}{14} \selectfont Luis Staudt}
	\end{center}
\end{titlepage}

% ############################################################################
% CONTENT STARTS HERE
% ############################################################################

{\let\cleardoublepage\relax \chapter*{Aufgabe 1}}

\section*{Server-Client Architektur}
\begin{figure}[!htbpf]
	\centering
	\includegraphics[width=0.8\textwidth]{images/epa/client_server_architecture}
\end{figure}
\newpage


\section*{Logische Sicht}
\begin{figure}[!htbpf]
	\centering
	\includegraphics[angle=90, height=0.8\textheight]{images/epa/logical_view_1}
\end{figure}
\newpage

\begin{figure}[!htbpf]
	\centering
	\includegraphics[angle=90, height=0.9\textheight]{images/epa/logical_view_2}
\end{figure}
\newpage


\subsection*{Datenflussdiagramm}
\begin{figure}[!htbpf]
	\centering
	\includegraphics[angle=90, height=0.9\textheight]{images/epa/data_river_diagram}
\end{figure}
\newpage

\section*{Physische Sicht}
\begin{figure}[!htbpf]
	\centering
	\includegraphics[width=0.8\textwidth]{images/epa/physical_view}
\end{figure}
\newpage


{\let\cleardoublepage\relax \chapter*{Aufgabe 2}}

\section*{Schichtenarchitektur}
\begin{figure}[!htbpf]
	\centering
	\includegraphics[width=0.8\textwidth]{images/gka/layer_architecture}
\end{figure}
\newpage

\section*{MVC-Architektur}
\begin{figure}[!htbpf]
	\centering
	\includegraphics[width=0.8\textwidth]{images/gka/mvc_architecture}
\end{figure}
\newpage

\section*{Datenflussdiagramm - Schichtenarchitektur}
\begin{figure}[!htbpf]
	\centering
	\includegraphics[angle=90, height=0.9\textheight]{images/gka/data_river_diagram_layer_architecture}
\end{figure}
\newpage

\section*{Datenflussdiagramm - MVC-Architektur}
\begin{figure}[!htbpf]
	\centering
	\includegraphics[angle=90, height=0.9\textheight]{images/gka/data_river_diagram_mvc_architecture}
\end{figure}
\newpage

\section*{Detaillierter Vergleich der Architekturen}
\begin{table}[htbp]
	\centering
	\begin{tabular}{|p{3cm}|p{5cm}|p{5cm}|}
		\hline
		\textbf{Kriterium} & \textbf{Schichtenarchitektur} & \textbf{MVC-Architektur} \\
		\hline
		\textbf{Kontrollfluss}
		
		Wie wird die Steuerung zwischen den Komponenten verteilt?
		&
		Hierarchisch, streng gerichtet von oben nach unten
		
		Jede Schicht dient der darüberliegenden &
		Zyklisch, dreieckig
		
		Controller koordiniert zwischen View und Model \\
		\hline
		\textbf{Abhängigkeiten}
		
		Wie stark sind die Komponenten voneinander abhängig?
		&
		Unidirektional, vorhersehbar
		
		Jede Schicht hängt nur von der direkt darunter liegenden Schicht ab &
		Bidirektional, komplexer
		
		View kann Model beobachten
		
		Controller hängt von View und Model ab \\
		\hline
		\textbf{Modularität}
		
		Austauschbarkeit, Erweiterbarkeit &
		Gut für horizontale Modularität
		
		Schichten können ausgetauscht werden &
		Besser für UI-bezogene Modularität
		
		Views können leicht geändert werden \\
		\hline
		\textbf{Testbarkeit}
		
		Wie gut lassen sich die einzelnen Komponenten testen?
		&
		Gut für Unit-Tests der Geschäftslogik
		
		Mocks für untere Schichten nötig
		
		UI-Tests schwieriger isolierbar &
		Sehr gut für isolierte Tests
		
		Model kann ohne UI getestet werden
		
		Controller separat testbar \\
		\hline
		\textbf{Wartbarkeit}
		
		Wie einfach ist es, das System zu warten und zu ändern?
		&
		Gut für große Änderungen in einer Schicht
		
		Funktionen gut kategorisierbar
		
		Klare Verantwortlichkeiten je Schicht &
		Sehr gut für UI-Änderungen
		
		Komplexere Struktur bei großen Anwendungen
		
		Weniger Doppelcode bei Features \\
		\hline
	\end{tabular}
	\caption{Vergleich zwischen Schichtenarchitektur und MVC-Architektur}
\end{table}
\newpage


{\let\cleardoublepage\relax \chapter*{Aufgabe 3}}

Fahrzeug-Diagnosesystem in einer modernen Autowerkstatt

\section*{Beschreibung}

\subsection*{Zentrale Datenhaltung}
Fahrzeugdaten aus verschiedensten Quellen (Bordsensoren, Fehlerspeicher, historische Reparaturdaten, Herstellerdatenbanken) werden in einer zentralen Struktur gespeichert.

\subsection*{Unabhängige Analysekomponenten}
Verschiedene Diagnosemodule können unabhängig voneinander auf dieselben Daten zugreifen.

\subsection*{Ereignis- und zustandsbasierte Verarbeitung}
Das System kann sowohl auf neue eingehende Daten (wie bei einer traditionellen Datenbank) als auch auf bestimmte Datenzustände (wie bei einem Blackboard) reagieren.

\subsection*{Erweiterbarkeit}
Neue Diagnosemodule können ohne Änderung bestehender Komponenten hinzugefügt werden.

\subsection*{Integration verschiedener Expertisen}
Ähnlich wie bei Blackboard-Systemen können verschiedene \("\)Knowledge Sources\("\) (Fahrwerksdiagnose, Motordiagnose, elektrische Systeme) unabhängig arbeite

\newpage
\section*{Repository-Architektur}
\begin{figure}[!htbpf]
	\centering
	\includegraphics[width=0.8\textwidth]{images/repository_architecture}
\end{figure}
\newpage


{\let\cleardoublepage\relax \chapter*{Aufgabe 4}}

\section*{Pipe and Filter Architektur}
\begin{figure}[!htbpf]
	\centering
	\includegraphics[width=0.8\textwidth]{images/pipe_and_filter_logical_view}
\end{figure}
\newpage

\section*{Datenflussdiagramm - Pipe and Filter Architektur}
\begin{figure}[!htbpf]
	\centering
	\includegraphics[angle=90, height=0.9\textheight]{images/data_river_diagram_pipe_and_filter}
\end{figure}
\newpage

% ############################################################################
% CONTENT ENDS HERE
% ############################################################################

\end{document}
