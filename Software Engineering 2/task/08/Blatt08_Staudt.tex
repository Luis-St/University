\documentclass[
	fontsize=12pt,          % default font size 12pt
	paper=a4,               % DIN A4 page format
	numbers=noenddot,       % remove dots behind chapter numbers (e.g. 1.5 not 1.5.)
	listof=totoc,           % add list of figures, tables, etc. to ToC
	listof=entryprefix,     % add entry name to figures, tables, etc.
	listof=nochaptergap,    % no chapter gap for figures, tables, etc.
	bibliography=totoc,     % add bibliography to ToC but without a chapter number
	parskip=half            % half line spacing between paragraphs
	openany                 % chapters can start on any page
]{scrbook}
\usepackage{tikz}

% ##################################################
% ENCODING
% ##################################################
\usepackage{cmap}               % PDF character encoding

%\usepackage[T1]{fontenc}        % 8-bit font encoding
%\usepackage[utf8]{inputenc}     % UTF-8 input encoding
%\usepackage[german]{babel} %%%%% Sprache festlegen

\usepackage[utf8]{inputenc}
\usepackage[T1]{fontenc}
\usepackage{lmodern}
\usepackage{ngerman}


% ##################################################
% GENERAL
% ##################################################
\usepackage{scrhack}            % better KOMA adaptions
\usepackage[table]{xcolor}      % color support
\usepackage{chngcntr}           % for renumbering stuff


\usepackage{amsthm}


\usepackage{lipsum}             % lorem ipsum generator (used for example content)


\usepackage{mathtools}


% ##################################################
% PDF SETTINGS
% ##################################################
\usepackage[
	colorlinks=false,
	linkcolor=black,
	citecolor=black,
	filecolor=black,
	urlcolor=black,
	bookmarks=true,
	bookmarksopen=true,
	bookmarksopenlevel=3,
	bookmarksnumbered,
	plainpages=false,
	pdfpagelabels=true,
	hyperfootnotes
]{hyperref}


% ##################################################
% FONTS AND SPACING
% ##################################################
\renewcommand{\familydefault}{\sfdefault}   % default font
\usepackage[onehalfspacing]{setspace}       % default 1.5 line spacing
\raggedbottom   % don't stretch spacing to fit page length

\usepackage{anyfontsize} % use any font size

% font sizes and styles
\addtokomafont{chapter}{\sffamily\large\bfseries}  % chapter heading 
\addtokomafont{section}{\sffamily\normalsize\bfseries}
\addtokomafont{subsection}{\sffamily\normalsize\mdseries}
\addtokomafont{caption}{\sffamily\normalsize\mdseries}

% url font style
\usepackage{relsize}
\renewcommand*{\UrlFont}{\ttfamily\smaller\relax}


% ##################################################
% PAGE FORMATTING
% ##################################################
% Page layout / Seitenränder
\usepackage[
	bindingoffset=0cm,
	inner=2.5cm,
	outer=2.5cm,
	top=3cm,
	bottom=2cm
]{geometry}

% Page header
\usepackage[
	headsepline,        % seperator line beneath page header on normal pages
	plainheadsepline    % seperator line beneath page header on pages like ToC
]{scrlayer-scrpage}
\clearpairofpagestyles                  % clear default settings
\addtokomafont{pagehead}{\normalfont}   % use normal font for page header
\ohead*{\thepage}                       % page number
\ihead*{\leftmark}                      % chapter name


% ##################################################
% IMAGES AND FIGURES
% ##################################################
\usepackage{graphicx}       % support for including images
\graphicspath{{pictures/}}  % default path
\usepackage{float}          % better control over float positions

% simple numbering without chapter
\renewcommand{\thefigure}{\arabic{figure}}
\counterwithout{figure}{chapter}

% wrap text around figures
\usepackage{wrapfig}


% ##################################################
% TABLES
% ##################################################
% multi row and multi column table functionality
\usepackage{booktabs}   % beautiful table style
\usepackage{multirow}

% simple numbering without chapter
\renewcommand{\thetable}{\arabic{table}}
\counterwithout{table}{chapter}


% ##################################################
% SOURCE CODE LISTINGS
% ##################################################
\usepackage{listings}
\usepackage{beramono}   % use a typewriter font which supports bold characters

\renewcommand{\lstlistlistingname}{List of Code Listings}   % 
\renewcommand{\lstlistingname}{Code Listing}
\newcommand{\listoflolentryname}{\lstlistingname}   % prefix for List of Code Listings

% define colors for source code highlighting
\definecolor{codegreen}{rgb}{0,0.6,0}
\definecolor{codegray}{rgb}{0.5,0.5,0.5}
\definecolor{codepurple}{rgb}{0.5,0,0.33}
\definecolor{codepurblue}{rgb}{0.16,0.0,1.0}
\definecolor{backcolour}{rgb}{0.95,0.95,0.92}

% ##################################################
% TABLE OF CONTENTS
% ##################################################
\KOMAoptions{toc=chapterentrydotfill}       % dotted lines for chapters
\addtokomafont{chapterentry}{\normalfont}   % use normal font for chapter entries
\setuptoc{toc}{totoc}                       % add ToC to ToC

% spacing
\DeclareTOCStyleEntry[beforeskip=0cm]{chapter}{chapter}
\DeclareTOCStyleEntry[beforeskip=0cm]{section}{section}
\DeclareTOCStyleEntry[beforeskip=0cm]{default}{subsection}

% colons after entry names
\BeforeStartingTOC[lof]{\def\autodot{:}}
\BeforeStartingTOC[lot]{\def\autodot{:}}
\BeforeStartingTOC[lol]{\def\autodot{:}}


% ##################################################
% BIBLIOGRAPHY
% ##################################################
\iffalse
\usepackage{csquotes} % context sensitive quotation
\setlength\bibitemsep{.5\baselineskip} % increase spacing between entries
\setcounter{biburlnumpenalty}{9000} % break URLs on numbers
\setcounter{biburllcpenalty}{9000}  % break URLs on lower case letters
\setcounter{biburlucpenalty}{9000}  % break URLs on upper case letters

\fi

% ##################################################
% ABBREVIATIONS
% ##################################################
\usepackage[printonlyused]{acronym}


% ##################################################
% APPENDIX
% ##################################################
\usepackage[title,titletoc]{appendix}

% appendix chapter
\newcommand{\appendixchapter}[1]{
\cleardoublepage
\pagenumbering{arabic}
\renewcommand{\thepage}{\thechapter-\arabic{page}}
\chapter{#1}
}

% insert monthly report pdf as picture in order to keep page header
\newcommand{\monthlyreport}[2]{
\section{#1}
\centering
\includegraphics[trim=55 35 55 35,clip,width=1\textwidth]{#2}
\clearpage
}


% ##################################################
% Theoreme
% ##################################################

% Umgebung fuer Beispiele
\newtheorem{beispiel}{Beispiel}

% Umgebung fuer These
\newtheorem{these}{These}

% Umgebung fuer Definitionen
\newtheorem{definition}{Definition}


% ##################################################
% MISC
% ##################################################
% better referencing of images, tables, etc.
\usepackage[nameinlink, noabbrev]{cleveref}


% ############################################################################
% WIE MACH ICH DAS HIER?
% 
% 1. Schreibe im "titlepage" Abschnitt den Titel in das element mit "\fontsize{22}{22}"
% 2. Aus obsidian raus kopieren mit "Copy to LaTeX"
% 3. Zwischen "CONTENT STARTS HERE" und "CONTENT ENDS HERE" den Inhalt einfügen
% 4. Sections anpassen. Damit man ne gescheite Chapter > Section > Subsection Struktur hat
% 5. Leere Seite nach Titelseite am Anfang mit {\let\cleardoublepage\relax \chapter{Erstes Kapitel}} verhindern
%
% TABELLEN
% Müssen extra gemacht werden, da obsidian das nicht unterstützt
% Kann ich nur empfehlen: https://tableconvert.com/markdown-to-latex
%
% BILDER
% Muss man vermutlich auch extra machen, hab ich aber noch nicht probiert
% ############################################################################

\begin{document}

\begin{titlepage}
	\pagestyle{empty}
	
	% HFU Logo
	\begin{flushright}
		\begin{figure}[ht]
			\flushright
			\includegraphics[height=2cm]{../../for_latex/hfu}
		\end{figure}
	\end{flushright}
	
	\begin{center}
		\vspace{3cm}
		
		{\fontsize{22}{22} \selectfont \textbf{Blatt 8}}\\[5mm]
		{\fontsize{18}{18} \selectfont Software Engineering 2}
		
		\vspace{12cm}
		
		{\fontsize{14}{14} \selectfont Luis Staudt}
	\end{center}
\end{titlepage}

% ############################################################################
% CONTENT STARTS HERE
% ############################################################################

{\let\cleardoublepage\relax \chapter*{Aufgabe 1}}

\section*{Client-Server Architektur mit Microservices}

\subsection*{Überblick der Architektur}

Die vorgeschlagene Architektur folgt dem Microservices-Architekturstil und besteht aus mehreren unabhängigen Services, die über definierte APIs kommunizieren.
Ein API-Gateway dient als zentraler Eingangspunkt für alle Client-Anfragen.

\begin{figure}[!htbpf] % Mermaid-Diagramm 1: Überblick der Microservices-Architektur (mermaid_diagram_1)
	\centering
	\includegraphics[width=\textwidth]{images/mermaid_diagram_1}
	\caption{Überblick der Microservices-Architektur}
\end{figure}

\section*{Designaspekte der Microservices-Architektur}
\subsection*{a) Dekomposition von Microservices}
\subsubsection*{Logische Sicht der Microservices}

\newpage
\begin{figure}[!htbpf] % Mermaid-Diagramm 2: Logische Sicht der Microservices-Dekomposition (mermaid_diagram_2)
	\centering
	\includegraphics[angle=90, height=0.9\textheight]{images/mermaid_diagram_2}
	\caption{Logische Sicht der Microservices-Dekomposition}
\end{figure}

\subsubsection*{Beschreibung der Microservices}

\textbf{Authentication Service:}
Verwaltet die Benutzerauthentifizierung und -autorisierung für alle Clients (Kunden, Händler, Servicepartner).
Stellt JWT-Token aus und validiert Berechtigungen für den Zugriff auf verschiedene Services.
Implementiert Multi-Factor-Authentication für erhöhte Sicherheit.


\textbf{Vehicle Configuration Service:} Verwaltet die Fahrzeugkonfiguration und ermöglicht Kunden die Auswahl von Modellen, Motorisierungen, Farben und Ausstattungsmerkmalen.
Stellt Konfigurations-APIs bereit und validiert die Kompatibilität der gewählten Optionen.
Speichert Konfigurationsdaten und stellt diese anderen Services zur Verfügung.


\textbf{Pricing Service:} Berechnet Echtzeit-Preise basierend auf der Fahrzeugkonfiguration und aktuellen Marktbedingungen.
Verwaltet Preislisten, Rabatte und Aktionen für verschiedene Märkte und Kundengruppen.
Integriert externe Preisdatenquellen und führt dynamische Preisberechnungen durch.


\textbf{Customer Management Service:} Verwaltet alle Kundeninformationen, Profile und Präferenzen zentral.
Stellt APIs für CRUD-Operationen auf Kundendaten bereit und gewährleistet DSGVO-Konformität.
Integriert Kundensegmentierung und personalisierte Empfehlungen.

\textbf{Order Management Service:} Verarbeitet Bestellungen von der Konfiguration bis zur Auslieferung und verwaltet den gesamten Bestellworkflow.
Koordiniert mit anderen Services für Preisberechnung, Lagerverfügbarkeit und Zahlungsabwicklung.
Stellt Bestellstatus-Updates und Tracking-Informationen bereit.

\textbf{Inventory Service:} Verwaltet die Verfügbarkeit von Fahrzeugen, Komponenten und Ersatzteilen in Echtzeit.
Synchronisiert Lagerbestände über verschiedene Standorte und Lieferanten hinweg.
Stellt Verfügbarkeitsprüfungen für Konfigurationen und geschätzte Lieferzeiten bereit.

\textbf{Payment Service:} Verarbeitet alle Zahlungstransaktionen sicher und unterstützt verschiedene Zahlungsmethoden.
Integriert externe Zahlungsdienstleister und gewährleistet PCI-DSS-Konformität.
Verwaltet Zahlungspläne, Raten und Rückerstattungen.

\textbf{Financing Service:} Stellt verschiedene Finanzierungsoptionen bereit und berechnet Raten und Konditionen.
Integriert externe Finanzpartner und führt Bonitätsprüfungen durch.
Verwaltet Leasingverträge und Finanzierungsanträge.

\textbf{Insurance Service:} Bietet Versicherungsoptionen und ermöglicht den Abschluss von Versicherungen direkt im Portal.
Integriert Versicherungspartner und berechnet Prämien basierend auf Fahrzeug und Fahrer.
Verwaltet Versicherungsverträge und Schadensmeldungen.

\textbf{Vehicle Status Service:} Überwacht den Status aller verkauften Fahrzeuge und sammelt Telemetriedaten.
Stellt Echtzeitinformationen über Fahrzeugzustand, Position und Leistungsdaten bereit.
Erkennt potenzielle Probleme und löst präventive Wartungsempfehlungen aus.

\textbf{Maintenance Service:} Verwaltet Wartungspläne, Servicetermine und Reparaturhistorien für alle Fahrzeuge.
Sendet Erinnerungen für anstehende Wartungen und koordiniert mit Servicepartnern.
Stellt Wartungsempfehlungen basierend auf Fahrzeugdaten und -nutzung bereit.

\textbf{OTA Update Service:} Verwaltet Over-The-Air-Softwareupdates für Fahrzeuge und stellt sichere Update-Mechanismen bereit.
Führt Rollout-Strategien durch und überwacht Update-Erfolg.
Ermöglicht Feature-Freischaltungen und Konfigurationsänderungen remote.

\textbf{Service Booking Service:} Ermöglicht die Buchung verschiedener Services wie Helpdesk, Notruf oder Zusatzleistungen.
Verwaltet Service-Kataloge und Verfügbarkeiten von Servicepartnern.
Koordiniert Terminplanung und Ressourcenzuweisung.

\textbf{Notification Service:} Sendet Benachrichtigungen über verschiedene Kanäle (E-Mail, SMS, Push) an Kunden und Partner.
Verwaltet Benachrichtigungsvorlagen und Personalisierung.
Stellt Event-basierte Benachrichtigungen für wichtige Systemereignisse bereit.

\subsection*{b) Kommunikation zwischen Microservices}
\subsubsection*{Synchrone vs. Asynchrone Kommunikation}
Für das Auto Portal wird eine hybride Kommunikationsstrategie vorgeschlagen, die sowohl synchrone als auch asynchrone Kommunikation nutzt.
Synchrone Kommunikation wird für zeitkritische Operationen wie Preisberechnungen und Verfügbarkeitsprüfungen verwendet, da Kunden sofortiges Feedback erwarten.
Asynchrone Kommunikation wird für nicht-zeitkritische Prozesse wie Benachrichtigungen, Wartungsempfehlungen und Datenreplikation eingesetzt.
Diese Strategie optimiert die Benutzererfahrung durch schnelle Antwortzeiten bei kritischen Operationen und verbessert gleichzeitig die Systemresilienz durch lose Kopplung bei unkritischen Prozessen.

\newpage
\begin{figure}[!htbpf] % Mermaid-Diagramm 3: Synchrone Kommunikation - Preisberechnung mit Verfügbarkeitsprüfung (mermaid_diagram_3)
	\centering
	\includegraphics[angle=90, height=0.9\textheight]{images/mermaid_diagram_3}
	\caption{Synchrone Kommunikation - Preisberechnung mit Verfügbarkeitsprüfung}
\end{figure}

\newpage
\begin{figure}[!htbpf] % Mermaid-Diagramm 4: Asynchrone Kommunikation - Event-basierte Bestellverarbeitung (mermaid_diagram_4)
	\centering
	\includegraphics[angle=90, height=0.9\textheight]{images/mermaid_diagram_4}
	\caption{Asynchrone Kommunikation - Event-basierte Bestellverarbeitung}
\end{figure}

\subsubsection*{Direkte vs. Indirekte Kommunikation}
Direkte Kommunikation wird für einfache Request-Response-Szenarien zwischen Services verwendet, bei denen eine klare 1:1-Beziehung besteht.
Indirekte Kommunikation über Message Broker wird für komplexe Workflows eingesetzt, bei denen mehrere Services auf Ereignisse reagieren müssen.
Der Message Broker entkoppelt Services voneinander und ermöglicht eine bessere Skalierbarkeit und Fehlertoleranz.
Pattern wie Publish-Subscribe werden für Ereignisse verwendet, die von mehreren Services verarbeitet werden müssen, während direkte API-Aufrufe für spezifische Datenanfragen genutzt werden.

\begin{figure}[!htbpf] % Mermaid-Diagramm 5: Direkte Kommunikation zwischen Services (mermaid_diagram_5)
	\centering
	\includegraphics[width=\textwidth]{images/mermaid_diagram_5}
	\caption{Direkte Kommunikation zwischen Services}
	\label{fig:direct_comm}
\end{figure}

\begin{figure}[!htbpf] % Mermaid-Diagramm 6: Indirekte Kommunikation über Message Broker (mermaid_diagram_6)
	\centering
	\includegraphics[width=\textwidth]{images/mermaid_diagram_6}
	\caption{Indirekte Kommunikation über Message Broker}
	\label{fig:indirect_comm}
\end{figure}

\subsection*{c) Verteilung und gemeinsame Nutzung von Daten}
\subsubsection*{Datenverwaltungsstrategie}
Das System folgt dem \("\)Database per Service\("\)-Pattern, wobei jeder Microservice seine eigenen Daten verwaltet und andere Services nicht direkt auf diese Datenbank zugreifen können.
Für gemeinsam genutzte Referenzdaten werden dedizierte Services bereitgestellt.
Event Sourcing wird für kritische Geschäftsereignisse verwendet, um Konsistenz zwischen Services zu gewährleisten.
CQRS (Command Query Responsibility Segregation) trennt Lese- und Schreibvorgänge für bessere Performance und Skalierbarkeit.

\begin{figure}[!htbpf] % Mermaid-Diagramm 7: Datenverteilung und -sharing zwischen Microservices (mermaid_diagram_7)
	\centering
	\includegraphics[width=\textwidth]{images/mermaid_diagram_7}
	\caption{Datenverteilung und -sharing zwischen Microservices}
	\label{fig:data_sharing}
\end{figure}

\subsubsection*{Datenklassifizierung}

\textbf{Exklusive Servicedaten:}
\begin{itemize}
	\item Authentication Service: Benutzeranmeldedaten, Passwort-Hashes, Session-Tokens
	\item Customer Service: Persönliche Kundendaten, Präferenzen, Kommunikationshistorie
	\item Payment Service: Zahlungsinformationen, Transaktionsdetails, Kreditkartendaten
	\item Order Service: Bestelldetails, Status, Workflow-Zustand
\end{itemize}

\textbf{Gemeinsam genutzte Daten:}
\begin{itemize}
	\item Vehicle Reference Data: Modelle, Spezifikationen, Kompatibilitätsregeln
	\item Pricing Data: Basispresse, Währungskurse, regionale Aufschläge
	\item Location Data: Händlerstandorte, Servicezentren, Lagerstandorte
	\item Configuration Rules: Gültige Kombinationen, Abhängigkeiten
\end{itemize}

\subsection*{d) Koordination von Microservices}
\subsubsection*{Koordinationsstrategie}
Für die Koordination der Microservices wird eine hybride Strategie aus Choreografie und Orchestration verwendet.
Choreografie wird für einfache, lose gekoppelte Workflows bevorzugt, da sie die Autonomie der Services erhöht und Single Points of Failure vermeidet.
Orchestration wird nur für komplexe Geschäftsprozesse eingesetzt, die eine zentrale Kontrolle erfordern, wie die Bestellabwicklung.
Das API Gateway fungiert als zentraler Koordinationspunkt für Client-Anfragen, während interne Service-zu-Service-Kommunikation hauptsächlich event-driven abläuft.
Service Mesh-Technologien unterstützen bei der Service Discovery und Load Balancing.
\begin{figure}[H]
	\centering
	\begin{tikzpicture}[node distance=2.5cm]
		% API Gateway coordination
		\node (client) [client] {Client};
		\node (gateway) [gateway, below of=client] {API Gateway};
		\node (orchestrator) [service, below of=gateway] {Order Orchestrator};
		
		% Services
		\node (inventory) [service, left of=orchestrator, xshift=-2.5cm] {Inventory\\Service};
		\node (pricing) [service, above left of=orchestrator, xshift=-1.5cm, yshift=1cm] {Pricing\\Service};
		\node (payment) [service, right of=orchestrator, xshift=2.5cm] {Payment\\Service};
		\node (customer) [service, above right of=orchestrator, xshift=1.5cm, yshift=1cm] {Customer\\Service};
		
		% Orchestration flow
		\draw [arrow] (client) -- node[left] {Place Order} (gateway);
		\draw [arrow] (gateway) -- node[left] {Create Order} (orchestrator);
		
		\draw [arrow] (orchestrator) -- node[above] {1. Check Stock} (inventory);
		\draw [arrow] (orchestrator) -- node[above left] {2. Calculate Price} (pricing);
		\draw [arrow] (orchestrator) -- node[above right] {3. Validate Customer} (customer);
		\draw [arrow] (orchestrator) -- node[above] {4. Process Payment} (payment);
		
		% Return path
		\draw [arrow] (orchestrator) -- node[right] {Order Confirmation} (gateway);
		\draw [arrow] (gateway) -- node[right] {Response} (client);
	
	\end{tikzpicture}
	\caption{Orchestration Pattern für komplexe Bestellabwicklung}
	\label{fig:orchestration}
\end{figure}

\begin{figure}[H]
	\centering
	\begin{tikzpicture}[node distance=2cm]
		% Event-driven choreography
		\node (vehicle) [service] {Vehicle Status\\Service};
		\node (eventbus) [gateway, right of=vehicle, xshift=3cm] {Event Bus};
		
		\node (maintenance) [service, above right of=eventbus, xshift=2cm, yshift=1.5cm] {Maintenance\\Service};
		\node (notification) [service, right of=eventbus, xshift=3cm] {Notification\\Service};
		\node (customer) [service, below right of=eventbus, xshift=2cm, yshift=-1.5cm] {Customer\\Service};
		\node (dealer) [service, below of=eventbus, yshift=-2cm] {Dealer\\Service};
		
		% Event flow
		\draw [message] (vehicle) -- node[above] {Maintenance Due Event} (eventbus);
		\draw [message] (eventbus) -- node[above right] {Schedule Service} (maintenance);
		\draw [message] (eventbus) -- node[above] {Send Alert} (notification);
		\draw [message] (eventbus) -- node[below right] {Update Profile} (customer);
		\draw [message] (eventbus) -- node[below] {Dealer Notification} (dealer);
	
	\end{tikzpicture}
	\caption{Choreography Pattern für Event-basierte Koordination}
	\label{fig:choreography}
\end{figure}


\subsection*{e) Fehlermanagement und Überwachung}
\subsubsection*{Überwachungsstrategie}
Das Überwachungssystem implementiert ein zentralisiertes Monitoring-Dashboard, das Metriken, Logs und Traces von allen Microservices sammelt und korreliert.
Health Checks überwachen die Verfügbarkeit und Performance jedes Services kontinuierlich.
Distributed Tracing verfolgt Anfragen durch das gesamte System und identifiziert Bottlenecks.
Circuit Breaker Pattern verhindert Kaskadenausfälle bei Service-Fehlern.
Alerting-Mechanismen benachrichtigen Operations-Teams bei kritischen Problemen automatisch.
\begin{figure}[H]
	\centering
	\begin{tikzpicture}[node distance=2cm]
		% Monitoring Infrastructure
		\node (dashboard) [gateway] {Monitoring Dashboard};
		
		% Monitoring Components
		\node (metrics) [service, above left of=dashboard, xshift=-2cm, yshift=1.5cm] {Metrics\\Collector};
		\node (logs) [service, above of=dashboard, yshift=1.5cm] {Log\\Aggregator};
		\node (traces) [service, above right of=dashboard, xshift=2cm, yshift=1.5cm] {Trace\\Collector};
		\node (alerts) [service, below of=dashboard, yshift=-1.5cm] {Alert\\Manager};
		
		% Microservices being monitored
		\node (service1) [service, below left of=metrics, yshift=-2cm] {Config\\Service};
		\node (service2) [service, below of=logs, yshift=-2cm] {Order\\Service};
		\node (service3) [service, below right of=traces, yshift=-2cm] {Payment\\Service};
		
		% Data stores
		\node (metricsdb) [database, left of=metrics, xshift=-2cm] {Metrics\\DB};
		\node (logsdb) [database, above of=logs, yshift=1cm] {Log\\Store};
		\node (tracesdb) [database, right of=traces, xshift=2cm] {Trace\\DB};
		
		% Health Check Service
		\node (health) [service, right of=dashboard, xshift=3cm] {Health Check\\Service};
		
		% Connections
		\draw [arrow] (metrics) -- (dashboard);
		\draw [arrow] (logs) -- (dashboard);
		\draw [arrow] (traces) -- (dashboard);
		\draw [arrow] (dashboard) -- (alerts);
		\draw [arrow] (dashboard) -- (health);
		
		\draw [arrow] (metrics) -- (metricsdb);
		\draw [arrow] (logs) -- (logsdb);
		\draw [arrow] (traces) -- (tracesdb);
		
		% Monitoring connections
		\draw [message] (service1) -- (metrics);
		\draw [message] (service1) -- (logs);
		\draw [message] (service2) -- (logs);
		\draw [message] (service2) -- (traces);
		\draw [message] (service3) -- (traces);
		\draw [message] (service3) -- (metrics);
		
		% Health checks
		\draw [message] (health) -- (service1);
		\draw [message] (health) -- (service2);
		\draw [message] (health) -- (service3);
	
	\end{tikzpicture}
	\caption{Integriertes Überwachungs-Dashboard für Microservices}
	\label{fig:monitoring}
\end{figure}

\subsubsection*{Fehlerbehandlungsstrategien}
\begin{enumerate}
	\item \textbf{Circuit Breaker Pattern:} Verhindert wiederholte Aufrufe fehlgeschlagener Services und ermöglicht schnelle Wiederherstellung.
	\item \textbf{Retry mit Exponential Backoff:} Automatische Wiederholung fehlgeschlagener Anfragen mit zunehmenden Wartezeiten.
	\item \textbf{Bulkhead Pattern:} Isolierung kritischer Ressourcen zur Verhinderung von Kaskadenausfällen.
	\item \textbf{Graceful Degradation:} Reduzierte Funktionalität bei Teilausfällen statt kompletter Systemausfall.
	\item \textbf{Chaos Engineering:} Proaktive Testung der Systemresilienz durch kontrollierte Fehlerinjektionen.
\end{enumerate}

\section*{Fazit}
Die vorgeschlagene Microservices-Architektur für das Auto Portal bietet eine skalierbare, resiliente und wartbare Lösung für die komplexen Anforderungen des Online-Automobilverkaufs.
Durch die sorgfältige Dekomposition in fachlich abgegrenzte Services, eine durchdachte Kommunikationsstrategie und robuste Überwachungsmechanismen wird eine Architektur geschaffen, die sowohl aktuelle als auch zukünftige Geschäftsanforderungen erfüllen kann.

Die hybride Ansätze bei Kommunikation und Koordination ermöglichen es, die Vorteile verschiedener Patterns zu nutzen und gleichzeitig die Komplexität überschaubar zu halten.
Das umfassende Monitoring- und Fehlerbehandlungskonzept gewährleistet hohe Verfügbarkeit und schnelle Problemidentifikation.

Diese Architektur bildet eine solide Grundlage für die digitale Transformation des Automobilvertriebs und unterstützt innovative Services wie Over-The-Air-Updates und präventive Wartung.

% ############################################################################
% CONTENT ENDS HERE
% ############################################################################

\end{document}
