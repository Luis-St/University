\documentclass[
	fontsize=12pt,          % default font size 12pt
	paper=a4,               % DIN A4 page format
	numbers=noenddot,       % remove dots behind chapter numbers (e.g. 1.5 not 1.5.)
	listof=totoc,           % add list of figures, tables, etc. to ToC
	listof=entryprefix,     % add entry name to figures, tables, etc.
	listof=nochaptergap,    % no chapter gap for figures, tables, etc.
	bibliography=totoc,     % add bibliography to ToC but without a chapter number
	parskip=half            % half line spacing between paragraphs
	openany                 % chapters can start on any page
]{scrbook}
\usepackage{tikz}

% ##################################################
% ENCODING
% ##################################################
\usepackage{cmap}               % PDF character encoding

%\usepackage[T1]{fontenc}        % 8-bit font encoding
%\usepackage[utf8]{inputenc}     % UTF-8 input encoding
%\usepackage[german]{babel} %%%%% Sprache festlegen

\usepackage[utf8]{inputenc}
\usepackage[T1]{fontenc}
\usepackage{lmodern}
\usepackage{ngerman}


% ##################################################
% GENERAL
% ##################################################
\usepackage{scrhack}            % better KOMA adaptions
\usepackage[table]{xcolor}      % color support
\usepackage{chngcntr}           % for renumbering stuff


\usepackage{amsthm}


\usepackage{lipsum}             % lorem ipsum generator (used for example content)


\usepackage{mathtools}


% ##################################################
% PDF SETTINGS
% ##################################################
\usepackage[
	colorlinks=false,
	linkcolor=black,
	citecolor=black,
	filecolor=black,
	urlcolor=black,
	bookmarks=true,
	bookmarksopen=true,
	bookmarksopenlevel=3,
	bookmarksnumbered,
	plainpages=false,
	pdfpagelabels=true,
	hyperfootnotes
]{hyperref}


% ##################################################
% FONTS AND SPACING
% ##################################################
\renewcommand{\familydefault}{\sfdefault}   % default font
\usepackage[onehalfspacing]{setspace}       % default 1.5 line spacing
\raggedbottom   % don't stretch spacing to fit page length

\usepackage{anyfontsize} % use any font size

% font sizes and styles
\addtokomafont{chapter}{\sffamily\large\bfseries}  % chapter heading 
\addtokomafont{section}{\sffamily\normalsize\bfseries}
\addtokomafont{subsection}{\sffamily\normalsize\mdseries}
\addtokomafont{caption}{\sffamily\normalsize\mdseries}

% url font style
\usepackage{relsize}
\renewcommand*{\UrlFont}{\ttfamily\smaller\relax}


% ##################################################
% PAGE FORMATTING
% ##################################################
% Page layout / Seitenränder
\usepackage[
	bindingoffset=0cm,
	inner=2.5cm,
	outer=2.5cm,
	top=3cm,
	bottom=2cm
]{geometry}

% Page header
\usepackage[
	headsepline,        % seperator line beneath page header on normal pages
	plainheadsepline    % seperator line beneath page header on pages like ToC
]{scrlayer-scrpage}
\clearpairofpagestyles                  % clear default settings
\addtokomafont{pagehead}{\normalfont}   % use normal font for page header
\ohead*{\thepage}                       % page number
\ihead*{\leftmark}                      % chapter name


% ##################################################
% IMAGES AND FIGURES
% ##################################################
\usepackage{graphicx}       % support for including images
\graphicspath{{pictures/}}  % default path
\usepackage{float}          % better control over float positions

% simple numbering without chapter
\renewcommand{\thefigure}{\arabic{figure}}
\counterwithout{figure}{chapter}

% wrap text around figures
\usepackage{wrapfig}


% ##################################################
% TABLES
% ##################################################
% multi row and multi column table functionality
\usepackage{booktabs}   % beautiful table style
\usepackage{multirow}

% simple numbering without chapter
\renewcommand{\thetable}{\arabic{table}}
\counterwithout{table}{chapter}


% ##################################################
% SOURCE CODE LISTINGS
% ##################################################
\usepackage{listings}
\usepackage{beramono}   % use a typewriter font which supports bold characters

\renewcommand{\lstlistlistingname}{List of Code Listings}   % 
\renewcommand{\lstlistingname}{Code Listing}
\newcommand{\listoflolentryname}{\lstlistingname}   % prefix for List of Code Listings

% define colors for source code highlighting
\definecolor{codegreen}{rgb}{0,0.6,0}
\definecolor{codegray}{rgb}{0.5,0.5,0.5}
\definecolor{codepurple}{rgb}{0.5,0,0.33}
\definecolor{codepurblue}{rgb}{0.16,0.0,1.0}
\definecolor{backcolour}{rgb}{0.95,0.95,0.92}

% ##################################################
% TABLE OF CONTENTS
% ##################################################
\KOMAoptions{toc=chapterentrydotfill}       % dotted lines for chapters
\addtokomafont{chapterentry}{\normalfont}   % use normal font for chapter entries
\setuptoc{toc}{totoc}                       % add ToC to ToC

% spacing
\DeclareTOCStyleEntry[beforeskip=0cm]{chapter}{chapter}
\DeclareTOCStyleEntry[beforeskip=0cm]{section}{section}
\DeclareTOCStyleEntry[beforeskip=0cm]{default}{subsection}

% colons after entry names
\BeforeStartingTOC[lof]{\def\autodot{:}}
\BeforeStartingTOC[lot]{\def\autodot{:}}
\BeforeStartingTOC[lol]{\def\autodot{:}}


% ##################################################
% BIBLIOGRAPHY
% ##################################################
\iffalse
\usepackage{csquotes} % context sensitive quotation
\setlength\bibitemsep{.5\baselineskip} % increase spacing between entries
\setcounter{biburlnumpenalty}{9000} % break URLs on numbers
\setcounter{biburllcpenalty}{9000}  % break URLs on lower case letters
\setcounter{biburlucpenalty}{9000}  % break URLs on upper case letters

\fi

% ##################################################
% ABBREVIATIONS
% ##################################################
\usepackage[printonlyused]{acronym}


% ##################################################
% APPENDIX
% ##################################################
\usepackage[title,titletoc]{appendix}

% appendix chapter
\newcommand{\appendixchapter}[1]{
\cleardoublepage
\pagenumbering{arabic}
\renewcommand{\thepage}{\thechapter-\arabic{page}}
\chapter{#1}
}

% insert monthly report pdf as picture in order to keep page header
\newcommand{\monthlyreport}[2]{
\section{#1}
\centering
\includegraphics[trim=55 35 55 35,clip,width=1\textwidth]{#2}
\clearpage
}


% ##################################################
% Theoreme
% ##################################################

% Umgebung fuer Beispiele
\newtheorem{beispiel}{Beispiel}

% Umgebung fuer These
\newtheorem{these}{These}

% Umgebung fuer Definitionen
\newtheorem{definition}{Definition}


% ##################################################
% MISC
% ##################################################
% better referencing of images, tables, etc.
\usepackage[nameinlink, noabbrev]{cleveref}


% ############################################################################
% WIE MACH ICH DAS HIER?
% 
% 1. Schreibe im "titlepage" Abschnitt den Titel in das element mit "\fontsize{22}{22}"
% 2. Aus obsidian raus kopieren mit "Copy to LaTeX"
% 3. Zwischen "CONTENT STARTS HERE" und "CONTENT ENDS HERE" den Inhalt einfügen
% 4. Sections anpassen. Damit man ne gescheite Chapter > Section > Subsection Struktur hat
% 5. Leere Seite nach Titelseite am Anfang mit {\let\cleardoublepage\relax \chapter{Erstes Kapitel}} verhindern
%
% TABELLEN
% Müssen extra gemacht werden, da obsidian das nicht unterstützt
% Kann ich nur empfehlen: https://tableconvert.com/markdown-to-latex
%
% BILDER
% Muss man vermutlich auch extra machen, hab ich aber noch nicht probiert
% ############################################################################

\begin{document}

\begin{titlepage}
	\pagestyle{empty}
	
	% HFU Logo
	\begin{flushright}
		\begin{figure}[ht]
			\flushright
			\includegraphics[height=2cm]{../../for_latex/hfu}
		\end{figure}
	\end{flushright}
	
	\begin{center}
		\vspace{3cm}
		
		{\fontsize{22}{22} \selectfont \textbf{Blatt 3}}\\[5mm]
		{\fontsize{18}{18} \selectfont Software Engineering 2}
		
		\vspace{12cm}
		
		{\fontsize{14}{14} \selectfont Luis Staudt}
	\end{center}
\end{titlepage}

% ############################################################################
% CONTENT STARTS HERE
% ############################################################################


\section*{User Stories mit Story Cards}

\subsection*{User Story 1}

Als Kunde \\
möchte ich eine Pizza nach meinen individuellen Vorlieben konfigurieren können \\
um eine maßgeschneiderte Bestellung aufgeben zu können, die genau meinen Geschmack trifft.

\vspace{0.5cm}
\textbf{Akzeptanzkriterien:}
\begin{itemize}
	\item Ich kann die Pizzagröße aus vier Optionen (Klein, Mittel, Groß, Familienpizza) auswählen.
	\item Ich kann zwischen verschiedenen Teigarten wählen (Klassisch, dünn, Vollkorn, glutenfrei, Käserand).
	\item Ich kann eine Saucenbasis auswählen (Tomatensauce, BBQ-Sauce, Creme Fraiche, Pesto).
	\item Ich kann eine oder mehrere Käsesorten auswählen oder veganen Käse wählen.
	\item Ich kann Toppings aus verschiedenen Kategorien hinzufügen.
	\item Der Preis wird automatisch aktualisiert, wenn ich Optionen ändere.
	\item Bei unzulässigen Kombinationen erhalte ich sofort eine Warnung und Hinweise.
\end{itemize}

\textbf{Story Points:} 8 \\
\textbf{Priorität:} Hoch

\subsection*{User Story 2}

Als Kunde \\
möchte ich meine früheren Bestellungen einsehen können \\
um schnell Lieblingsbestellungen erneut aufgeben zu können oder meine Bestellhistorie zu überprüfen.

\vspace{0.5cm}
\textbf{Akzeptanzkriterien:}
\begin{itemize}
	\item Ich kann eine chronologische Liste meiner früheren Bestellungen einsehen.
	\item Zu jeder früheren Bestellung werden Details angezeigt (Datum, bestellte Artikel, Gesamtpreis).
	\item Ich kann eine frühere Bestellung auswählen und als neue Bestellung übernehmen.
	\item Ich kann eine frühere Bestellung auswählen und vor dem erneuten Bestellen anpassen.
	\item Die Bestellhistorie ist nach der Anmeldung sofort verfügbar.
\end{itemize}

\textbf{Story Points:} 5 \\
\textbf{Priorität:} Mittel

\newpage

\section*{Anwendungsfalldiagramm}

\begin{figure}[h]
	\centering
	\begin{tikzpicture}[
		actor/.style={draw, rounded corners, fill=white, minimum width=1cm, minimum height=2cm},
		usecase/.style={draw, ellipse, fill=white, minimum width=3cm, minimum height=1cm},
		system/.style={draw, rounded corners, fill=white!90, minimum width=8cm, minimum height=10cm},
		line/.style={draw, -latex},
		include/.style={draw, -latex, dashed},
	]
		% System box
		\node[system] (system) at (6,0) {};
		\node at (6,4.5) {Pizza-Bestellsystem};
		
		% Actors
		\node[actor] (kunde) at (0,0) {Kunde};
		\node[actor] (bezahlsystem) at (14,0) {Bezahlsystem};
		
		% Use cases
		\node[usecase] (konfigurieren) at (6,2) {Pizza konfigurieren};
		\node[usecase] (bestellen) at (6,0) {Bestellung aufgeben};
		\node[usecase] (bezahlen) at (6,-2) {Bezahlung durchführen};
		\node[usecase] (historie) at (6,-4) {Frühere Bestellungen ansehen};
		
		% Connections from actors to use cases
		\draw[line] (kunde) -- (konfigurieren);
		\draw[line] (kunde) -- (bestellen);
		\draw[line] (kunde) -- (bezahlen);
		\draw[line] (kunde) -- (historie);
		\draw[line] (bezahlsystem) -- (bezahlen);
		
		% Include relationships
		\draw[include] (bestellen) -- node[midway, above] {<<include>>} (konfigurieren);
		\draw[include] (bestellen) -- node[midway, below] {<<include>>} (bezahlen);
	\end{tikzpicture}
	\caption{Anwendungsfalldiagramm für das Pizza-Bestellsystem}
\end{figure}

\newpage

\section*{Anwendungsfalltabellen}

\subsection*{Anwendungsfall: Pizza konfigurieren}

\begin{tabular}{|p{4cm}|p{10cm}|}
	\hline
	\rowcolor[HTML]{C0C0C0}
	\textbf{Anwendungsfall} & \textbf{Pizza konfigurieren}
	\\ \hline
	\textbf{Akteur} & Kunde
	\\ \hline
	\textbf{Beschreibung} & Der Kunde stellt eine individuelle Pizza zusammen, indem er verschiedene Optionen für Größe, Teig, Sauce, Käse und Toppings auswählt.
	\\ \hline
	\textbf{Auslöser} & Der Kunde möchte eine Pizza bestellen oder eine existierende Konfiguration ändern.
	\\ \hline
	\textbf{Vorbedingung} & Der Kunde hat die Bestellseite aufgerufen.
	\\ \hline
	\textbf{Nachbedingung (Erfolg)} & Die Pizza ist vollständig konfiguriert und bereit zur Bestellung.
	\\ \hline
	\textbf{Nachbedingung (Fehlschlag)} & Die Pizzakonfiguration wurde abgebrochen oder enthält ungültige Kombinationen.
	\\ \hline
	\multirow{6}{*}{\textbf{Standardablauf}} & 1.
	Der Kunde wählt eine Pizzagröße aus.
	\\ \cline{2-2}
	& 2.
	Der Kunde wählt eine Teigart aus.
	\\ \cline{2-2}
	& 3.
	Der Kunde wählt eine Saucenbasis aus.
	\\ \cline{2-2}
	& 4.
	Der Kunde wählt die gewünschte Käseart(en) aus.
	\\ \cline{2-2}
	& 5.
	Der Kunde wählt Toppings aus.
	\\ \cline{2-2}
	& 6.
	Der Kunde wählt den Backgrad aus.
	\\ \cline{2-2}
	& 7.
	Das System berechnet und zeigt den Preis an.
	\\ \hline
	\multirow{3}{*}{\textbf{Alternative Abläufe}} & 3a.
	Bei glutenfreiem Teig werden nur Tomatensauce oder BBQ-Sauce als Optionen angezeigt.
	\\ \cline{2-2}
	& 4a.
	Bei Auswahl von veganem Käse werden keine weiteren Käseoptionen angezeigt.
	\\ \cline{2-2}
	& 5a.
	Bei Auswahl von Spezialitäten werden andere Toppingoptionen deaktiviert.
	\\ \hline
	\textbf{Ausnahmen} & Wenn unzulässige Kombinationen gewählt werden, wird eine Fehlermeldung angezeigt und der Kunde wird aufgefordert, die Auswahl zu korrigieren.
	\\ \hline
	\textbf{Häufigkeit} & Sehr häufig (bei jeder Bestellung)
	\\ \hline
\end{tabular}

\subsection*{Anwendungsfall: Frühere Bestellungen ansehen}

\begin{tabular}{|p{4cm}|p{10cm}|}
	\hline
	\rowcolor[HTML]{C0C0C0}
	\textbf{Anwendungsfall}            & \textbf{Frühere Bestellungen ansehen}                                                    \\ \hline
	\textbf{Akteur}                     & Kunde                                                                                    \\ \hline
	\textbf{Beschreibung}               & Der Kunde kann seine früher getätigten Bestellungen einsehen und bei Bedarf wiederholen. \\ \hline
	\textbf{Auslöser}                   & Der Kunde navigiert zum Bestellarchiv oder wählt \("\)Meine Bestellungen\("\).           \\ \hline
	\textbf{Vorbedingung}               & Der Kunde ist angemeldet und hat bereits frühere Bestellungen getätigt.                  \\ \hline
	\textbf{Nachbedingung (Erfolg)}     & Die Liste der früheren Bestellungen wird angezeigt.                                      \\ \hline
	\textbf{Nachbedingung (Fehlschlag)} & Es werden keine Bestellungen angezeigt oder ein Systemfehler tritt auf.                  \\ \hline
	\multirow{3}{*}{\textbf{Standardablauf}} & 1.
	Der Kunde wählt die Option \("\)Frühere Bestellungen\("\) aus. \\ \cline{2-2}
	& 2.
	Das System zeigt eine chronologisch sortierte Liste aller früheren Bestellungen an. \\ \cline{2-2}
	& 3.
	Der Kunde kann Details zu jeder Bestellung einsehen. \\ \hline
	\multirow{2}{*}{\textbf{Alternative Abläufe}} & 3a.
	Der Kunde wählt eine frühere Bestellung aus, um sie zu wiederholen. \\ \cline{2-2}
	& 3b.
	Der Kunde wählt eine frühere Bestellung aus, um sie vor dem erneuten Bestellen anzupassen. \\ \hline
	\textbf{Ausnahmen}                  & Wenn keine früheren Bestellungen existieren, wird eine entsprechende Meldung angezeigt.  \\ \hline
	\textbf{Häufigkeit}                 & Regelmäßig (bei wiederkehrenden Kunden)                                                  \\ \hline
\end{tabular}

\newpage

\section*{Aktivitätsdiagramm für den Pizza-Konfigurator}

Anmerkung: LaTex bekommt es leider nicht besser hin das Diagramm darzustellen.

\begin{figure}[h]
	\centering
	\scalebox{0.75}{ % Etwas verkleinern, aber nicht zu stark
		\begin{tikzpicture}[node distance=1.2cm]
			% Knoten-Stile definieren (kleinere Größen)
			\tikzstyle{start} = [circle, minimum width=0.7cm, fill=black, inner sep=0pt, text=white, font=\small]
			\tikzstyle{end} = [circle, minimum width=0.7cm, fill=black, inner sep=0pt, draw=black, double, text=white, font=\small]
			\tikzstyle{activity} = [rectangle, rounded corners, minimum width=2.5cm, minimum height=0.8cm, text centered, draw=black, fill=blue!20, font=\small]
			\tikzstyle{decision} = [diamond, aspect=2, minimum width=2.5cm, minimum height=0.8cm, text centered, draw=black, fill=green!20, font=\small]
			\tikzstyle{arrow} = [thick,->,>=stealth]
			
			% Startpunkt
			\node (start) [start] {Start};
			
			% Aktivitäten (mit reduzierten vertikalen Abständen)
			\node (pizza_size) [activity, below of=start, yshift=-0.3cm] {Pizzagröße auswählen};
			\node (dough_type) [activity, below of=pizza_size, yshift=-0.3cm] {Teigart auswählen};
			
			% Entscheidung für Käserand (horizontale Abstände reduziert)
			\node (cheese_edge_decision) [decision, below of=dough_type, yshift=-0.8cm] {Käserand?};
			\node (size_check) [decision, right of=cheese_edge_decision, xshift=2cm] {Klein?};
			\node (error) [activity, right of=size_check, xshift=2cm] {Fehlermeldung};
			
			% Fortsetzung des Hauptflusses
			\node (sauce) [activity, below of=cheese_edge_decision, yshift=-0.8cm] {Sauce wählen};
			
			% Entscheidung für glutenfreien Teig
			\node (gluten_free_decision) [decision, below of=sauce, yshift=-0.8cm] {Glutenfrei?};
			\node (sauce_restriction) [activity, right of=gluten_free_decision, xshift=2cm] {Nur Tomaten-/BBQ};
			
			% Fortsetzung des Hauptflusses
			\node (cheese) [activity, below of=gluten_free_decision, yshift=-0.8cm] {Käseart wählen};
			
			% Entscheidung für veganen Käse
			\node (vegan_decision) [decision, below of=cheese, yshift=-0.8cm] {Vegan?};
			\node (cheese_restriction) [activity, right of=vegan_decision, xshift=2cm] {Keine weiteren Käsesorten};
			
			% Fortsetzung des Hauptflusses
			\node (toppings) [activity, below of=vegan_decision, yshift=-0.8cm] {Toppings wählen};
			\node (baking) [activity, below of=toppings, yshift=-0.3cm] {Backgrad wählen};
			\node (extras) [activity, below of=baking, yshift=-0.3cm] {Extras hinzufügen};
			\node (price) [activity, below of=extras, yshift=-0.3cm] {Preis berechnen};
			\node (add_to_cart) [activity, below of=price, yshift=-0.3cm] {Zur Bestellung hinzufügen};
			
			% Endpunkt
			\node (end) [end, below of=add_to_cart, yshift=-0.3cm] {Ende};
			
			% Pfeile
			\draw [arrow] (start) -- (pizza_size);
			\draw [arrow] (pizza_size) -- (dough_type);
			\draw [arrow] (dough_type) -- (cheese_edge_decision);
			
			% Käserand-Entscheidung
			\draw [arrow] (cheese_edge_decision) -- node[anchor=south, font=\scriptsize] {Ja} (size_check);
			\draw [arrow] (cheese_edge_decision) -- node[anchor=east, font=\scriptsize] {Nein} (sauce);
			\draw [arrow] (size_check) -- node[anchor=south, font=\scriptsize] {Ja} (error);
			\draw [arrow] (size_check) -- node[anchor=east, font=\scriptsize] {Nein} ++(0,-0.4) -| (sauce);
			\draw [arrow] (error) -- ++(0,-0.4) -| (sauce);
			
			% Glutenfreier Teig-Entscheidung
			\draw [arrow] (sauce) -- (gluten_free_decision);
			\draw [arrow] (gluten_free_decision) -- node[anchor=south, font=\scriptsize] {Ja} (sauce_restriction);
			\draw [arrow] (gluten_free_decision) -- node[anchor=east, font=\scriptsize] {Nein} (cheese);
			\draw [arrow] (sauce_restriction) -- ++(0,-0.4) -| (cheese);
			
			% Veganer Käse-Entscheidung
			\draw [arrow] (cheese) -- (vegan_decision);
			\draw [arrow] (vegan_decision) -- node[anchor=south, font=\scriptsize] {Ja} (cheese_restriction);
			\draw [arrow] (vegan_decision) -- node[anchor=east, font=\scriptsize] {Nein} (toppings);
			\draw [arrow] (cheese_restriction) -- ++(0,-0.4) -| (toppings);
			
			% Hauptfluss fortsetzen
			\draw [arrow] (toppings) -- (baking);
			\draw [arrow] (baking) -- (extras);
			\draw [arrow] (extras) -- (price);
			\draw [arrow] (price) -- (add_to_cart);
			\draw [arrow] (add_to_cart) -- (end);
		\end{tikzpicture}
	}
	\caption{Aktivitätsdiagramm für den Pizza-Konfigurator}
\end{figure}

% ############################################################################
% CONTENT ENDS HERE
% ############################################################################

\end{document}
