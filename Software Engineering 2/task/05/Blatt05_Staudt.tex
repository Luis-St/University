\documentclass[
	fontsize=12pt,          % default font size 12pt
	paper=a4,               % DIN A4 page format
	numbers=noenddot,       % remove dots behind chapter numbers (e.g. 1.5 not 1.5.)
	listof=totoc,           % add list of figures, tables, etc. to ToC
	listof=entryprefix,     % add entry name to figures, tables, etc.
	listof=nochaptergap,    % no chapter gap for figures, tables, etc.
	bibliography=totoc,     % add bibliography to ToC but without a chapter number
	parskip=half            % half line spacing between paragraphs
	openany                 % chapters can start on any page
]{scrbook}
\usepackage{tikz}

% ##################################################
% ENCODING
% ##################################################
\usepackage{cmap}               % PDF character encoding

%\usepackage[T1]{fontenc}        % 8-bit font encoding
%\usepackage[utf8]{inputenc}     % UTF-8 input encoding
%\usepackage[german]{babel} %%%%% Sprache festlegen

\usepackage[utf8]{inputenc}
\usepackage[T1]{fontenc}
\usepackage{lmodern}
\usepackage{ngerman}


% ##################################################
% GENERAL
% ##################################################
\usepackage{scrhack}            % better KOMA adaptions
\usepackage[table]{xcolor}      % color support
\usepackage{chngcntr}           % for renumbering stuff


\usepackage{amsthm}


\usepackage{lipsum}             % lorem ipsum generator (used for example content)


\usepackage{mathtools}


% ##################################################
% PDF SETTINGS
% ##################################################
\usepackage[
	colorlinks=false,
	linkcolor=black,
	citecolor=black,
	filecolor=black,
	urlcolor=black,
	bookmarks=true,
	bookmarksopen=true,
	bookmarksopenlevel=3,
	bookmarksnumbered,
	plainpages=false,
	pdfpagelabels=true,
	hyperfootnotes
]{hyperref}


% ##################################################
% FONTS AND SPACING
% ##################################################
\renewcommand{\familydefault}{\sfdefault}   % default font
\usepackage[onehalfspacing]{setspace}       % default 1.5 line spacing
\raggedbottom   % don't stretch spacing to fit page length

\usepackage{anyfontsize} % use any font size

% font sizes and styles
\addtokomafont{chapter}{\sffamily\large\bfseries}  % chapter heading 
\addtokomafont{section}{\sffamily\normalsize\bfseries}
\addtokomafont{subsection}{\sffamily\normalsize\mdseries}
\addtokomafont{caption}{\sffamily\normalsize\mdseries}

% url font style
\usepackage{relsize}
\renewcommand*{\UrlFont}{\ttfamily\smaller\relax}


% ##################################################
% PAGE FORMATTING
% ##################################################
% Page layout / Seitenränder
\usepackage[
	bindingoffset=0cm,
	inner=2.5cm,
	outer=2.5cm,
	top=3cm,
	bottom=2cm
]{geometry}

% Page header
\usepackage[
	headsepline,        % seperator line beneath page header on normal pages
	plainheadsepline    % seperator line beneath page header on pages like ToC
]{scrlayer-scrpage}
\clearpairofpagestyles                  % clear default settings
\addtokomafont{pagehead}{\normalfont}   % use normal font for page header
\ohead*{\thepage}                       % page number
\ihead*{\leftmark}                      % chapter name


% ##################################################
% IMAGES AND FIGURES
% ##################################################
\usepackage{graphicx}       % support for including images
\graphicspath{{pictures/}}  % default path
\usepackage{float}          % better control over float positions

% simple numbering without chapter
\renewcommand{\thefigure}{\arabic{figure}}
\counterwithout{figure}{chapter}

% wrap text around figures
\usepackage{wrapfig}


% ##################################################
% TABLES
% ##################################################
% multi row and multi column table functionality
\usepackage{booktabs}   % beautiful table style
\usepackage{multirow}

% simple numbering without chapter
\renewcommand{\thetable}{\arabic{table}}
\counterwithout{table}{chapter}


% ##################################################
% SOURCE CODE LISTINGS
% ##################################################
\usepackage{listings}
\usepackage{beramono}   % use a typewriter font which supports bold characters

\renewcommand{\lstlistlistingname}{List of Code Listings}   % 
\renewcommand{\lstlistingname}{Code Listing}
\newcommand{\listoflolentryname}{\lstlistingname}   % prefix for List of Code Listings

% define colors for source code highlighting
\definecolor{codegreen}{rgb}{0,0.6,0}
\definecolor{codegray}{rgb}{0.5,0.5,0.5}
\definecolor{codepurple}{rgb}{0.5,0,0.33}
\definecolor{codepurblue}{rgb}{0.16,0.0,1.0}
\definecolor{backcolour}{rgb}{0.95,0.95,0.92}

% ##################################################
% TABLE OF CONTENTS
% ##################################################
\KOMAoptions{toc=chapterentrydotfill}       % dotted lines for chapters
\addtokomafont{chapterentry}{\normalfont}   % use normal font for chapter entries
\setuptoc{toc}{totoc}                       % add ToC to ToC

% spacing
\DeclareTOCStyleEntry[beforeskip=0cm]{chapter}{chapter}
\DeclareTOCStyleEntry[beforeskip=0cm]{section}{section}
\DeclareTOCStyleEntry[beforeskip=0cm]{default}{subsection}

% colons after entry names
\BeforeStartingTOC[lof]{\def\autodot{:}}
\BeforeStartingTOC[lot]{\def\autodot{:}}
\BeforeStartingTOC[lol]{\def\autodot{:}}


% ##################################################
% BIBLIOGRAPHY
% ##################################################
\iffalse
\usepackage{csquotes} % context sensitive quotation
\setlength\bibitemsep{.5\baselineskip} % increase spacing between entries
\setcounter{biburlnumpenalty}{9000} % break URLs on numbers
\setcounter{biburllcpenalty}{9000}  % break URLs on lower case letters
\setcounter{biburlucpenalty}{9000}  % break URLs on upper case letters

\fi

% ##################################################
% ABBREVIATIONS
% ##################################################
\usepackage[printonlyused]{acronym}


% ##################################################
% APPENDIX
% ##################################################
\usepackage[title,titletoc]{appendix}

% appendix chapter
\newcommand{\appendixchapter}[1]{
\cleardoublepage
\pagenumbering{arabic}
\renewcommand{\thepage}{\thechapter-\arabic{page}}
\chapter{#1}
}

% insert monthly report pdf as picture in order to keep page header
\newcommand{\monthlyreport}[2]{
\section{#1}
\centering
\includegraphics[trim=55 35 55 35,clip,width=1\textwidth]{#2}
\clearpage
}


% ##################################################
% Theoreme
% ##################################################

% Umgebung fuer Beispiele
\newtheorem{beispiel}{Beispiel}

% Umgebung fuer These
\newtheorem{these}{These}

% Umgebung fuer Definitionen
\newtheorem{definition}{Definition}


% ##################################################
% MISC
% ##################################################
% better referencing of images, tables, etc.
\usepackage[nameinlink, noabbrev]{cleveref}


% ############################################################################
% WIE MACH ICH DAS HIER?
% 
% 1. Schreibe im "titlepage" Abschnitt den Titel in das element mit "\fontsize{22}{22}"
% 2. Aus obsidian raus kopieren mit "Copy to LaTeX"
% 3. Zwischen "CONTENT STARTS HERE" und "CONTENT ENDS HERE" den Inhalt einfügen
% 4. Sections anpassen. Damit man ne gescheite Chapter > Section > Subsection Struktur hat
% 5. Leere Seite nach Titelseite am Anfang mit {\let\cleardoublepage\relax \chapter{Erstes Kapitel}} verhindern
%
% TABELLEN
% Müssen extra gemacht werden, da obsidian das nicht unterstützt
% Kann ich nur empfehlen: https://tableconvert.com/markdown-to-latex
%
% BILDER
% Muss man vermutlich auch extra machen, hab ich aber noch nicht probiert
% ############################################################################

\begin{document}

\begin{titlepage}
	\pagestyle{empty}
	
	% HFU Logo
	\begin{flushright}
		\begin{figure}[ht]
			\flushright
			\includegraphics[height=2cm]{../../for_latex/hfu}
		\end{figure}
	\end{flushright}
	
	\begin{center}
		\vspace{3cm}
		
		{\fontsize{22}{22} \selectfont \textbf{Blatt 5}}\\[5mm]
		{\fontsize{18}{18} \selectfont Software Engineering 2}
		
		\vspace{12cm}
		
		{\fontsize{14}{14} \selectfont Luis Staudt}
	\end{center}
\end{titlepage}

% ############################################################################
% CONTENT STARTS HERE
% ############################################################################

{\let\cleardoublepage\relax \chapter*{Aufgabe 1}}

\section*{Fallstudie 1: Software Elektronische Patientenakte}

\subsection*{Logische Sicht}
\begin{figure}[!htbpf]
	\centering
	\includegraphics[width=0.7\textwidth]{./views/logic_view_1}
\end{figure}
\newpage

\subsection*{Physische Sicht}
\begin{figure}[!htbpf]
	\centering
	\includegraphics[width=0.7\textwidth]{./views/physical_view_1}
\end{figure}
\newpage

\section*{Fallstudie 2: Software zur Generierung von Klausuren}

\subsection*{Logische Sicht}
\begin{figure}[!htbpf]
	\centering
	\includegraphics[width=0.7\textwidth]{./views/logic_view_2}
\end{figure}
\newpage

\subsection*{Physische Sicht}
\begin{figure}[!htbpf]
	\centering
	\includegraphics[width=0.7\textwidth]{./views/physical_view_2}
\end{figure}
\newpage


{\let\cleardoublepage\relax \chapter*{Aufgabe 2}}
Die wichtigsten Architekturaspekte für ein Softwareprodukt für Finanzunternehmen wären meiner Meinung nach die Softwarekompatibilität und die nichtfunktionalen Produkteigenschaften.
\\
Die Softwarekompatibilität ist entscheidend, da Finanzunternehmen oft komplexe Systemlandschaften mit Legacy-Systemen haben,
die nahtlos integriert werden müssen, um Datenflüsse ohne Verluste zu gewährleisten.
Die nichtfunktionalen Produkteigenschaften wie Sicherheit, Compliance und Zuverlässigkeit sind im Finanzsektor besonders wichtig,
da sie strengen regulatorischen Anforderungen unterliegen und mit sensiblen Kundendaten arbeiten.
Ein Produkt, das diese beiden Aspekte priorisiert, bietet Finanzunternehmen nicht nur die technische Integration,
sondern auch die Gewissheit, dass es die hohen Branchenstandards für Datensicherheit und Betriebsstabilität erfüllt,
was für die Akzeptanz und den Erfolg im Markt ausschlaggebend ist.


{\let\cleardoublepage\relax \chapter*{Aufgabe 3}}

\section*{Zentralisiertes Sicherheitsmodell}

\subsection*{Vorteile}
\begin{itemize}
	\item Einfachere Verwaltung und Überwachung, da alle Sicherheitskontrollen an einem Ort implementiert werden.
	\item Konsistente Durchsetzung von Sicherheitsrichtlinien und -standards.
	\item Geringere Kosten für Sicherheitsinfrastruktur und Personal.
	\item Vereinfachte Compliance-Nachweise und Audits.
	\item Klare Verantwortlichkeiten und Zuständigkeiten.
\end{itemize}


\subsection*{Nachteile}
\begin{itemize}
	\item Single Point of Failure -- bei Kompromittierung sind alle Daten gefährdet.
	\item Höheres Schadensausmaß bei erfolgreichen Angriffen.
	\item Kann zu Leistungsengpässen führen, besonders bei geografisch verteilten Nutzern.
	\item Begrenzte Skalierbarkeit bei wachsenden Datenmengen.
	\item Potentiell höhere Latenz bei Datenzugriffen.
\end{itemize}


\section*{Verteiltes Sicherheitsmodell}

\subsection*{Vorteile}
\begin{itemize}
	\item Höhere Ausfallsicherheit -- kein einzelner Fehlerpunkt.
	\item Größere Skalierbarkeit für wachsende Datenmengen.
	\item Bessere Performance durch lokale Datenzugriffe.
	\item Geringeres Risiko eines vollständigen Datenverlusts.
	\item Kann besser an lokale rechtliche Anforderungen angepasst werden.
\end{itemize}

\subsection*{Nachteile}
\begin{itemize}
	\item Komplexere Verwaltung und Überwachung mehrerer Sicherheitssysteme.
	\item Schwierigere Durchsetzung einheitlicher Sicherheitsrichtlinien.
	\item Höhere Kosten für verteilte Sicherheitsinfrastruktur.
	\item Kompliziertere Compliance-Nachweise und Audits.
	\item Größere Angriffsfläche durch mehrere potenzielle Einstiegspunkte.
\end{itemize}


{\let\cleardoublepage\relax \chapter*{Aufgabe 4}}

\section*{Fallstudie 1: Software Elektronische Patientenakte}

\begin{tabular}{|p{4cm}|p{11.5cm}|}
	\hline
	\textbf{Technologieaspekte} & \textbf{Architekturentscheidung Fallstudie Elektronische Patientenakte mit Begründung} \\
	\hline
	Datenbank & \textbf{Relationale SQL-Datenbank}.
	Eine relationale Datenbank gewährleistet durch das ACID-Prinzip die Integrität der medizinischen Daten und bildet komplexe Beziehungen zwischen Patienten, Behandlungen und Zugriffsberechtigungen optimal ab.
	Die ausgereiften Sicherheitsmechanismen und Auditmöglichkeiten erfüllen die regulatorischen Anforderungen im Gesundheitsbereich.
	\\ \hline
	Plattform & \textbf{Webplattform mit nativer mobiler App}.
	Die Webplattform ermöglicht medizinischem Personal einen vollumfänglichen Zugriff an stationären Arbeitsplätzen, während die native App durch optimierte Bedienung und biometrische Authentifizierung sicheren mobilen Zugriff bietet.
	Diese Kombination gewährleistet höchste Sicherheitsstandards bei gleichzeitiger universeller Verfügbarkeit für alle Nutzergruppen.
	\\ \hline
	Server & \textbf{Hybridlösung mit dedizierten Servern und spezialisierter Healthcare-Cloud}.
	Dedizierte Server in deutschen Rechenzentren sichern sensible Kernfunktionen und gewährleisten die Einhaltung strengster Datenschutzanforderungen.
	Die ergänzende Healthcare-Cloud (z.B.\ Azure oder AWS mit Gesundheitszertifizierungen) bietet Skalierbarkeit und Hochverfügbarkeit bei gleichzeitiger Einhaltung von Standards wie HIPAA.
	\\ \hline
	Open Source & \textbf{Selektiver Einsatz geprüfter Open-Source-Komponenten}.
	Für die Patientenakte sollten nur sicherheitsgeprüfte Open-Source-Komponenten wie OpenEHR oder HAPI FHIR für Standardschnittstellen eingesetzt werden.
	Sicherheitskritische Kernfunktionen sollten durch proprietäre, speziell entwickelte und zertifizierte Lösungen abgedeckt werden, um höchste Sicherheitsstandards zu garantieren.
	\\ \hline
\end{tabular}

\section*{Fallstudie 2: Software zur Generierung von Klausuren}

\begin{tabular}{|p{4cm}|p{11.5cm}|}
	\hline
	\textbf{Technologieaspekte} & \textbf{Architekturentscheidung Fallstudie Klausurgenerierung mit Begründung} \\
	\hline
	Datenbank & \textbf{Relationale SQL-Datenbank}.
	Die klar strukturierten Aufgaben mit definierten Beziehungen zu Lehrveranstaltungen, Modulen und Musterlösungen passen optimal zum relationalen Modell.
	SQLite eignet sich besonders gut, da es als eingebettete Datenbank keine separate Serverinstallation erfordert und die lokale Desktop-Installation erleichtert.
	\\ \hline
	Plattform & \textbf{Desktop-Anwendung mit optionaler Web-Oberfläche}.
	Eine plattformübergreifende Desktop-Anwendung (z.B.\ mit Electron) erfüllt die explizite Anforderung der lokalen Installation und unterstützt den typischen Arbeitsablauf von Dozierenden.
	Eine optionale Web-Oberfläche könnte für einfache Aufgaben wie das Durchsuchen der Aufgabensammlung ergänzend angeboten werden.
	\\ \hline
	Server & \textbf{Keine Cloud-Lösung, nur lokal}.
	Die Software sollte vollständig offline funktionieren, da eine lokale Desktop-Installation gefordert ist und sensible Prüfungsdaten lokal geschützt werden müssen.
	Für erweiterte Funktionen wie die Zusammenarbeit zwischen Dozierenden könnte optional ein lokaler Server im Hochschulnetzwerk ausreichen.
	\\ \hline
	Open Source & \textbf{Umfangreicher Einsatz von Open-Source-Komponenten}.
	Bibliotheken für PDF-Generierung (wie PDFKit), Rich-Text-Editoren (wie TinyMCE) und UI-Frameworks beschleunigen die Entwicklung und verbessern die Wartbarkeit.
	Da es sich nicht um ein hochsensibles System handelt, ist der umfassende Einsatz von Open-Source-Komponenten mit geringeren Sicherheitsbedenken verbunden.
	\\ \hline
\end{tabular}

% ############################################################################
% CONTENT ENDS HERE
% ############################################################################

\end{document}
