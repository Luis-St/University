\documentclass[
	fontsize=12pt,          % default font size 12pt
	paper=a4,               % DIN A4 page format
	numbers=noenddot,       % remove dots behind chapter numbers (e.g. 1.5 not 1.5.)
	listof=totoc,           % add list of figures, tables, etc. to ToC
	listof=entryprefix,     % add entry name to figures, tables, etc.
	listof=nochaptergap,    % no chapter gap for figures, tables, etc.
	bibliography=totoc,     % add bibliography to ToC but without a chapter number
	parskip=half            % half line spacing between paragraphs
	openany                 % chapters can start on any page
]{scrbook}

% ##################################################
% ENCODING
% ##################################################
\usepackage{cmap}               % PDF character encoding

%\usepackage[T1]{fontenc}        % 8-bit font encoding
%\usepackage[utf8]{inputenc}     % UTF-8 input encoding
%\usepackage[german]{babel} %%%%% Sprache festlegen

\usepackage[utf8]{inputenc}
\usepackage[T1]{fontenc}
\usepackage{lmodern}
\usepackage{ngerman}


% ##################################################
% GENERAL
% ##################################################
\usepackage{scrhack}            % better KOMA adaptions
\usepackage[table]{xcolor}      % color support
\usepackage{chngcntr}           % for renumbering stuff


\usepackage{amsthm}


\usepackage{lipsum}             % lorem ipsum generator (used for example content)


\usepackage{mathtools}


% ##################################################
% PDF SETTINGS
% ##################################################
\usepackage[
	colorlinks=false,
	linkcolor=black,
	citecolor=black,
	filecolor=black,
	urlcolor=black,
	bookmarks=true,
	bookmarksopen=true,
	bookmarksopenlevel=3,
	bookmarksnumbered,
	plainpages=false,
	pdfpagelabels=true,
	hyperfootnotes
]{hyperref}


% ##################################################
% FONTS AND SPACING
% ##################################################
\renewcommand{\familydefault}{\sfdefault}   % default font
\usepackage[onehalfspacing]{setspace}       % default 1.5 line spacing
\raggedbottom   % don't stretch spacing to fit page length

\usepackage{anyfontsize} % use any font size

% font sizes and styles
\addtokomafont{chapter}{\sffamily\large\bfseries}  % chapter heading 
\addtokomafont{section}{\sffamily\normalsize\bfseries}
\addtokomafont{subsection}{\sffamily\normalsize\mdseries}
\addtokomafont{caption}{\sffamily\normalsize\mdseries}

% url font style
\usepackage{relsize}
\renewcommand*{\UrlFont}{\ttfamily\smaller\relax}


% ##################################################
% PAGE FORMATTING
% ##################################################
% Page layout / Seitenränder
\usepackage[
	bindingoffset=0cm,
	inner=2.5cm,
	outer=2.5cm,
	top=3cm,
	bottom=2cm
]{geometry}

% Page header
\usepackage[
	headsepline,        % seperator line beneath page header on normal pages
	plainheadsepline    % seperator line beneath page header on pages like ToC
]{scrlayer-scrpage}
\clearpairofpagestyles                  % clear default settings
\addtokomafont{pagehead}{\normalfont}   % use normal font for page header
\ohead*{\thepage}                       % page number
\ihead*{\leftmark}                      % chapter name


% ##################################################
% IMAGES AND FIGURES
% ##################################################
\usepackage{graphicx}       % support for including images
\graphicspath{{pictures/}}  % default path
\usepackage{float}          % better control over float positions

% simple numbering without chapter
\renewcommand{\thefigure}{\arabic{figure}}
\counterwithout{figure}{chapter}

% wrap text around figures
\usepackage{wrapfig}


% ##################################################
% TABLES
% ##################################################
% multi row and multi column table functionality
\usepackage{booktabs}   % beautiful table style
\usepackage{multirow}

% simple numbering without chapter
\renewcommand{\thetable}{\arabic{table}}
\counterwithout{table}{chapter}


% ##################################################
% SOURCE CODE LISTINGS
% ##################################################
\usepackage{listings}
\usepackage{beramono}   % use a typewriter font which supports bold characters

\renewcommand{\lstlistlistingname}{List of Code Listings}   % 
\renewcommand{\lstlistingname}{Code Listing}
\newcommand{\listoflolentryname}{\lstlistingname}   % prefix for List of Code Listings

% define colors for source code highlighting
\definecolor{codegreen}{rgb}{0,0.6,0}
\definecolor{codegray}{rgb}{0.5,0.5,0.5}
\definecolor{codepurple}{rgb}{0.5,0,0.33}
\definecolor{codepurblue}{rgb}{0.16,0.0,1.0}
\definecolor{backcolour}{rgb}{0.95,0.95,0.92}

% ##################################################
% TABLE OF CONTENTS
% ##################################################
\KOMAoptions{toc=chapterentrydotfill}       % dotted lines for chapters
\addtokomafont{chapterentry}{\normalfont}   % use normal font for chapter entries
\setuptoc{toc}{totoc}                       % add ToC to ToC

% spacing
\DeclareTOCStyleEntry[beforeskip=0cm]{chapter}{chapter}
\DeclareTOCStyleEntry[beforeskip=0cm]{section}{section}
\DeclareTOCStyleEntry[beforeskip=0cm]{default}{subsection}

% colons after entry names
\BeforeStartingTOC[lof]{\def\autodot{:}}
\BeforeStartingTOC[lot]{\def\autodot{:}}
\BeforeStartingTOC[lol]{\def\autodot{:}}


% ##################################################
% BIBLIOGRAPHY
% ##################################################
\iffalse
\usepackage{csquotes} % context sensitive quotation
\setlength\bibitemsep{.5\baselineskip} % increase spacing between entries
\setcounter{biburlnumpenalty}{9000} % break URLs on numbers
\setcounter{biburllcpenalty}{9000}  % break URLs on lower case letters
\setcounter{biburlucpenalty}{9000}  % break URLs on upper case letters

\fi

% ##################################################
% ABBREVIATIONS
% ##################################################
\usepackage[printonlyused]{acronym}


% ##################################################
% APPENDIX
% ##################################################
\usepackage[title,titletoc]{appendix}

% appendix chapter
\newcommand{\appendixchapter}[1]{
\cleardoublepage
\pagenumbering{arabic}
\renewcommand{\thepage}{\thechapter-\arabic{page}}
\chapter{#1}
}

% insert monthly report pdf as picture in order to keep page header
\newcommand{\monthlyreport}[2]{
\section{#1}
\centering
\includegraphics[trim=55 35 55 35,clip,width=1\textwidth]{#2}
\clearpage
}


% ##################################################
% Theoreme
% ##################################################

% Umgebung fuer Beispiele
\newtheorem{beispiel}{Beispiel}

% Umgebung fuer These
\newtheorem{these}{These}

% Umgebung fuer Definitionen
\newtheorem{definition}{Definition}


% ##################################################
% MISC
% ##################################################
% better referencing of images, tables, etc.
\usepackage[nameinlink, noabbrev]{cleveref}


% ############################################################################
% WIE MACH ICH DAS HIER?
% 
% 1. Schreibe im "titlepage" Abschnitt den Titel in das element mit "\fontsize{22}{22}"
% 2. Aus obsidian raus kopieren mit "Copy to LaTeX"
% 3. Zwischen "CONTENT STARTS HERE" und "CONTENT ENDS HERE" den Inhalt einfügen
% 4. Sections anpassen. Damit man ne gescheite Chapter > Section > Subsection Struktur hat
% 5. Leere Seite nach Titelseite am Anfang mit {\let\cleardoublepage\relax \chapter{Erstes Kapitel}} verhindern
%
% TABELLEN
% Müssen extra gemacht werden, da obsidian das nicht unterstützt
% Kann ich nur empfehlen: https://tableconvert.com/markdown-to-latex
%
% BILDER
% Muss man vermutlich auch extra machen, hab ich aber noch nicht probiert
% ############################################################################

\begin{document}

\begin{titlepage}
	\pagestyle{empty}
	
	% HFU Logo
	\begin{flushright}
		\begin{figure}[ht]
			\flushright
			\includegraphics[height=2cm]{../../for_latex/hfu}
		\end{figure}
	\end{flushright}
	
	\begin{center}
		\vspace{3cm}
		
		{\fontsize{22}{22} \selectfont \textbf{Blatt 2}}\\[5mm]
		{\fontsize{18}{18} \selectfont Software Engineering 2}
		
		\vspace{12cm}
		
		{\fontsize{14}{14} \selectfont Luis Staudt}
	\end{center}
\end{titlepage}

% ############################################################################
% CONTENT STARTS HERE
% ############################################################################

{\let\cleardoublepage\relax \chapter*{Aufgabe 1}}
\begin{table}[h]
	\centering
	\begin{tabular}{|p{0.3\textwidth}|p{0.7\textwidth}|}
		\hline
		\textbf{Name} & \textbf{Eigenschaften} \\
		\hline
		Software für die Steuerung einer Kaffeemaschine &
		\begin{itemize}
			\item \textbf{Kunde:} Nicht explizit genannt, vermutlich ein Hersteller von Kaffeemaschinen
			\item \textbf{Art der Applikation:} Eingebettete Software zur Steuerung der Hardware einer Kaffeemaschine
			\item \textbf{Zusätzliche Komponenten:}
			\begin{itemize}
				\item Benötigt: Schnittstellen zu Sensoren und Aktuatoren der Kaffeemaschine
				\item Nicht benötigt: Keine externe Datenbank, keine Netzwerkverbindung
			\end{itemize}
			\item \textbf{Besonderheiten:} Fest definierte Anforderungen, stabile Spezifikation
		\end{itemize} \\
		\hline
		Software \("\)Pizza bestellen\("\) &
		\begin{itemize}
			\item \textbf{Kunde:} Pizza-Kette oder ähnlicher Gastronomiebetrieb mit direktem Kontakt zum Entwicklungsteam
			\item \textbf{Art der Applikation:} Web- oder Mobile-Anwendung für Kunden zur Bestellung von Pizza
			\item \textbf{Zusätzliche Komponenten:}
			\begin{itemize}
				\item Benötigt: Datenbank für Bestellungen und Kundenprofile, Zahlungsschnittstelle, Sicherheitsmaßnahmen für Kundendaten
				\item Nicht explizit erwähnt: Mail-Server für Bestellbestätigungen
			\end{itemize}
			\item \textbf{Besonderheiten:} Kundenspezifische Entwicklung, Anforderungen können sich ändern
		\end{itemize} \\
		\hline
	\end{tabular}
	\caption{Zusammenfassung der Software-Projekte (Teil 1)}
	\label{tab:software-projekte-1}
\end{table}

\begin{table}[hf]
	\centering
	\begin{tabular}{|p{0.3\textwidth}|p{0.7\textwidth}|}
		\hline
		\textbf{Name} & \textbf{Eigenschaften} \\
		\hline
		Software \("\)Elektronische Patientenakte\("\) &
		\begin{itemize}
			\item \textbf{Kunde:} Öffentlicher Auftraggeber (Bund), gewonnen durch Ausschreibung
			\item \textbf{Art der Applikation:} Datenbanksystem mit verteilten Zugriffsrechten für medizinische Einrichtungen
			\item \textbf{Zusätzliche Komponenten:}
			\begin{itemize}
				\item Benötigt: Hochsichere Datenbank, Authentifizierungssystem, Zugriffskontrolle, Audit-Trail-System
				\item Nicht explizit erwähnt: Schnittstellen zu bestehenden Krankenhausinformationssystemen
			\end{itemize}
			\item \textbf{Besonderheiten:} Höchste Sicherheitsstandards, umfangreiche Dokumentationspflicht
		\end{itemize} \\
		\hline
		Software zur Generierung von Klausuren &
		\begin{itemize}
			\item \textbf{Kunde:} Hochschulen bzw. Bildungseinrichtungen
			\item \textbf{Art der Applikation:} Desktop-Software zur lokalen Installation
			\item \textbf{Zusätzliche Komponenten:}
			\begin{itemize}
				\item Benötigt: Lokale Datenbank für Aufgabensammlung, Druckfunktionalität
				\item Nicht benötigt: Keine zentrale Serverinfrastruktur, keine Netzwerkkomponenten
			\end{itemize}
			\item \textbf{Besonderheiten:} Muss lokal installierbar sein, keine Onlineverbindung erforderlich
		\end{itemize} \\
		\hline
	\end{tabular}
	\caption{Zusammenfassung der Software-Projekte (Teil 2)}
	\label{tab:software-projekte-2}
\end{table}

\newpage

{\let\cleardoublepage\relax \chapter*{Aufgabe 2}}
\begin{table}[hf]
	\centering
	\begin{tabular}{|p{0.3\textwidth}|p{0.7\textwidth}|}
		\hline
		\textbf{Projekt} & \textbf{Anwendbarkeit agiler Prinzipien} \\
		\hline
		Kaffeemaschinen-Steuerung &
		Nur bedingt anwendbar.
		Stabile Anforderungen und sequenzielle Funktionsabhängigkeiten passen eher zum Wasserfall-Modell.
		Hardware-Software-Interaktion erfordert detaillierte Vorabplanung.
		Fehlender direkter Kundenkontakt während der Entwicklung limitiert agile Kundenzusammenarbeit. \\
		\hline
		Pizza-Bestellsoftware &
		Sehr gut anwendbar.
		Direkter Kundenkontakt ermöglicht Anforderungsanpassungen.
		Kundenspezifische Natur unterstützt iterative Entwicklung mit frühem Feedback.
		Die drei Hauptkomponenten können inkrementell entwickelt werden.
		Nicht-funktionale Anforderungen betonen Individuen über Prozesse. \\
		\hline
		Elektronische Patientenakte &
		Teilweise anwendbar mit Anpassungsbedarf.
		Umfassende Dokumentationspflicht steht im Kontrast zum agilen Prinzip \("\)Funktionierende Software über Dokumentation\("\).
		Hohe Sicherheitsstandards erfordern gründliche Planung.
		Inkrementelle Funktionsentwicklung möglich.
		Komplexe Stakeholder-Landschaft erfordert formellere Abstimmungsprozesse. \\
		\hline
		Klausurgenerator &
		Moderat anwendbar.
		Klare Funktionalitäten eignen sich für iterative Entwicklung mit frühen Prototypen.
		Desktop-Installation vereinfacht Bereitstellung früher Versionen.
		Direkte Einbeziehung von Dozierenden möglich.
		Stabile Anforderungen und lokale Nutzung erschweren kontinuierliche Integration. \\
		\hline
	\end{tabular}
	\caption{Anwendbarkeit agiler Prinzipien auf verschiedene Softwareprojekte}
	\label{tab:agile-anwendbarkeit}
\end{table}

\newpage

\section*{1. Software für die Steuerung einer Kaffeemaschine}
Die Entwicklung der Kaffeemaschinen-Steuerungssoftware \textbf{eignet sich weniger} für agile Methoden:
\begin{itemize}
	\item Technisches System mit klaren physikalischen Einschränkungen
	\item Dokumentation wichtig für Sicherheit, Regulierung und Wartung
	\item Stabile, vordefinierte Anforderungen
	\item Deterministische Prozessfolge mit klaren Abhängigkeiten
	\item Geringer Raum für Änderungen
\end{itemize}


\section*{2. Software „Pizza bestellen"}
Die Pizza-Bestellsoftware \textbf{eignet sich sehr gut} für agile Entwicklungsmethoden:
\begin{itemize}
	\item Kundenorientierte Software erfordert tiefes Verständnis der Benutzerinteraktionen
	\item Schnelle Markteinführung wichtiger als umfassende Dokumentation
	\item Direkte Zusammenarbeit mit dem Auftraggeber möglich
	\item Anforderungen anpassbar während der Entwicklung
	\item Iterative Verbesserung der Benutzererfahrung und Sicherheit
\end{itemize}


\section*{3. Software „Elektronische Patientenakte"}
Die Patientenakten-Software \textbf{eignet sich teilweise} für agile Methoden:
\begin{itemize}
	\item Enge Zusammenarbeit zwischen Fachexperten und Entwicklern nötig
	\item Umfassende Dokumentation für Regulierung und Patientensicherheit unerlässlich
	\item Öffentliche Ausschreibung mit weniger flexiblen Vertragsbedingungen
	\item Hohe Sicherheitsanforderungen erfordern stabile Architektur
	\item Hybrider Ansatz empfehlenswert
\end{itemize}


\section*{4. Software zur Generierung von Klausuren}
Die Klausurgenerierungs-Software \textbf{eignet sich gut} für agile Entwicklungsmethoden:
\begin{itemize}
	\item Enge Interaktion mit Dozierenden als Endnutzer
	\item Frühes Feedback wichtiger als umfassende Vorabdokumentation
	\item Kontinuierliche Abstimmung zu spezifischen Anforderungen nötig
	\item Anforderungen können sich während der Implementierung ändern
	\item Überschaubare Komplexität, Fokus auf Benutzerfreundlichkeit
\end{itemize}

\newpage

{\let\cleardoublepage\relax \chapter*{Aufgabe 3}}
\section*{1. Grundidee/Grundprinzip}
Scrum ist ein agiles Framework für komplexe Produktentwicklung, das auf empirischer Prozesskontrolle basiert. Es folgt drei Grundprinzipien:
\begin{itemize}
	\item \textbf{Transparenz:} Alle wichtigen Aspekte des Prozesses müssen für alle Beteiligten sichtbar sein
	\item \textbf{Überprüfung:} Regelmäßige Überprüfung der Artefakte und des Fortschritts
	\item \textbf{Anpassung:} Schnelle Anpassung bei Abweichungen vom Ziel
\end{itemize}

Scrum organisiert die Arbeit in kurzen, zeitlich begrenzten Entwicklungszyklen (Sprints) und betont selbstorganisierte, cross-funktionale Teams.

\section*{2. Aktivitäten}
\begin{itemize}
	\item \textbf{Sprint:} Zeitlich begrenzte Entwicklungsphase (1-4 Wochen) mit festem Ziel
	\item \textbf{Sprint Planning:} Meeting zur Festlegung der Sprintziele und Aufgabenauswahl
	\item \textbf{Daily Scrum:} Tägliches 15-minütiges Statusmeeting des Teams
	\item \textbf{Sprint Review:} Präsentation des fertigen Inkrements am Ende des Sprints
	\item \textbf{Sprint Retrospective:} Reflexion über den abgeschlossenen Sprint mit Fokus auf Verbesserungspotential
	\item \textbf{Product Backlog Refinement:} Kontinuierliche Pflege und Priorisierung des Backlogs
\end{itemize}

\section*{3. Rollen und Verantwortlichkeiten}
\begin{itemize}
	\item \textbf{Product Owner:}
	\begin{itemize}
		\item Verantwortlich für Produktvision und Maximierung des Wertes
		\item Verwaltet das Product Backlog und priorisiert Anforderungen
		\item Entscheidet über Produktfunktionalitäten
	\end{itemize}
	
	\item \textbf{Scrum Master:}
	\begin{itemize}
		\item Fördert und unterstützt Scrum-Praktiken
		\item Beseitigt Hindernisse
		\item Schützt das Team vor externen Störungen
		\item Coaching des Teams und der Organisation
	\end{itemize}
	
	\item \textbf{Entwicklungsteam:}
	\begin{itemize}
		\item Selbstorganisiert und cross-funktional
		\item Verantwortlich für die Umsetzung der Anforderungen
		\item Gemeinsame Verantwortung für das Ergebnis
	\end{itemize}
\end{itemize}

\section*{4. Produkte/Dokumente (Deliverables)}
\begin{itemize}
	\item \textbf{Product Backlog:} Priorisierte Liste aller gewünschten Produktfeatures
	\item \textbf{Sprint Backlog:} Auswahl von Product Backlog-Einträgen für den aktuellen Sprint
	\item \textbf{Inkrement:} Summe aller abgeschlossenen Product Backlog-Einträge, die \("\)Done\("\) sind
	\item \textbf{Definition of Done:} Gemeinsames Verständnis darüber, wann ein Inkrement als fertig gilt
	\item \textbf{Burndown Chart:} Visualisierung des verbleibenden Arbeitsaufwands
\end{itemize}

\section*{5. Methoden, Richtlinien, Standards und Werkzeuge}
\begin{itemize}
	\item \textbf{User Stories:} Format zur Beschreibung von Anforderungen aus Benutzersicht
	\item \textbf{Planning Poker:} Methode zur Aufwandsschätzung
	\item \textbf{Task Board:} Visualisierung des Arbeitsfortschritts (oft mit Kanban-Boards)
	\item \textbf{Velocity:} Metrik zur Messung der Teamleistung
	\item \textbf{Werkzeuge:} JIRA, Trello, Azure DevOps, physical boards, etc.
	\item \textbf{Timeboxing:} Strenge zeitliche Begrenzung aller Meetings und Aktivitäten
	\item \textbf{Continuous Integration/Continuous Delivery:} Technische Praktiken, die oft mit Scrum kombiniert werden
\end{itemize}

\section*{6. Vor- und Nachteile}

\subsection*{Vorteile:}
\begin{itemize}
	\item Hohe Flexibilität und Anpassungsfähigkeit bei sich ändernden Anforderungen
	\item Frühe und regelmäßige Lieferung von funktionsfähiger Software
	\item Hohe Transparenz über Fortschritt und Hindernisse
	\item Starker Fokus auf Kundenzufriedenheit
	\item Verbesserung der Teamkommunikation und -zusammenarbeit
	\item Reduzierte Risiken durch regelmäßiges Feedback
\end{itemize}

\subsection*{Nachteile:}
\begin{itemize}
	\item Weniger geeignet für sehr große oder verteilte Teams ohne Anpassungen
	\item Erfordert erfahrene und engagierte Teammitglieder
	\item Kann bei mangelndem Commitment des Managements scheitern
	\item Herausfordernd bei festen Lieferterminen und festem Budget
	\item Dokumentation kann vernachlässigt werden
	\item Schwieriger bei sicherheitskritischen Systemen ohne zusätzliche Maßnahmen
\end{itemize}

\section*{Bewertungstabelle der Modelleigenschaften für Scrum}
\begin{table}[h]
	\centering
	\begin{tabular}{|l|l|c|}
		\hline
		\textbf{Eigenschaften des Modells}          &                     & \textbf{Note} \\
		\hline
		\multirow{3}{*}{Projektgröße und Komplexität} & Klein               & 1             \\
		& Mittel              & 1             \\
		& Groß                & 3             \\
		\hline
		\multirow{2}{*}{Qualität von Anforderungen}   & Klar                & 2             \\
		& Vage                & 1             \\
		\hline
		\multirow{3}{*}{Änderungen an Anforderungen}  & Keine               & 4             \\
		& Moderat             & 1             \\
		& Häufig              & 1             \\
		\hline
		\multirow{3}{*}{Sicherheit}                   & Sicherheitskritisch & 3             \\
		& Hoch                & 2             \\
		& Mittel              & 1             \\
		\hline
	\end{tabular}
	\caption{Bewertung der Modelleigenschaften für Scrum (1 = sehr gut geeignet, 6 = ungeeignet)}
\end{table}

\newpage

{\let\cleardoublepage\relax \chapter*{Aufgabe 4}}
\section*{1. Grundidee/Grundprinzip}
Das V-Modell XT ist ein Vorgehensmodell für die Planung und Durchführung von Systementwicklungsprojekten.
Es erweitert das klassische V-Modell, wobei XT für \("\)Extreme Tailoring\("\) steht.
Folgende Grundprinzipien charakterisieren das Modell:
\begin{itemize}
	\item \textbf{V-förmiger Entwicklungsprozess:} Entwicklungs- und Testaktivitäten bilden die Form eines V
	\item \textbf{Validierung und Verifikation:} Jeder Entwicklungsphase steht eine entsprechende Testphase gegenüber
	\item \textbf{Produktzentrierung:} Fokus auf Ergebnisprodukten statt auf Aktivitäten
	\item \textbf{Anpassbarkeit:} Projekte können das Modell durch Tailoring an ihre Bedürfnisse anpassen
	\item \textbf{Qualitätssicherung:} Durch geregelte Überprüfung und Abnahme der Projektergebnisse
	\item \textbf{Einheitliche Struktur:} Standardisierte Vorlagen und Prozesse für bessere Vergleichbarkeit
\end{itemize}


\section*{2. Aktivitäten}
\begin{itemize}
	\item \textbf{Projektmanagement:} Projektplanung, -steuerung und -überwachung
	\item \textbf{Systemanforderungsanalyse:} Erhebung und Dokumentation von Anforderungen
%	\item \textbf{Systemarchitekturentwurf:} Definition der Gesamtarchitektur
	\item \textbf{Feinentwurf:} Detaillierte Spezifikation einzelner Systemkomponenten
	\item \textbf{Implementierung:} Umsetzung des Entwurfs in Code
	\item \textbf{Integration:} Zusammenführung der Komponenten
	\item \textbf{Systemtest:} Überprüfung des Gesamtsystems
	\item \textbf{Abnahmetest:} Validierung gegen ursprüngliche Anforderungen
	\item \textbf{Konfigurationsmanagement:} Verwaltung der Systemversionen und Änderungen
	\item \textbf{Qualitätssicherung:} Kontinuierliche Überwachung der Qualität
	\item \textbf{Problemmanagement:} Systematische Bearbeitung von Problemen
	\item \textbf{Änderungsmanagement:} Kontrolle und Durchführung von Änderungen
\end{itemize}


\section*{3. Rollen und Verantwortlichkeiten}
\begin{itemize}
	\item \textbf{Projektleiter:}
	\begin{itemize}
		\item Verantwortlich für Projektplanung und -steuerung
		\item Führung des Projektteams
%		\item Berichterstattung an höheres Management
	\end{itemize}
	
%	\item \textbf{Anforderungsanalytiker:}
%	\begin{itemize}
%		\item Erhebung und Dokumentation von Anforderungen
%		\item Kommunikation mit Stakeholdern
%		\item Überprüfung der Anforderungsumsetzung
%	\end{itemize}
%
%	\item \textbf{Systemarchitekt:}
%	\begin{itemize}
%		\item Entwicklung der Systemarchitektur
%		\item Festlegung technischer Standards
%		\item Überwachung der architekturkonformen Implementierung
%	\end{itemize}
	
	\item \textbf{Entwickler:}
	\begin{itemize}
		\item Detailentwurf und Implementierung
		\item Komponententests
		\item Fehlerbehebung
	\end{itemize}
	
	\item \textbf{Tester:}
	\begin{itemize}
		\item Erstellung und Durchführung von Testplänen
		\item Dokumentation von Testergebnissen
		\item Überprüfung der Qualitätsstandards
	\end{itemize}
	
	\item \textbf{QS-Beauftragter:}
	\begin{itemize}
		\item Planung und Überwachung der Qualitätssicherungsmaßnahmen
		\item Durchführung von Reviews und Audits
		\item Überwachung der Prozesskonformität
	\end{itemize}
	
%	\item \textbf{Konfigurationsmanager:}
%	\begin{itemize}
%		\item Versionsverwaltung
%		\item Kontrolle der Änderungen
%		\item Sicherstellung der Integrität der Produkte
%	\end{itemize}
	
	\item \textbf{Auftraggeber/Anwendervertreter:}
	\begin{itemize}
		\item Abnahme der Lieferungen
		\item Bereitstellung fachlicher Anforderungen
		\item Entscheidung über Änderungsanträge
	\end{itemize}
\end{itemize}


\section*{4. Produkte/Dokumente (Deliverables)}
\begin{itemize}
	\item \textbf{Projekthandbuch:} Zentrale Festlegungen zum Projektablauf
	\item \textbf{Projektplan:} Zeitliche und ressourcenbezogene Planung
	\item \textbf{Anforderungsspezifikation:} Funktionale und nicht-funktionale Anforderungen
%	\item \textbf{Systemarchitekturdokument:} Beschreibung der Systemstruktur
	\item \textbf{Feinentwurfsdokumente:} Detaillierte Spezifikation der Komponenten
	\item \textbf{Quellcode und ausführbare Programme:} Implementierung
	\item \textbf{Testpläne und Testfälle:} Grundlage für die systematische Überprüfung
	\item \textbf{Testberichte:} Dokumentation der Testergebnisse
	\item \textbf{Benutzerdokumentation:} Anleitungen für Anwender
%	\item \textbf{Betriebsdokumentation:} Anleitungen für Administratoren
%	\item \textbf{Qualitätssicherungsberichte:} Ergebnisse der QS-Maßnahmen
	\item \textbf{Abnahmeprotokolle:} Bestätigung der Lieferungen
	\item \textbf{Statusberichte:} Regelmäßige Projektfortschrittsberichte
\end{itemize}


\section*{5. Methoden, Richtlinien, Standards und Werkzeuge}
\begin{itemize}
	\item \textbf{Projektmanagement-Richtlinien:} Standardisierte Verfahren für Projektplanung und -steuerung
%	\item \textbf{Dokumentvorlagen:} Für alle erforderlichen Dokumente und Produkte
	\item \textbf{Review- und Inspektionsverfahren:} Systematische Überprüfung der Dokumente
	\item \textbf{Konfigurationsmanagement-Richtlinien:} Verfahren zur Versionsverwaltung
	\item \textbf{Tailoring-Konzept:} Anpassung des Modells an Projektbedürfnisse
%	\item \textbf{Entscheidungspunkte:} Definierte Meilensteine mit Überprüfung des Projektfortschritts
	\item \textbf{Risikomanagementsystem:} Systematische Identifikation und Behandlung von Risiken
	\item \textbf{Änderungsmanagement-Prozess:} Kontrollierte Durchführung von Änderungen
%	\item \textbf{Verfolgbarkeitskonzept (Traceability):} Nachvollziehbarkeit von Anforderungen zur Implementierung
	\item \textbf{Werkzeuge:} CASE-Tools, Projektmanagement-Software, Versionskontrollsysteme, Requirements-Management-Tools
\end{itemize}


\section*{6. Vor- und Nachteile}

\subsection*{Vorteile:}
\begin{itemize}
	\item Umfassende Dokumentation und Nachvollziehbarkeit
	\item Frühzeitige Fehlererkennung durch Verifikation
	\item Klare Struktur und definierte Prozesse
	\item Hohe Qualitätssicherung durch systematische Überprüfungen
	\item Sehr gut geeignet für sicherheitskritische und regulierte Umgebungen
%	\item Gute Planbarkeit und Risikosteuerung
%	\item Anpassbar an unterschiedliche Projektarten durch Tailoring
\end{itemize}

\subsection*{Nachteile:}
\begin{itemize}
	\item Hoher Dokumentationsaufwand
	\item Weniger flexibel bei sich ändernden Anforderungen
	\item Kann bürokratisch und starr wirken
	\item Zeitaufwändiger Tailoring-Prozess
	\item Verzögerte Rückmeldung zu Funktionalität durch späte Testphasen
%	\item Potenziell höhere Kosten für kleinere Projekte
%	\item Weniger geeignet für explorative oder innovative Projekte
\end{itemize}

\newpage


\section*{Bewertungstabelle der Modelleigenschaften für V-Modell XT}
\begin{table}[h]
	\centering
	\begin{tabular}{|l|l|c|}
		\hline
		\textbf{Eigenschaften des Modells}          &                     & \textbf{Note} \\
		\hline
		\multirow{3}{*}{Projektgröße und Komplexität} & Klein               & 4             \\
		& Mittel              & 2             \\
		& Groß                & 1             \\
		\hline
		\multirow{2}{*}{Qualität von Anforderungen}   & Klar                & 1             \\
		& Vage                & 5             \\
		\hline
		\multirow{3}{*}{Änderungen an Anforderungen}  & Keine               & 1             \\
		& Moderat             & 3             \\
		& Häufig              & 5             \\
		\hline
		\multirow{3}{*}{Sicherheit}                   & Sicherheitskritisch & 1             \\
		& Hoch                & 1             \\
		& Mittel              & 2             \\
		\hline
	\end{tabular}
	\caption{Bewertung der Modelleigenschaften für V-Modell XT (1 = sehr gut geeignet, 6 = ungeeignet)}
\end{table}

\newpage

{\let\cleardoublepage\relax \chapter*{Aufgabe 5}}
\section*{1. Software für die Steuerung einer Kaffeemaschine}
Für die Entwicklung der Kaffeemaschinen-Steuerungssoftware empfehle ich das \textbf{V-Modell XT}.

\subsection*{\textbf{Begründung:}}
\begin{enumerate}
	\item Anforderungen sind klar definiert und stabil, optimal für V-Modell XT (Note 1)
	\item Funktionales Flussdiagramm entspricht V-förmiger Struktur des Modells
	\item Für Sicherheitsaspekte (Elektrizität, heißes Wasser) bietet V-Modell XT systematische Validierung
	\item Besonders geeignet für sicherheitskritische Systeme (Note 1)
%	\item Profitiert von produktzentrierter Herangehensweise des V-Modell XT
%	\item Klare Dokumentation unterstützt spätere Wartung
%	\item Geringere Flexibilität stellt kein Problem dar, da wenig Änderungen zu erwarten sind
\end{enumerate}


\section*{2. Software "Pizza bestellen"}
Für die Entwicklung der Pizza-Bestellsoftware empfehle ich \textbf{Scrum}.

\subsection*{\textbf{Begründung:}}
\begin{enumerate}
	\item Kundenspezifische Software mit anpassbaren Anforderungen, ideal für Scrum
	\item Exzelliert bei häufigen oder moderaten Änderungen (Note 1)
	\item Direkter Kontakt mit Auftraggeber ermöglicht regelmäßiges Feedback
%	\item Nicht-funktionale Anforderungen gut als Teil der Definition of Done integrierbar
%	\item Selbstorganisierte Teamarbeit fördert innovative Lösungen
	\item Besonders geeignet für Projekte mittlerer Größe (Note 1)
\end{enumerate}


\section*{3. Software "Elektronische Patientenakte"}
Für die Entwicklung der elektronischen Patientenakte empfehle ich eindeutig das \textbf{V-Modell XT}.

\subsection*{\textbf{Begründung:}}
\begin{enumerate}
	\item Muss höchsten Sicherheitsstandards entsprechen, V-Modell XT ideal für sicherheitskritische Systeme (Note 1)
	\item Regierungsauftrag fordert umfassende Dokumentation über gesamten Lebenszyklus
	\item V-Modell XT bietet diese für große Projekte (Note 1)
%	\item Gesetzliche Vorgaben und Compliance-Anforderungen werden durch klar definierte Prozesse erfüllt
%	\item Sequenzieller Ansatz mit definierten Entscheidungspunkten bietet notwendige Kontrolle
\end{enumerate}


\section*{4. Software zur Generierung von Klausuren}
Für die Entwicklung der Klausurgenerierungs-Software empfehle ich das \textbf{V-Modell XT}, wenn auch mit etwas Tailoring für Effizienz.

\subsection*{\textbf{Begründung:}}
\begin{enumerate}
	\item Stärke des V-Modell XT bei stabilen Anforderungen (Note 1)
	\item Zuverlässigkeit und Korrektheit haben oberste Priorität, sichergestellt durch systematische Testphasen
	\item Lokale Installation erfordert gründliche Systemarchitektur und Kompatibilitätstests
%	\item Datenbankfunktionalitäten profitieren von strukturierter Anforderungsanalyse
%	\item Zuverlässige Klausurerstellung durch umfassende Testphasen sichergestellt
\end{enumerate}

\newpage

% ############################################################################
% CONTENT ENDS HERE
% ############################################################################

\end{document}
