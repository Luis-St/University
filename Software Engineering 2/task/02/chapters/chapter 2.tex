\begin{table}[hf]
	\centering
	\begin{tabular}{|p{0.3\textwidth}|p{0.7\textwidth}|}
		\hline
		\textbf{Projekt} & \textbf{Anwendbarkeit agiler Prinzipien} \\
		\hline
		Kaffeemaschinen-Steuerung &
		Nur bedingt anwendbar.
		Stabile Anforderungen und sequenzielle Funktionsabhängigkeiten passen eher zum Wasserfall-Modell.
		Hardware-Software-Interaktion erfordert detaillierte Vorabplanung.
		Fehlender direkter Kundenkontakt während der Entwicklung limitiert agile Kundenzusammenarbeit. \\
		\hline
		Pizza-Bestellsoftware &
		Sehr gut anwendbar.
		Direkter Kundenkontakt ermöglicht Anforderungsanpassungen.
		Kundenspezifische Natur unterstützt iterative Entwicklung mit frühem Feedback.
		Die drei Hauptkomponenten können inkrementell entwickelt werden.
		Nicht-funktionale Anforderungen betonen Individuen über Prozesse. \\
		\hline
		Elektronische Patientenakte &
		Teilweise anwendbar mit Anpassungsbedarf.
		Umfassende Dokumentationspflicht steht im Kontrast zum agilen Prinzip \("\)Funktionierende Software über Dokumentation\("\).
		Hohe Sicherheitsstandards erfordern gründliche Planung.
		Inkrementelle Funktionsentwicklung möglich.
		Komplexe Stakeholder-Landschaft erfordert formellere Abstimmungsprozesse. \\
		\hline
		Klausurgenerator &
		Moderat anwendbar.
		Klare Funktionalitäten eignen sich für iterative Entwicklung mit frühen Prototypen.
		Desktop-Installation vereinfacht Bereitstellung früher Versionen.
		Direkte Einbeziehung von Dozierenden möglich.
		Stabile Anforderungen und lokale Nutzung erschweren kontinuierliche Integration. \\
		\hline
	\end{tabular}
	\caption{Anwendbarkeit agiler Prinzipien auf verschiedene Softwareprojekte}
	\label{tab:agile-anwendbarkeit}
\end{table}

\newpage

\section*{1. Software für die Steuerung einer Kaffeemaschine}
Die Entwicklung der Kaffeemaschinen-Steuerungssoftware \textbf{eignet sich weniger} für agile Methoden:
\begin{itemize}
	\item Technisches System mit klaren physikalischen Einschränkungen
	\item Dokumentation wichtig für Sicherheit, Regulierung und Wartung
	\item Stabile, vordefinierte Anforderungen
	\item Deterministische Prozessfolge mit klaren Abhängigkeiten
	\item Geringer Raum für Änderungen
\end{itemize}


\section*{2. Software „Pizza bestellen"}
Die Pizza-Bestellsoftware \textbf{eignet sich sehr gut} für agile Entwicklungsmethoden:
\begin{itemize}
	\item Kundenorientierte Software erfordert tiefes Verständnis der Benutzerinteraktionen
	\item Schnelle Markteinführung wichtiger als umfassende Dokumentation
	\item Direkte Zusammenarbeit mit dem Auftraggeber möglich
	\item Anforderungen anpassbar während der Entwicklung
	\item Iterative Verbesserung der Benutzererfahrung und Sicherheit
\end{itemize}


\section*{3. Software „Elektronische Patientenakte"}
Die Patientenakten-Software \textbf{eignet sich teilweise} für agile Methoden:
\begin{itemize}
	\item Enge Zusammenarbeit zwischen Fachexperten und Entwicklern nötig
	\item Umfassende Dokumentation für Regulierung und Patientensicherheit unerlässlich
	\item Öffentliche Ausschreibung mit weniger flexiblen Vertragsbedingungen
	\item Hohe Sicherheitsanforderungen erfordern stabile Architektur
	\item Hybrider Ansatz empfehlenswert
\end{itemize}


\section*{4. Software zur Generierung von Klausuren}
Die Klausurgenerierungs-Software \textbf{eignet sich gut} für agile Entwicklungsmethoden:
\begin{itemize}
	\item Enge Interaktion mit Dozierenden als Endnutzer
	\item Frühes Feedback wichtiger als umfassende Vorabdokumentation
	\item Kontinuierliche Abstimmung zu spezifischen Anforderungen nötig
	\item Anforderungen können sich während der Implementierung ändern
	\item Überschaubare Komplexität, Fokus auf Benutzerfreundlichkeit
\end{itemize}
