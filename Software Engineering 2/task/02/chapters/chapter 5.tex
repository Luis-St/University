\section*{1. Software für die Steuerung einer Kaffeemaschine}
Für die Entwicklung der Kaffeemaschinen-Steuerungssoftware empfehle ich das \textbf{V-Modell XT}.

\subsection*{\textbf{Begründung:}}
\begin{enumerate}
	\item Anforderungen sind klar definiert und stabil, optimal für V-Modell XT (Note 1)
	\item Funktionales Flussdiagramm entspricht V-förmiger Struktur des Modells
	\item Für Sicherheitsaspekte (Elektrizität, heißes Wasser) bietet V-Modell XT systematische Validierung
	\item Besonders geeignet für sicherheitskritische Systeme (Note 1)
%	\item Profitiert von produktzentrierter Herangehensweise des V-Modell XT
%	\item Klare Dokumentation unterstützt spätere Wartung
%	\item Geringere Flexibilität stellt kein Problem dar, da wenig Änderungen zu erwarten sind
\end{enumerate}


\section*{2. Software "Pizza bestellen"}
Für die Entwicklung der Pizza-Bestellsoftware empfehle ich \textbf{Scrum}.

\subsection*{\textbf{Begründung:}}
\begin{enumerate}
	\item Kundenspezifische Software mit anpassbaren Anforderungen, ideal für Scrum
	\item Exzelliert bei häufigen oder moderaten Änderungen (Note 1)
	\item Direkter Kontakt mit Auftraggeber ermöglicht regelmäßiges Feedback
%	\item Nicht-funktionale Anforderungen gut als Teil der Definition of Done integrierbar
%	\item Selbstorganisierte Teamarbeit fördert innovative Lösungen
	\item Besonders geeignet für Projekte mittlerer Größe (Note 1)
\end{enumerate}


\section*{3. Software "Elektronische Patientenakte"}
Für die Entwicklung der elektronischen Patientenakte empfehle ich eindeutig das \textbf{V-Modell XT}.

\subsection*{\textbf{Begründung:}}
\begin{enumerate}
	\item Muss höchsten Sicherheitsstandards entsprechen, V-Modell XT ideal für sicherheitskritische Systeme (Note 1)
	\item Regierungsauftrag fordert umfassende Dokumentation über gesamten Lebenszyklus
	\item V-Modell XT bietet diese für große Projekte (Note 1)
%	\item Gesetzliche Vorgaben und Compliance-Anforderungen werden durch klar definierte Prozesse erfüllt
%	\item Sequenzieller Ansatz mit definierten Entscheidungspunkten bietet notwendige Kontrolle
\end{enumerate}


\section*{4. Software zur Generierung von Klausuren}
Für die Entwicklung der Klausurgenerierungs-Software empfehle ich das \textbf{V-Modell XT}, wenn auch mit etwas Tailoring für Effizienz.

\subsection*{\textbf{Begründung:}}
\begin{enumerate}
	\item Stärke des V-Modell XT bei stabilen Anforderungen (Note 1)
	\item Zuverlässigkeit und Korrektheit haben oberste Priorität, sichergestellt durch systematische Testphasen
	\item Lokale Installation erfordert gründliche Systemarchitektur und Kompatibilitätstests
%	\item Datenbankfunktionalitäten profitieren von strukturierter Anforderungsanalyse
%	\item Zuverlässige Klausurerstellung durch umfassende Testphasen sichergestellt
\end{enumerate}
