\section*{1. Grundidee/Grundprinzip}
Scrum ist ein agiles Framework für komplexe Produktentwicklung, das auf empirischer Prozesskontrolle basiert. Es folgt drei Grundprinzipien:
\begin{itemize}
	\item \textbf{Transparenz:} Alle wichtigen Aspekte des Prozesses müssen für alle Beteiligten sichtbar sein
	\item \textbf{Überprüfung:} Regelmäßige Überprüfung der Artefakte und des Fortschritts
	\item \textbf{Anpassung:} Schnelle Anpassung bei Abweichungen vom Ziel
\end{itemize}

Scrum organisiert die Arbeit in kurzen, zeitlich begrenzten Entwicklungszyklen (Sprints) und betont selbstorganisierte, cross-funktionale Teams.

\section*{2. Aktivitäten}
\begin{itemize}
	\item \textbf{Sprint:} Zeitlich begrenzte Entwicklungsphase (1-4 Wochen) mit festem Ziel
	\item \textbf{Sprint Planning:} Meeting zur Festlegung der Sprintziele und Aufgabenauswahl
	\item \textbf{Daily Scrum:} Tägliches 15-minütiges Statusmeeting des Teams
	\item \textbf{Sprint Review:} Präsentation des fertigen Inkrements am Ende des Sprints
	\item \textbf{Sprint Retrospective:} Reflexion über den abgeschlossenen Sprint mit Fokus auf Verbesserungspotential
	\item \textbf{Product Backlog Refinement:} Kontinuierliche Pflege und Priorisierung des Backlogs
\end{itemize}

\section*{3. Rollen und Verantwortlichkeiten}
\begin{itemize}
	\item \textbf{Product Owner:}
	\begin{itemize}
		\item Verantwortlich für Produktvision und Maximierung des Wertes
		\item Verwaltet das Product Backlog und priorisiert Anforderungen
		\item Entscheidet über Produktfunktionalitäten
	\end{itemize}
	
	\item \textbf{Scrum Master:}
	\begin{itemize}
		\item Fördert und unterstützt Scrum-Praktiken
		\item Beseitigt Hindernisse
		\item Schützt das Team vor externen Störungen
		\item Coaching des Teams und der Organisation
	\end{itemize}
	
	\item \textbf{Entwicklungsteam:}
	\begin{itemize}
		\item Selbstorganisiert und cross-funktional
		\item Verantwortlich für die Umsetzung der Anforderungen
		\item Gemeinsame Verantwortung für das Ergebnis
	\end{itemize}
\end{itemize}

\section*{4. Produkte/Dokumente (Deliverables)}
\begin{itemize}
	\item \textbf{Product Backlog:} Priorisierte Liste aller gewünschten Produktfeatures
	\item \textbf{Sprint Backlog:} Auswahl von Product Backlog-Einträgen für den aktuellen Sprint
	\item \textbf{Inkrement:} Summe aller abgeschlossenen Product Backlog-Einträge, die \("\)Done\("\) sind
	\item \textbf{Definition of Done:} Gemeinsames Verständnis darüber, wann ein Inkrement als fertig gilt
	\item \textbf{Burndown Chart:} Visualisierung des verbleibenden Arbeitsaufwands
\end{itemize}

\section*{5. Methoden, Richtlinien, Standards und Werkzeuge}
\begin{itemize}
	\item \textbf{User Stories:} Format zur Beschreibung von Anforderungen aus Benutzersicht
	\item \textbf{Planning Poker:} Methode zur Aufwandsschätzung
	\item \textbf{Task Board:} Visualisierung des Arbeitsfortschritts (oft mit Kanban-Boards)
	\item \textbf{Velocity:} Metrik zur Messung der Teamleistung
	\item \textbf{Werkzeuge:} JIRA, Trello, Azure DevOps, physical boards, etc.
	\item \textbf{Timeboxing:} Strenge zeitliche Begrenzung aller Meetings und Aktivitäten
	\item \textbf{Continuous Integration/Continuous Delivery:} Technische Praktiken, die oft mit Scrum kombiniert werden
\end{itemize}

\section*{6. Vor- und Nachteile}

\subsection*{Vorteile:}
\begin{itemize}
	\item Hohe Flexibilität und Anpassungsfähigkeit bei sich ändernden Anforderungen
	\item Frühe und regelmäßige Lieferung von funktionsfähiger Software
	\item Hohe Transparenz über Fortschritt und Hindernisse
	\item Starker Fokus auf Kundenzufriedenheit
	\item Verbesserung der Teamkommunikation und -zusammenarbeit
	\item Reduzierte Risiken durch regelmäßiges Feedback
\end{itemize}

\subsection*{Nachteile:}
\begin{itemize}
	\item Weniger geeignet für sehr große oder verteilte Teams ohne Anpassungen
	\item Erfordert erfahrene und engagierte Teammitglieder
	\item Kann bei mangelndem Commitment des Managements scheitern
	\item Herausfordernd bei festen Lieferterminen und festem Budget
	\item Dokumentation kann vernachlässigt werden
	\item Schwieriger bei sicherheitskritischen Systemen ohne zusätzliche Maßnahmen
\end{itemize}

\section*{Bewertungstabelle der Modelleigenschaften für Scrum}
\begin{table}[h]
	\centering
	\begin{tabular}{|l|l|c|}
		\hline
		\textbf{Eigenschaften des Modells}          &                     & \textbf{Note} \\
		\hline
		\multirow{3}{*}{Projektgröße und Komplexität} & Klein               & 1             \\
		& Mittel              & 1             \\
		& Groß                & 3             \\
		\hline
		\multirow{2}{*}{Qualität von Anforderungen}   & Klar                & 2             \\
		& Vage                & 1             \\
		\hline
		\multirow{3}{*}{Änderungen an Anforderungen}  & Keine               & 4             \\
		& Moderat             & 1             \\
		& Häufig              & 1             \\
		\hline
		\multirow{3}{*}{Sicherheit}                   & Sicherheitskritisch & 3             \\
		& Hoch                & 2             \\
		& Mittel              & 1             \\
		\hline
	\end{tabular}
	\caption{Bewertung der Modelleigenschaften für Scrum (1 = sehr gut geeignet, 6 = ungeeignet)}
\end{table}
