\section*{1. Grundidee/Grundprinzip}
Das V-Modell XT ist ein Vorgehensmodell für die Planung und Durchführung von Systementwicklungsprojekten.
Es erweitert das klassische V-Modell, wobei XT für \("\)Extreme Tailoring\("\) steht.
Folgende Grundprinzipien charakterisieren das Modell:
\begin{itemize}
	\item \textbf{V-förmiger Entwicklungsprozess:} Entwicklungs- und Testaktivitäten bilden die Form eines V
	\item \textbf{Validierung und Verifikation:} Jeder Entwicklungsphase steht eine entsprechende Testphase gegenüber
	\item \textbf{Produktzentrierung:} Fokus auf Ergebnisprodukten statt auf Aktivitäten
	\item \textbf{Anpassbarkeit:} Projekte können das Modell durch Tailoring an ihre Bedürfnisse anpassen
	\item \textbf{Qualitätssicherung:} Durch geregelte Überprüfung und Abnahme der Projektergebnisse
	\item \textbf{Einheitliche Struktur:} Standardisierte Vorlagen und Prozesse für bessere Vergleichbarkeit
\end{itemize}


\section*{2. Aktivitäten}
\begin{itemize}
	\item \textbf{Projektmanagement:} Projektplanung, -steuerung und -überwachung
	\item \textbf{Systemanforderungsanalyse:} Erhebung und Dokumentation von Anforderungen
%	\item \textbf{Systemarchitekturentwurf:} Definition der Gesamtarchitektur
	\item \textbf{Feinentwurf:} Detaillierte Spezifikation einzelner Systemkomponenten
	\item \textbf{Implementierung:} Umsetzung des Entwurfs in Code
	\item \textbf{Integration:} Zusammenführung der Komponenten
	\item \textbf{Systemtest:} Überprüfung des Gesamtsystems
	\item \textbf{Abnahmetest:} Validierung gegen ursprüngliche Anforderungen
	\item \textbf{Konfigurationsmanagement:} Verwaltung der Systemversionen und Änderungen
	\item \textbf{Qualitätssicherung:} Kontinuierliche Überwachung der Qualität
	\item \textbf{Problemmanagement:} Systematische Bearbeitung von Problemen
	\item \textbf{Änderungsmanagement:} Kontrolle und Durchführung von Änderungen
\end{itemize}


\section*{3. Rollen und Verantwortlichkeiten}
\begin{itemize}
	\item \textbf{Projektleiter:}
	\begin{itemize}
		\item Verantwortlich für Projektplanung und -steuerung
		\item Führung des Projektteams
%		\item Berichterstattung an höheres Management
	\end{itemize}
	
%	\item \textbf{Anforderungsanalytiker:}
%	\begin{itemize}
%		\item Erhebung und Dokumentation von Anforderungen
%		\item Kommunikation mit Stakeholdern
%		\item Überprüfung der Anforderungsumsetzung
%	\end{itemize}
%
%	\item \textbf{Systemarchitekt:}
%	\begin{itemize}
%		\item Entwicklung der Systemarchitektur
%		\item Festlegung technischer Standards
%		\item Überwachung der architekturkonformen Implementierung
%	\end{itemize}
	
	\item \textbf{Entwickler:}
	\begin{itemize}
		\item Detailentwurf und Implementierung
		\item Komponententests
		\item Fehlerbehebung
	\end{itemize}
	
	\item \textbf{Tester:}
	\begin{itemize}
		\item Erstellung und Durchführung von Testplänen
		\item Dokumentation von Testergebnissen
		\item Überprüfung der Qualitätsstandards
	\end{itemize}
	
	\item \textbf{QS-Beauftragter:}
	\begin{itemize}
		\item Planung und Überwachung der Qualitätssicherungsmaßnahmen
		\item Durchführung von Reviews und Audits
		\item Überwachung der Prozesskonformität
	\end{itemize}
	
%	\item \textbf{Konfigurationsmanager:}
%	\begin{itemize}
%		\item Versionsverwaltung
%		\item Kontrolle der Änderungen
%		\item Sicherstellung der Integrität der Produkte
%	\end{itemize}
	
	\item \textbf{Auftraggeber/Anwendervertreter:}
	\begin{itemize}
		\item Abnahme der Lieferungen
		\item Bereitstellung fachlicher Anforderungen
		\item Entscheidung über Änderungsanträge
	\end{itemize}
\end{itemize}


\section*{4. Produkte/Dokumente (Deliverables)}
\begin{itemize}
	\item \textbf{Projekthandbuch:} Zentrale Festlegungen zum Projektablauf
	\item \textbf{Projektplan:} Zeitliche und ressourcenbezogene Planung
	\item \textbf{Anforderungsspezifikation:} Funktionale und nicht-funktionale Anforderungen
%	\item \textbf{Systemarchitekturdokument:} Beschreibung der Systemstruktur
	\item \textbf{Feinentwurfsdokumente:} Detaillierte Spezifikation der Komponenten
	\item \textbf{Quellcode und ausführbare Programme:} Implementierung
	\item \textbf{Testpläne und Testfälle:} Grundlage für die systematische Überprüfung
	\item \textbf{Testberichte:} Dokumentation der Testergebnisse
	\item \textbf{Benutzerdokumentation:} Anleitungen für Anwender
%	\item \textbf{Betriebsdokumentation:} Anleitungen für Administratoren
%	\item \textbf{Qualitätssicherungsberichte:} Ergebnisse der QS-Maßnahmen
	\item \textbf{Abnahmeprotokolle:} Bestätigung der Lieferungen
	\item \textbf{Statusberichte:} Regelmäßige Projektfortschrittsberichte
\end{itemize}


\section*{5. Methoden, Richtlinien, Standards und Werkzeuge}
\begin{itemize}
	\item \textbf{Projektmanagement-Richtlinien:} Standardisierte Verfahren für Projektplanung und -steuerung
%	\item \textbf{Dokumentvorlagen:} Für alle erforderlichen Dokumente und Produkte
	\item \textbf{Review- und Inspektionsverfahren:} Systematische Überprüfung der Dokumente
	\item \textbf{Konfigurationsmanagement-Richtlinien:} Verfahren zur Versionsverwaltung
	\item \textbf{Tailoring-Konzept:} Anpassung des Modells an Projektbedürfnisse
%	\item \textbf{Entscheidungspunkte:} Definierte Meilensteine mit Überprüfung des Projektfortschritts
	\item \textbf{Risikomanagementsystem:} Systematische Identifikation und Behandlung von Risiken
	\item \textbf{Änderungsmanagement-Prozess:} Kontrollierte Durchführung von Änderungen
%	\item \textbf{Verfolgbarkeitskonzept (Traceability):} Nachvollziehbarkeit von Anforderungen zur Implementierung
	\item \textbf{Werkzeuge:} CASE-Tools, Projektmanagement-Software, Versionskontrollsysteme, Requirements-Management-Tools
\end{itemize}


\section*{6. Vor- und Nachteile}

\subsection*{Vorteile:}
\begin{itemize}
	\item Umfassende Dokumentation und Nachvollziehbarkeit
	\item Frühzeitige Fehlererkennung durch Verifikation
	\item Klare Struktur und definierte Prozesse
	\item Hohe Qualitätssicherung durch systematische Überprüfungen
	\item Sehr gut geeignet für sicherheitskritische und regulierte Umgebungen
%	\item Gute Planbarkeit und Risikosteuerung
%	\item Anpassbar an unterschiedliche Projektarten durch Tailoring
\end{itemize}

\subsection*{Nachteile:}
\begin{itemize}
	\item Hoher Dokumentationsaufwand
	\item Weniger flexibel bei sich ändernden Anforderungen
	\item Kann bürokratisch und starr wirken
	\item Zeitaufwändiger Tailoring-Prozess
	\item Verzögerte Rückmeldung zu Funktionalität durch späte Testphasen
%	\item Potenziell höhere Kosten für kleinere Projekte
%	\item Weniger geeignet für explorative oder innovative Projekte
\end{itemize}

\newpage


\section*{Bewertungstabelle der Modelleigenschaften für V-Modell XT}
\begin{table}[h]
	\centering
	\begin{tabular}{|l|l|c|}
		\hline
		\textbf{Eigenschaften des Modells}          &                     & \textbf{Note} \\
		\hline
		\multirow{3}{*}{Projektgröße und Komplexität} & Klein               & 4             \\
		& Mittel              & 2             \\
		& Groß                & 1             \\
		\hline
		\multirow{2}{*}{Qualität von Anforderungen}   & Klar                & 1             \\
		& Vage                & 5             \\
		\hline
		\multirow{3}{*}{Änderungen an Anforderungen}  & Keine               & 1             \\
		& Moderat             & 3             \\
		& Häufig              & 5             \\
		\hline
		\multirow{3}{*}{Sicherheit}                   & Sicherheitskritisch & 1             \\
		& Hoch                & 1             \\
		& Mittel              & 2             \\
		\hline
	\end{tabular}
	\caption{Bewertung der Modelleigenschaften für V-Modell XT (1 = sehr gut geeignet, 6 = ungeeignet)}
\end{table}
