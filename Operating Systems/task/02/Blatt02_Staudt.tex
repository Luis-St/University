\documentclass[
	fontsize=12pt,          % default font size 12pt
	paper=a4,               % DIN A4 page format
	numbers=noenddot,       % remove dots behind chapter numbers (e.g. 1.5 not 1.5.)
	listof=totoc,           % add list of figures, tables, etc. to ToC
	listof=entryprefix,     % add entry name to figures, tables, etc.
	listof=nochaptergap,    % no chapter gap for figures, tables, etc.
	bibliography=totoc,     % add bibliography to ToC but without a chapter number
	parskip=half            % half line spacing between paragraphs
	openany                 % chapters can start on any page
]{scrbook}

% ##################################################
% ENCODING
% ##################################################
\usepackage{cmap}               % PDF character encoding

%\usepackage[T1]{fontenc}        % 8-bit font encoding
%\usepackage[utf8]{inputenc}     % UTF-8 input encoding
%\usepackage[german]{babel} %%%%% Sprache festlegen

\usepackage[utf8]{inputenc}
\usepackage[T1]{fontenc}
\usepackage{lmodern}
\usepackage{ngerman}


% ##################################################
% GENERAL
% ##################################################
\usepackage{scrhack}            % better KOMA adaptions
\usepackage[table]{xcolor}      % color support
\usepackage{chngcntr}           % for renumbering stuff


\usepackage{amsthm}


\usepackage{lipsum}             % lorem ipsum generator (used for example content)


\usepackage{mathtools}


% ##################################################
% PDF SETTINGS
% ##################################################
\usepackage[
	colorlinks=false,
	linkcolor=black,
	citecolor=black,
	filecolor=black,
	urlcolor=black,
	bookmarks=true,
	bookmarksopen=true,
	bookmarksopenlevel=3,
	bookmarksnumbered,
	plainpages=false,
	pdfpagelabels=true,
	hyperfootnotes
]{hyperref}


% ##################################################
% FONTS AND SPACING
% ##################################################
\renewcommand{\familydefault}{\sfdefault}   % default font
\usepackage[onehalfspacing]{setspace}       % default 1.5 line spacing
\raggedbottom   % don't stretch spacing to fit page length

\usepackage{anyfontsize} % use any font size

% font sizes and styles
\addtokomafont{chapter}{\sffamily\large\bfseries}  % chapter heading 
\addtokomafont{section}{\sffamily\normalsize\bfseries}
\addtokomafont{subsection}{\sffamily\normalsize\mdseries}
\addtokomafont{caption}{\sffamily\normalsize\mdseries}

% url font style
\usepackage{relsize}
\renewcommand*{\UrlFont}{\ttfamily\smaller\relax}


% ##################################################
% PAGE FORMATTING
% ##################################################
% Page layout / Seitenränder
\usepackage[
	bindingoffset=0cm,
	inner=2.5cm,
	outer=2.5cm,
	top=3cm,
	bottom=2cm
]{geometry}

% Page header
\usepackage[
	headsepline,        % seperator line beneath page header on normal pages
	plainheadsepline    % seperator line beneath page header on pages like ToC
]{scrlayer-scrpage}
\clearpairofpagestyles                  % clear default settings
\addtokomafont{pagehead}{\normalfont}   % use normal font for page header
\ohead*{\thepage}                       % page number
\ihead*{\leftmark}                      % chapter name


% ##################################################
% IMAGES AND FIGURES
% ##################################################
\usepackage{graphicx}       % support for including images
\graphicspath{{pictures/}}  % default path
\usepackage{float}          % better control over float positions

% simple numbering without chapter
\renewcommand{\thefigure}{\arabic{figure}}
\counterwithout{figure}{chapter}

% wrap text around figures
\usepackage{wrapfig}


% ##################################################
% TABLES
% ##################################################
% multi row and multi column table functionality
\usepackage{booktabs}   % beautiful table style
\usepackage{multirow}

% simple numbering without chapter
\renewcommand{\thetable}{\arabic{table}}
\counterwithout{table}{chapter}


% ##################################################
% SOURCE CODE LISTINGS
% ##################################################
\usepackage{listings}
\usepackage{beramono}   % use a typewriter font which supports bold characters

\renewcommand{\lstlistlistingname}{List of Code Listings}   % 
\renewcommand{\lstlistingname}{Code Listing}
\newcommand{\listoflolentryname}{\lstlistingname}   % prefix for List of Code Listings

% define colors for source code highlighting
\definecolor{codegreen}{rgb}{0,0.6,0}
\definecolor{codegray}{rgb}{0.5,0.5,0.5}
\definecolor{codepurple}{rgb}{0.5,0,0.33}
\definecolor{codepurblue}{rgb}{0.16,0.0,1.0}
\definecolor{backcolour}{rgb}{0.95,0.95,0.92}

% ##################################################
% TABLE OF CONTENTS
% ##################################################
\KOMAoptions{toc=chapterentrydotfill}       % dotted lines for chapters
\addtokomafont{chapterentry}{\normalfont}   % use normal font for chapter entries
\setuptoc{toc}{totoc}                       % add ToC to ToC

% spacing
\DeclareTOCStyleEntry[beforeskip=0cm]{chapter}{chapter}
\DeclareTOCStyleEntry[beforeskip=0cm]{section}{section}
\DeclareTOCStyleEntry[beforeskip=0cm]{default}{subsection}

% colons after entry names
\BeforeStartingTOC[lof]{\def\autodot{:}}
\BeforeStartingTOC[lot]{\def\autodot{:}}
\BeforeStartingTOC[lol]{\def\autodot{:}}


% ##################################################
% BIBLIOGRAPHY
% ##################################################
\iffalse
\usepackage{csquotes} % context sensitive quotation
\setlength\bibitemsep{.5\baselineskip} % increase spacing between entries
\setcounter{biburlnumpenalty}{9000} % break URLs on numbers
\setcounter{biburllcpenalty}{9000}  % break URLs on lower case letters
\setcounter{biburlucpenalty}{9000}  % break URLs on upper case letters

\fi

% ##################################################
% ABBREVIATIONS
% ##################################################
\usepackage[printonlyused]{acronym}


% ##################################################
% APPENDIX
% ##################################################
\usepackage[title,titletoc]{appendix}

% appendix chapter
\newcommand{\appendixchapter}[1]{
\cleardoublepage
\pagenumbering{arabic}
\renewcommand{\thepage}{\thechapter-\arabic{page}}
\chapter{#1}
}

% insert monthly report pdf as picture in order to keep page header
\newcommand{\monthlyreport}[2]{
\section{#1}
\centering
\includegraphics[trim=55 35 55 35,clip,width=1\textwidth]{#2}
\clearpage
}


% ##################################################
% Theoreme
% ##################################################

% Umgebung fuer Beispiele
\newtheorem{beispiel}{Beispiel}

% Umgebung fuer These
\newtheorem{these}{These}

% Umgebung fuer Definitionen
\newtheorem{definition}{Definition}


% ##################################################
% MISC
% ##################################################
% better referencing of images, tables, etc.
\usepackage[nameinlink, noabbrev]{cleveref}


% ############################################################################
% WIE MACH ICH DAS HIER?
% 
% 1. Schreibe im "titlepage" Abschnitt den Titel in das element mit "\fontsize{22}{22}"
% 2. Aus obsidian raus kopieren mit "Copy to LaTeX"
% 3. Zwischen "CONTENT STARTS HERE" und "CONTENT ENDS HERE" den Inhalt einfügen
% 4. Sections anpassen. Damit man ne gescheite Chapter > Section > Subsection Struktur hat
% 5. Leere Seite nach Titelseite am Anfang mit {\let\cleardoublepage\relax \chapter{Erstes Kapitel}} verhindern
%
% TABELLEN
% Müssen extra gemacht werden, da obsidian das nicht unterstützt
% Kann ich nur empfehlen: https://tableconvert.com/markdown-to-latex
%
% BILDER
% Muss man vermutlich auch extra machen, hab ich aber noch nicht probiert
% ############################################################################

\usepackage{amssymb}

\title{Betriebssysteme Aufgabenblatt 2 - Lösungen}
\author{Luis Staudt}
\date{}

\begin{document}

\begin{titlepage}
	\pagestyle{empty}
	
	% HFU Logo
	\begin{flushright}
		\begin{figure}[ht]
			\flushright
			\includegraphics[height=2cm]{../../for_latex/hfu}
		\end{figure}
	\end{flushright}
	
	\begin{center}
		\vspace{3cm}
		
		{\fontsize{22}{22} \selectfont \textbf{Blatt 2}}\\[5mm]
		{\fontsize{18}{18} \selectfont Betriebssysteme}
		
		\vspace{12cm}
		
		{\fontsize{14}{14} \selectfont Luis Staudt}
	\end{center}
\end{titlepage}

% ############################################################################
% CONTENT STARTS HERE
% ############################################################################

\section*{c) maximale Anzahl an Threads}

Auf zwei Systemen getestet:
Beide liefen über eine halbe Stunde ohne Probleme.
Ich vermute, dass es keine Begrenzung für die Anzahl der Threads gibt.
Es könnte eine Limitierung durch die Systemressourcen geben.

\section*{d) Programmabläufe mit unterschiedlichen Ergebnissen}

\subsection*{Endergebnis: 0 und 1}

In diesem Fall führt zuerst Thread X die ersten drei Zeilen seiner \texttt{tausche}-Methode auf Objekt a0 aus.
Dadurch wird der Wert von a0 zu 1 und der Wert von a1 zu 0.
Anschließend führt Thread Y die komplette \texttt{tausche}-Methode auf Objekt a1 aus.
Hierbei wird der Wert von a1 wieder auf 1 gesetzt und der Wert von a0 auf 0.
Am Ende haben wir also die ursprünglichen Werte.

\begin{table}[htbp]
	\centering
	\begin{tabular}{|l|l|c|c|c|c|}
		\hline
		\multicolumn{2}{|c|}{Zeilennr. Thread} & \multicolumn{2}{c|}{Objekt a0} & \multicolumn{2}{c|}{Objekt a1} \\
		\hline
		X in a0 & Y in a1 & wert & h & wert & h \\
		\hline
		1       & -       & 0    & 0 & 1    & - \\
		2       & -       & 1    & 0 & 1    & - \\
		3       & -       & 1    & 0 & 0    & - \\
		-       & 1       & 1    & 0 & 0    & 0 \\
		-       & 2       & 1    & 0 & 1    & 0 \\
		-       & 3       & 0    & 0 & 1    & 0 \\
		\hline
	\end{tabular}
	\caption{Thread-Ausführungsreihenfolge für Endergebnis: 0 und 1}
\end{table}

\subsection*{Endergebnis: 1 und 1}

Bei dieser Ausführungsreihenfolge führt zuerst Thread Y Zeile 1 seiner \texttt{tausche}-Methode aus und speichert den Wert 1 von a1 in seiner lokalen Variable h.
Dann führt Thread X alle drei Zeilen seiner \texttt{tausche}-Methode aus, wodurch a0 den Wert 1 erhält und a1 den Wert 0.
Anschließend führt Thread Y die restlichen Zeilen 2 und 3 aus,
wodurch a1 wieder Wert 1 erhält und a0 ebenfalls Wert 1 (aus der in Zeile 1 gespeicherten Variable h).

\begin{table}[htbp]
	\centering
	\begin{tabular}{|l|l|c|c|c|c|}
		\hline
		\multicolumn{2}{|c|}{Zeilennr. Thread} & \multicolumn{2}{c|}{Objekt a0} & \multicolumn{2}{c|}{Objekt a1} \\
		\hline
		X in a0 & Y in a1 & wert & h & wert & h \\
		\hline
		-       & 1       & 0    & - & 1    & 1 \\
		1       & -       & 0    & 0 & 1    & 1 \\
		2       & -       & 1    & 0 & 1    & 1 \\
		3       & -       & 1    & 0 & 0    & 1 \\
		-       & 2       & 1    & 0 & 1    & 1 \\
		-       & 3       & 1    & 0 & 1    & 1 \\
		\hline
	\end{tabular}
	\caption{Thread-Ausführungsreihenfolge für Endergebnis: 1 und 1}
\end{table}

\subsection*{Endergebnis: 1 und 0}

Hier führt Thread X zuerst Zeile 1 seiner \texttt{tausche}-Methode aus und speichert den Wert 0 von a0 in seiner lokalen Variable h.
Dann führt Thread Y alle drei Zeilen aus, wodurch a1 den Wert 0 erhält und a0 den Wert 1.
Anschließend führt Thread X die restlichen Zeilen 2 und 3 aus, bestätigt den Wert 1 für a0 und setzt a1 auf 0.

\begin{table}[htbp]
	\centering
	\begin{tabular}{|l|l|c|c|c|c|}
		\hline
		\multicolumn{2}{|c|}{Zeilennr. Thread} & \multicolumn{2}{c|}{Objekt a0} & \multicolumn{2}{c|}{Objekt a1} \\
		\hline
		X in a0 & Y in a1 & wert & h & wert & h \\
		\hline
		1       & -       & 0    & 0 & 1    & - \\
		-       & 1       & 0    & 0 & 1    & 1 \\
		-       & 2       & 1    & 0 & 1    & 1 \\
		-       & 3       & 1    & 0 & 0    & 1 \\
		2       & -       & 1    & 0 & 0    & - \\
		3       & -       & 1    & 0 & 0    & - \\
		\hline
	\end{tabular}
	\caption{Thread-Ausführungsreihenfolge für Endergebnis: 1 und 0}
\end{table}

\subsection*{Endergebnis: 0 und 0}

Bei dieser Ausführungsreihenfolge werden verschiedene Zeilen der \texttt{tausche}-Methode durch die Threads
ausgeführt, wobei am Ende beide Objekte den Wert 0 haben.
Thread Y führt seine Zeilen in einer bestimmten Reihenfolge aus, setzt a0 auf 1 und a1 auf 0.
Anschließend führt Thread X Zeilen 2 und 3 aus, wobei hier besonders der Unterschied zum vorherigen Fall ist,
dass a0 bereits vor Zeile 2 den Wert 0 besitzt und somit der Wert von a0 bei 0 bleibt.

\begin{table}[htbp]
	\centering
	\begin{tabular}{|l|l|c|c|c|c|}
		\hline
		\multicolumn{2}{|c|}{Zeilennr. Thread} & \multicolumn{2}{c|}{Objekt a0} & \multicolumn{2}{c|}{Objekt a1} \\
		\hline
		X in a0 & Y in a1 & wert & h & wert & h \\
		\hline
		1       & -       & 0    & 0 & 1    & - \\
		-       & 1       & 0    & 0 & 1    & 1 \\
		-       & 2       & 0    & 0 & 0    & 1 \\
		-       & 3       & 1    & 0 & 0    & 1 \\
		2       & -       & 0    & 0 & 0    & - \\
		3       & -       & 0    & 0 & 0    & - \\
		\hline
	\end{tabular}
	\caption{Thread-Ausführungsreihenfolge für Endergebnis: 0 und 0}
\end{table}

\section*{Änderung des Verhaltens mit privater Synchronisation}

\begin{itemize}
	\item Das Verhalten des Programms gegenüber I) ändert sich \textbf{nicht}.
	\item Begründung: Jedes A-Objekt hat sein eigenes privates Lock-Objekt \texttt{o}.
	\item Wenn Thread X auf \texttt{a0.tausche(a1)} zugreift, synchronisiert es auf das private Lock von \texttt{a0}.
	\item Wenn Thread Y auf \texttt{a1.tausche(a0)} zugreift, synchronisiert es auf das private Lock von \texttt{a1}.
	\item Da es sich um verschiedene Lock-Objekte handelt, können beide Threads immer noch parallel ausgeführt werden.
	\item Die gleichen vier Ergebnisse wie in Teil I) sind weiterhin möglich.
\end{itemize}

Die private Synchronisation bewirkt lediglich, dass die \texttt{tausche}-Methode atomar in Bezug auf dasselbe Objekt ausgeführt wird.
Da die beiden Threads jedoch auf unterschiedlichen Objekten operieren und somit unterschiedliche Locks verwenden, bleibt die Race-Condition zwischen den Threads bestehen.

\section*{Änderung des Verhaltens mit statischer Synchronisation}

\begin{itemize}
	\item Mit einem statischen Lock-Objekt ändert sich das Verhalten grundlegend.
	\item Da alle A-Objekte dasselbe Lock verwenden, kann zu jedem Zeitpunkt nur ein Thread in den synchronisierten Block eintreten.
	\item Dies bedeutet:
	\begin{itemize}
		\item Die Methode \texttt{tausche} wird atomar ausgeführt
		\item Es sind keine Verzahnungen von Threads mehr möglich
		\item Nur zwei Ausführungsreihenfolgen sind noch möglich, die aber zum gleichen Endergebnis führen
	\end{itemize}
	\item Ergebnis in beiden Fällen: \texttt{a0.wert = 0, a1.wert = 1}
	\begin{itemize}
		\item Fall 1: Thread X führt vollständig aus, dann Thread Y
		\item Fall 2: Thread Y führt vollständig aus, dann Thread X
	\end{itemize}
	\item In beiden Fällen tauschen sich die Werte erst komplett und dann wieder zurück.
\end{itemize}

Durch die Verwendung eines statischen Lock-Objekts wird ein globaler Synchronisationspunkt für alle Instanzen der Klasse A geschaffen.
Dies verhindert jegliche verzahnte Ausführung der \texttt{tausche}-Methode und garantiert, dass immer nur ein Thread zu einem Zeitpunkt innerhalb des synchronisierten Blocks ist.
Dadurch reduzieren sich die möglichen Endzustände auf nur einen Fall.

\section*{Bewertung der unterschiedlichen Synchronisationsmethoden}

\begin{itemize}
	\item \textbf{Methode III (statisches Lock)} ist besser geeignet, weil:
	\begin{itemize}
		\item Sie verhindert alle möglichen Race Conditions
		\item Sie ist einfacher zu verstehen
		\item Sie vermeidet Deadlocks
	\end{itemize}
	\item \textbf{Methode IV (verschachtelte Locks)} hat ein kritisches Problem:
	\begin{itemize}
		\item Sie kann zu Deadlocks führen, wenn zwei Threads die Methode gleichzeitig mit vertauschten Argumenten aufrufen
		\item Beispiel für Deadlock:
		\begin{itemize}
			\item Thread X ruft \texttt{a0.tausche(a1)} auf und erhält das Lock für \texttt{a0}
			\item Thread Y ruft \texttt{a1.tausche(a0)} auf und erhält das Lock für \texttt{a1}
			\item Thread X wartet auf das Lock für \texttt{a1} (das Y hält)
			\item Thread Y wartet auf das Lock für \texttt{a0} (das X hält)
			\item Beide Threads blockieren sich gegenseitig - klassischer Deadlock
		\end{itemize}
	\end{itemize}
\end{itemize}

Die Methode mit statischem Lock bietet den Vorteil der Einfachheit und Sicherheit, indem sie globale Synchronisation gewährleistet.
Die verschachtelte Synchronisation in Methode IV führt zu einem klassischen Deadlock-Problem durch zyklische Abhängigkeiten, was in produktiven Umgebungen kritisch wäre und vermieden werden sollte.

% ############################################################################
% CONTENT ENDS HERE
% ############################################################################

\end{document}
