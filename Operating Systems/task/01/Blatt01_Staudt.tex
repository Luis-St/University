\documentclass[
	fontsize=12pt,          % default font size 12pt
	paper=a4,               % DIN A4 page format
	numbers=noenddot,       % remove dots behind chapter numbers (e.g. 1.5 not 1.5.)
	listof=totoc,           % add list of figures, tables, etc. to ToC
	listof=entryprefix,     % add entry name to figures, tables, etc.
	listof=nochaptergap,    % no chapter gap for figures, tables, etc.
	bibliography=totoc,     % add bibliography to ToC but without a chapter number
	parskip=half            % half line spacing between paragraphs
	openany                 % chapters can start on any page
]{scrbook}

% ##################################################
% ENCODING
% ##################################################
\usepackage{cmap}               % PDF character encoding

%\usepackage[T1]{fontenc}        % 8-bit font encoding
%\usepackage[utf8]{inputenc}     % UTF-8 input encoding
%\usepackage[german]{babel} %%%%% Sprache festlegen

\usepackage[utf8]{inputenc}
\usepackage[T1]{fontenc}
\usepackage{lmodern}
\usepackage{ngerman}


% ##################################################
% GENERAL
% ##################################################
\usepackage{scrhack}            % better KOMA adaptions
\usepackage[table]{xcolor}      % color support
\usepackage{chngcntr}           % for renumbering stuff


\usepackage{amsthm}


\usepackage{lipsum}             % lorem ipsum generator (used for example content)


\usepackage{mathtools}


% ##################################################
% PDF SETTINGS
% ##################################################
\usepackage[
	colorlinks=false,
	linkcolor=black,
	citecolor=black,
	filecolor=black,
	urlcolor=black,
	bookmarks=true,
	bookmarksopen=true,
	bookmarksopenlevel=3,
	bookmarksnumbered,
	plainpages=false,
	pdfpagelabels=true,
	hyperfootnotes
]{hyperref}


% ##################################################
% FONTS AND SPACING
% ##################################################
\renewcommand{\familydefault}{\sfdefault}   % default font
\usepackage[onehalfspacing]{setspace}       % default 1.5 line spacing
\raggedbottom   % don't stretch spacing to fit page length

\usepackage{anyfontsize} % use any font size

% font sizes and styles
\addtokomafont{chapter}{\sffamily\large\bfseries}  % chapter heading 
\addtokomafont{section}{\sffamily\normalsize\bfseries}
\addtokomafont{subsection}{\sffamily\normalsize\mdseries}
\addtokomafont{caption}{\sffamily\normalsize\mdseries}

% url font style
\usepackage{relsize}
\renewcommand*{\UrlFont}{\ttfamily\smaller\relax}


% ##################################################
% PAGE FORMATTING
% ##################################################
% Page layout / Seitenränder
\usepackage[
	bindingoffset=0cm,
	inner=2.5cm,
	outer=2.5cm,
	top=3cm,
	bottom=2cm
]{geometry}

% Page header
\usepackage[
	headsepline,        % seperator line beneath page header on normal pages
	plainheadsepline    % seperator line beneath page header on pages like ToC
]{scrlayer-scrpage}
\clearpairofpagestyles                  % clear default settings
\addtokomafont{pagehead}{\normalfont}   % use normal font for page header
\ohead*{\thepage}                       % page number
\ihead*{\leftmark}                      % chapter name


% ##################################################
% IMAGES AND FIGURES
% ##################################################
\usepackage{graphicx}       % support for including images
\graphicspath{{pictures/}}  % default path
\usepackage{float}          % better control over float positions

% simple numbering without chapter
\renewcommand{\thefigure}{\arabic{figure}}
\counterwithout{figure}{chapter}

% wrap text around figures
\usepackage{wrapfig}


% ##################################################
% TABLES
% ##################################################
% multi row and multi column table functionality
\usepackage{booktabs}   % beautiful table style
\usepackage{multirow}

% simple numbering without chapter
\renewcommand{\thetable}{\arabic{table}}
\counterwithout{table}{chapter}


% ##################################################
% SOURCE CODE LISTINGS
% ##################################################
\usepackage{listings}
\usepackage{beramono}   % use a typewriter font which supports bold characters

\renewcommand{\lstlistlistingname}{List of Code Listings}   % 
\renewcommand{\lstlistingname}{Code Listing}
\newcommand{\listoflolentryname}{\lstlistingname}   % prefix for List of Code Listings

% define colors for source code highlighting
\definecolor{codegreen}{rgb}{0,0.6,0}
\definecolor{codegray}{rgb}{0.5,0.5,0.5}
\definecolor{codepurple}{rgb}{0.5,0,0.33}
\definecolor{codepurblue}{rgb}{0.16,0.0,1.0}
\definecolor{backcolour}{rgb}{0.95,0.95,0.92}

% ##################################################
% TABLE OF CONTENTS
% ##################################################
\KOMAoptions{toc=chapterentrydotfill}       % dotted lines for chapters
\addtokomafont{chapterentry}{\normalfont}   % use normal font for chapter entries
\setuptoc{toc}{totoc}                       % add ToC to ToC

% spacing
\DeclareTOCStyleEntry[beforeskip=0cm]{chapter}{chapter}
\DeclareTOCStyleEntry[beforeskip=0cm]{section}{section}
\DeclareTOCStyleEntry[beforeskip=0cm]{default}{subsection}

% colons after entry names
\BeforeStartingTOC[lof]{\def\autodot{:}}
\BeforeStartingTOC[lot]{\def\autodot{:}}
\BeforeStartingTOC[lol]{\def\autodot{:}}


% ##################################################
% BIBLIOGRAPHY
% ##################################################
\iffalse
\usepackage{csquotes} % context sensitive quotation
\setlength\bibitemsep{.5\baselineskip} % increase spacing between entries
\setcounter{biburlnumpenalty}{9000} % break URLs on numbers
\setcounter{biburllcpenalty}{9000}  % break URLs on lower case letters
\setcounter{biburlucpenalty}{9000}  % break URLs on upper case letters

\fi

% ##################################################
% ABBREVIATIONS
% ##################################################
\usepackage[printonlyused]{acronym}


% ##################################################
% APPENDIX
% ##################################################
\usepackage[title,titletoc]{appendix}

% appendix chapter
\newcommand{\appendixchapter}[1]{
\cleardoublepage
\pagenumbering{arabic}
\renewcommand{\thepage}{\thechapter-\arabic{page}}
\chapter{#1}
}

% insert monthly report pdf as picture in order to keep page header
\newcommand{\monthlyreport}[2]{
\section{#1}
\centering
\includegraphics[trim=55 35 55 35,clip,width=1\textwidth]{#2}
\clearpage
}


% ##################################################
% Theoreme
% ##################################################

% Umgebung fuer Beispiele
\newtheorem{beispiel}{Beispiel}

% Umgebung fuer These
\newtheorem{these}{These}

% Umgebung fuer Definitionen
\newtheorem{definition}{Definition}


% ##################################################
% MISC
% ##################################################
% better referencing of images, tables, etc.
\usepackage[nameinlink, noabbrev]{cleveref}


% ############################################################################
% WIE MACH ICH DAS HIER?
% 
% 1. Schreibe im "titlepage" Abschnitt den Titel in das element mit "\fontsize{22}{22}"
% 2. Aus obsidian raus kopieren mit "Copy to LaTeX"
% 3. Zwischen "CONTENT STARTS HERE" und "CONTENT ENDS HERE" den Inhalt einfügen
% 4. Sections anpassen. Damit man ne gescheite Chapter > Section > Subsection Struktur hat
% 5. Leere Seite nach Titelseite am Anfang mit {\let\cleardoublepage\relax \chapter{Erstes Kapitel}} verhindern
%
% TABELLEN
% Müssen extra gemacht werden, da obsidian das nicht unterstützt
% Kann ich nur empfehlen: https://tableconvert.com/markdown-to-latex
%
% BILDER
% Muss man vermutlich auch extra machen, hab ich aber noch nicht probiert
% ############################################################################

\usepackage{amssymb}

\title{Betriebssysteme Aufgabenblatt 1 - Lösungen}
\author{Luis Staudt}
\date{}

\begin{document}

\begin{titlepage}
	\pagestyle{empty}
	
	% HFU Logo
	\begin{flushright}
		\begin{figure}[ht]
			\flushright
			\includegraphics[height=2cm]{../../for_latex/hfu}
		\end{figure}
	\end{flushright}
	
	\begin{center}
		\vspace{3cm}
		
		{\fontsize{22}{22} \selectfont \textbf{Blatt 1}}\\[5mm]
		{\fontsize{18}{18} \selectfont Betriebssysteme}
		
		\vspace{12cm}
		
		{\fontsize{14}{14} \selectfont Luis Staudt}
	\end{center}
\end{titlepage}

% ############################################################################
% CONTENT STARTS HERE
% ############################################################################

\section*{a) Seitenfehler bei zweidimensionalem Feld}

Ausschnitt A (spaltenweise Zugriff):
2048 + 1 = 2049
Ausschnitt B verursacht 33 Seitenfehler (1 für das Programm und 32 für die zeilenweise geladenen Datenseiten),
da die Speicherzugriffe der natürlichen row-major Anordnung folgen und jede Seite nur einmal geladen wird.

Ausschnitt B (zeilenweise Zugriff):
32 + 1 = 33
Ausschnitt A verursacht hingegen 2.049 Seitenfehler (1 für das Programm und 2.048 für die Datenzugriffe),
weil der spaltenweise Zugriff auf zeilenweise gespeicherte Daten bei nur 3 verfügbaren Datenrahmen zu ständigem Ein- und Auslagern aller 32 Datenseiten für jede der 64 Spalten führt.

\section*{b) Seitentabelle mit virtuellen und physischen Adressen}

Bei einer virtuellen Adresse von 48 Bit und einer Seitengröße von 8 KiB (= $2^{13}$ Byte) werden 13 Bit für den Offset innerhalb einer Seite verwendet, somit bleiben 35 Bit ($48-13$) für die Seitennummer.
Die Seitentabelle enthält daher $2^{35} = 34.359.738.368$ Einträge, wobei die Information über die 32-Bit physische Adresse für diese Berechnung nicht benötigt wird.

\section*{c) Invertierte Seitentabellen und Hashtabelle}

Die invertierte Seitentabelle enthält einen Eintrag für jeden der $2^{15} (32.768)$ physischen Seitenrahmen $(256 MiB / 8 KiB)$.
Damit durchschnittlich weniger als ein Eintrag pro Hashtabellen-Slot existiert, muss die Hashtabelle mindestens $2^{16} (65.536)$ Einträge umfassen,
da dies die nächstgrößere Zweierpotenz ist, die größer als die Anzahl der physischen Rahmen ist.

\section*{d) Scheduling-Strategien und durchschnittliche Prozessdurchlaufzeit}

\subsection*{1. Round Robin (Multiprogramming mit fairer CPU-Aufteilung)}

Bei dieser Strategie wird jedem Job reihum eine kleine Zeitscheibe zugeteilt; wenn wir 1-Minuten-Zeitscheiben in der Reihenfolge A-B-C-D-E annehmen,
endet Job C nach 10 Minuten, Job D nach 18 Minuten, Job C nach 24 Minuten, Job D nach 28 Minuten und Job E nach 30 Minuten.
Die durchschnittliche Durchlaufzeit beträgt (10 + 18 + 24 + 28 + 30)/5 = 22 Minuten.

\subsection*{2. Prioritätsscheduling}

Bei Ausführung nach absteigender Priorität (B-E-A-C-D) ergibt sich:
Job B nach 6 Minuten fertig, Job E nach 14 Minuten (6+8), Job A nach 24 Minuten (14+10), Job C nach 26 Minuten (24+2) und Job D nach 30 Minuten (26+4).
Die durchschnittliche Durchlaufzeit beträgt (6 + 14 + 24 + 26 + 30)/5 = 20 Minuten.

\subsection*{3. First Come First Serve}

Bei der Ankunftsreihenfolge A-B-C-D-E ergibt sich: Job A nach 10 Minuten fertig, Job B nach 16 Minuten (10 + 6),
Job C nach 18 Minuten (16 + 2), Job D nach 22 Minuten (18 + 4) und Job E nach 30 Minuten (22 + 8).
Die durchschnittliche Durchlaufzeit beträgt (10 + 16 + 18 + 22 + 30)/5 = 19,2 Minuten.

\subsection*{4. Shortest Job First}

Bei Ausführung nach kürzester Laufzeit (C-D-B-E-A) ergibt sich:
Job C nach 2 Minuten fertig, Job D nach 6 Minuten (2 + 4), Job B nach 12 Minuten (6 + 6), Job E nach 20 Minuten (12 + 8) und Job A nach 30 Minuten (20 + 10).
Die durchschnittliche Durchlaufzeit beträgt (2 + 6 + 12 + 20 + 30)/5 = 14 Minuten.

\section*{e) CPU-Effizienz bei Round-Robin-Scheduling}

\subsection*{1. Warum gilt für Q = $\infty$ und Q > T dieselbe Effizienz?}

Wenn Q > T oder Q = $\infty$, wird der Prozess immer nach Zeit T blockieren, bevor sein Quantum abläuft.
In beiden Fällen läuft der Prozess für die Zeit T und blockiert dann, was zu identischem Verhalten führt.

\subsection*{2. CPU-Effizienz für Q = $\infty$ bzw. Q > T}

In diesem Fall läuft ein Prozess bis er nach Zeit T blockiert.
Dann erfolgt ein Prozesswechsel, der Zeit S dauert. Dieser Zyklus wiederholt sich.

In jedem Zyklus beträgt die Rechenzeit T und die Gesamtzeit T + S. Daher ist die CPU-Effizienz:

$$\eta = \frac{T}{T + S}$$

\subsection*{3. CPU-Effizienz für S < Q < T (T sehr viel größer als Q)}

In diesem Fall läuft ein Prozess für sein gesamtes Quantum Q, dann erfolgt ein Prozesswechsel, der Zeit S dauert.
Dieser Zyklus wiederholt sich, bis der Prozess insgesamt Zeit T gelaufen ist, woraufhin er blockiert.

Die Anzahl der Ausführungen eines Prozesses vor dem Blockieren ist ungefähr T/Q (da T $\gg$ Q). In jedem Zyklus beträgt die Rechenzeit Q und die Gesamtzeit Q + S. Für T/Q solche Zyklen beträgt die Gesamtrechenzeit T und die Gesamtzeit (T/Q) $\cdot$ (Q + S) = T + (T$\cdot$S)/Q.

Daher ist die CPU-Effizienz:

$$\eta = \frac{T}{T + \frac{T \cdot S}{Q}} = \frac{1}{1 + \frac{S}{Q}}$$

\subsection*{4. CPU-Effizienz für Q = S}

Mit der Formel aus III erhalten wir:

$$\eta = \frac{1}{1 + \frac{S}{Q}} = \frac{1}{1 + \frac{S}{S}} = \frac{1}{2}$$

Die Effizienz beträgt 50\%.

\subsection*{5. CPU-Effizienz für Q $\approx$ 0}

Wenn Q sehr klein ist, dominiert der Overhead des Prozesswechsels.
Mit der Formel aus III:

$$\eta = \frac{1}{1 + \frac{S}{Q}} \approx 0 \text{ für } Q \approx 0$$

Die Effizienz nähert sich 0\%.
Da das System hauptsächlich mit Prozesswechseln beschäftigt ist, wird die CPU nicht genutzt.

\section*{f) Einfluss von Seitenfehlern auf die Speicherzugriffszeit}

\subsection*{Zuweisung zahl 0}

Java nutzt referenzen auf Objekte, somit verweisen die gleichen Zahlen immer auf die gleiche stelle im Speicher,
somit wird für das gesamte Array nur ein einziger Speicherplatz reserviert.

\subsection*{Prozess mit ausreichend großem Heap}

Nein, deshalb mehrere Prozesse mit einem Heap von 8 GB.

\subsection*{Spalte Seitenfehler/s im Resourcemonitor}

Nach bereits 4 Prozessen waren bereits >300 Fehler/s pro Prozess.

\subsection*{Zeiten pro inneren Schleifendurchlauf}

Es waren keine Unterschiede zu erkennen, ich vermute aber das mit Seitenfehlern die Zeiten steigen sollten.

\subsection*{Swapfile Verwendung}

Vor dem Start 0 GB, danach ca zwischen 0 und 2 GB.
Kurzzeitig auf 4 GB, aber dann wieder runter.

\subsection*{CPU Auslastung}

Die CPU war nie bei 100\% Auslastung, aber die Auslastung war immer zwischen 50 und 80\%.
Die Auslastung von nur 50\5 bis 80\% lässt sich vermutlich da durch Erklären, dass die Prozesse nicht CPU intensiv sind und nicht alle Kerne ausgelastet werden.

% ############################################################################
% CONTENT ENDS HERE
% ############################################################################

\end{document}
