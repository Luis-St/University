{\let\cleardoublepage\relax \chapter*{Kategorie 1: Typfehler}}

\section*{1. Zuweisung falscher Datentypen}

\subsection*{Beschreibung:}\\
Bei der Zuweisung von Werten an Variablen muss der Datentyp des Wertes mit dem Datentyp der Variable kompatibel sein.
Java ist eine streng typisierte Sprache und erlaubt keine automatische Konvertierung zwischen inkompatiblen Typen.

\subsection*{Beispiele:}
\begin{lstlisting}
// Beispiel 1: int zu boolean
boolean flag = 42;  // FEHLER: int kann nicht zu boolean werden

// Beispiel 2: String zu int
int number = "123";  // FEHLER: String ist kein int

// Beispiel 3: double zu int ohne Cast
int x = 3.14;  // FEHLER: double kann nicht ohne Cast zu int werden

// Beispiel 4: char zu boolean
boolean isTrue = 'A';  // FEHLER: char ist kein boolean

// Beispiel 5: Array zu einzelnem Wert
int value = new int[5];  // FEHLER: Array kann nicht zu int werden
\end{lstlisting}

\begin{achtung}
	\begin{itemize}
		\item Links und rechts vom \texttt{=} müssen zusammenpassen
		\item Zahlen (wie 42) sind keine Wahrheitswerte (true/false)
		\item Text in Anführungszeichen ist immer ein String
		\item Kommazahlen (3.14) passen nicht in ganze Zahlen (int)
	\end{itemize}
\end{achtung}

% ============================================================
\newpage
\section*{2. Vergleich inkompatibler Typen}

\subsection*{Beschreibung:}\\
Beim Vergleich mit \texttt{==}, \texttt{!=}, \texttt{<}, \texttt{>}, \texttt{<=}, \texttt{>=} müssen die Operanden kompatible Typen haben.
Bei Objekten (wie String) sollte \texttt{.equals()} statt \texttt{==} verwendet werden.


\subsection*{Beispiele:}
\begin{lstlisting}
// Beispiel 1: String mit int vergleichen
String text = "Hello";
int number = 5;
if (text == number) { }  // FEHLER: String und int sind inkompatibel

// Beispiel 2: boolean mit int vergleichen
boolean flag = true;
if (flag == 1) { }  // FEHLER: boolean und int sind nicht vergleichbar

// Beispiel 3: double mit String vergleichen
double pi = 3.14;
if (pi > "3") { }  // FEHLER: double kann nicht mit String verglichen werden

// Beispiel 4: Array mit einzelnem Wert
int[] numbers = {1, 2, 3};
if (numbers == 1) { }  // FEHLER: Array kann nicht mit int verglichen werden

// Beispiel 5: char mit String
char c = 'A';
String s = "A";
if (c == s) { }  // FEHLER: char und String sind verschiedene Typen
\end{lstlisting}

\begin{achtung}
	\begin{itemize}[leftmargin=*]
		\item Man kann nur gleiche Datentypen vergleichen
		\item Zahlen mit Zahlen, Text mit Text
		\item true/false nur mit true/false vergleichen
		\item Bei Text: benutze \texttt{.equals()} statt \texttt{==}
	\end{itemize}
\end{achtung}

% ============================================================
\newpage
\section*{3. Verwendung von Operatoren mit falschen Typen}

\subsection*{Beschreibung:}\\
Arithmetische Operatoren (+, -, *, /, \%) funktionieren nur mit numerischen Typen.
Logische Operatoren (\&\&, ||, !) nur mit boolean.
Der +-Operator bei Strings ist eine Ausnahme für Konkatenation.

\subsection*{Beispiele:}
\begin{lstlisting}
// Beispiel 1: boolean addieren
boolean a = true;
int result = a + 5;  // FEHLER: boolean kann nicht addiert werden

// Beispiel 2: String subtrahieren
String text = "Hello";
String result = text - "ll";  // FEHLER: - funktioniert nicht mit Strings

// Beispiel 3: int mit && verknuepfen
int x = 5;
int y = 10;
if (x && y) { }  // FEHLER: && braucht boolean-Operanden

// Beispiel 4: String multiplizieren
String s = "abc";
String result = s * 3;  // FEHLER: * funktioniert nicht mit Strings

// Beispiel 5: boolean modulo
boolean b1 = true;
boolean b2 = false;
boolean result = b1 % b2;  // FEHLER: % funktioniert nicht mit boolean
\end{lstlisting}

\begin{achtung}
	\begin{itemize}[leftmargin=*]
		\item Rechnen (+, -, *, /) geht nur mit Zahlen
		\item \texttt{\&\&} und \texttt{||} gehen nur mit true/false
		\item Text kann man nur mit \texttt{+} zusammenfügen
		\item true/false kann man nicht rechnen
	\end{itemize}
\end{achtung}

% ============================================================
\newpage
\section*{4. Methodenaufrufe mit falschen Datentypen}

\subsection*{Beschreibung:}\\
Beim Aufruf einer Methode müssen die übergebenen Argumente mit den deklarierten Parametertypen übereinstimmen.
Die Reihenfolge und Anzahl müssen ebenfalls passen.

\subsection*{Beispiele:}
\begin{lstlisting}
// Beispiel 1: int-Methode mit double aufrufen
public static boolean isEven(int number) {
    return number % 2 == 0;
}
// Aufruf:
double x = 4.5;
boolean result = isEven(x);  // FEHLER: double passt nicht zu int

// Beispiel 2: String-Methode mit int aufrufen
public static void printMessage(String msg) {
    System.out.println(msg);
}
// Aufruf:
printMessage(42);  // FEHLER: int ist kein String

// Beispiel 3: Mehrere Parameter in falscher Reihenfolge
public static void display(String name, int age) {
    System.out.println(name + " ist " + age);
}
// Aufruf:
display(25, "Anna");  // FEHLER: Reihenfolge ist falsch

// Beispiel 4: boolean statt int
public static int multiply(int a, int b) {
    return a * b;
}
// Aufruf:
int result = multiply(5, true);  // FEHLER: boolean statt int

// Beispiel 5: Array statt einzelner Wert
public static void printNumber(int num) {
    System.out.println(num);
}
// Aufruf:
int[] numbers = {1, 2, 3};
printNumber(numbers);  // FEHLER: Array statt int
\end{lstlisting}

\begin{achtung}
	\begin{itemize}[leftmargin=*]
		\item Schau dir an, was die Methode erwartet
		\item Die Werte müssen in der richtigen Reihenfolge sein
		\item Wenn Methode \texttt{int} will, gib kein \texttt{String}
		\item Kommazahlen sind nicht das gleiche wie ganze Zahlen
	\end{itemize}
\end{achtung}
