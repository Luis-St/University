\begingroup
\let\cleardoublepage\relax


\chapter*{Kategorie 6: Array-Fehler}
\label{ch:kategorie-6:-array-fehler}
\addcontentsline{toc}{chapter}{Kategorie 6: Array-Fehler}
\endgroup


\section*{21. Falsche Array-Deklaration}
\addcontentsline{toc}{section}{21. Falsche Array-Deklaration}

\subsection*{Beschreibung:}\\
Die Größe eines Arrays wird bei der Deklaration nicht angegeben, sondern erst bei der Initialisierung.
Die eckigen Klammern gehören zum Typ, nicht zur Variable.

\subsection*{Beispiele:}
\begin{lstlisting}
// Beispiel 1: Groesse bei Deklaration
int[5] numbers;  // FEHLER: Groesse nicht bei Deklaration

// Korrekt waere:
int[] numbers = new int[5];
// oder
int[] numbers;
numbers = new int[5];

// Beispiel 2: Klammern am falschen Ort
int numbers[5];  // FEHLER: Groesse nicht erlaubt

// Beispiel 3: Falsche Syntax mit String
String[10] names;  // FEHLER: Groesse nicht bei Deklaration

// Beispiel 4: Mehrere Dimensionen falsch
int[3][4] matrix;  // FEHLER: Groessen nicht bei Deklaration

// Beispiel 5: Boolean-Array
boolean[20] flags;  // FEHLER: Groesse nicht hier
\end{lstlisting}

\begin{achtung}
	\begin{itemize}
		\item Erst sagen WAS: \texttt{int[] numbers;}
		\item Dann sagen WIE VIELE: \texttt{numbers = new int[5];}
		\item Nie Zahlen in die \texttt{[]} bei der Deklaration
		\item Die Zahl kommt später beim \texttt{new}
	\end{itemize}
\end{achtung}

% ============================================================
\newpage


\section*{22. Falsche Array-Initialisierung}
\addcontentsline{toc}{section}{22. Falsche Array-Initialisierung}

\subsection*{Beschreibung:}\\
Bei der Array-Initialisierung mit Werten darf keine Größe angegeben werden, wenn man die Werte in geschweiften Klammern auflistet.

\subsection*{Beispiele:}
\begin{lstlisting}
// Beispiel 1: Groesse und Werte gleichzeitig
int[] numbers = new int[3] {1, 2, 3};  // FEHLER: entweder Groesse oder Werte

// Korrekt waere:
int[] numbers = new int[3];  // Groesse ohne Werte
// oder
int[] numbers = {1, 2, 3};  // Werte ohne Groesse/new
// oder
int[] numbers = new int[] {1, 2, 3};  // new ohne Groesse

// Beispiel 2: String-Array
String[] names = new String[2] {"Max", "Anna"};  // FEHLER

// Korrekt:
String[] names = {"Max", "Anna"};

// Beispiel 3: Double-Array
double[] values = new double[4] {1.1, 2.2, 3.3, 4.4};  // FEHLER

// Korrekt:
double[] values = {1.1, 2.2, 3.3, 4.4};

// Beispiel 4: Boolean-Array
boolean[] flags = new boolean[3] {true, false, true};  // FEHLER

// Korrekt:
boolean[] flags = {true, false, true};

// Beispiel 5: Char-Array
char[] letters = new char[5] {'a', 'b', 'c', 'd', 'e'};  // FEHLER

// Korrekt:
char[] letters = {'a', 'b', 'c', 'd', 'e'};
\end{lstlisting}

\begin{achtung}
	\begin{itemize}
		\item Entweder Größe ODER Werte angeben
		\item Mit Werten: \texttt{\{1, 2, 3\}} - Java zählt selbst
		\item Mit Größe: \texttt{new int[5]} - leer, füllst du später
		\item Beides zusammen geht nicht!
	\end{itemize}
\end{achtung}
