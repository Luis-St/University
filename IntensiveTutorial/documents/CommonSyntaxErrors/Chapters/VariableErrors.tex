\begingroup
\let\cleardoublepage\relax


\chapter*{Kategorie 2: Variablen-Fehler}
\label{ch:kategorie-2:-variablen-fehler}
\addcontentsline{toc}{chapter}{Kategorie 2: Variablen-Fehler}
\endgroup


\section*{5. Verwendung nicht deklarierter Variablen}
\addcontentsline{toc}{section}{5. Verwendung nicht deklarierter Variablen}

\subsection*{Beschreibung:}\\
Jede Variable muss deklariert werden, bevor sie verwendet wird.
Java muss wissen, welcher Datentyp die Variable hat.

\subsection*{Beispiele:}\\
\begin{lstlisting}
// Beispiel 1: Variable nie deklariert
public static void main(String[] args) {
    System.out.println(name);  // FEHLER: name wurde nie deklariert
}

// Beispiel 2: Tippfehler im Variablennamen
int counter = 0;
counter++;
System.out.println(countr);  // FEHLER: Tippfehler (countr statt counter)

// Beispiel 3: Variable ausserhalb des Scopes
if (true) {
    int localVar = 5;
}
System.out.println(localVar);  // FEHLER: localVar existiert hier nicht mehr

// Beispiel 4: Falsche Gross-/Kleinschreibung
int myValue = 10;
System.out.println(MyValue);  // FEHLER: Java ist case-sensitive

// Beispiel 5: Variable in falscher Reihenfolge verwendet
System.out.println(result);  // FEHLER: result wird erst spaeter deklariert
int result = 5;
\end{lstlisting}

\begin{achtung}
	\begin{itemize}
		\item Erst \texttt{int x;} schreiben, dann \texttt{x} benutzen
		\item Achte auf Schreibfehler im Namen
		\item Groß- und Kleinbuchstaben machen einen Unterschied
		\item Variable muss vorher im Code stehen
	\end{itemize}
\end{achtung}

% ============================================================
\newpage


\section*{6. Verwendung nicht initialisierter Variablen}
\addcontentsline{toc}{section}{6. Verwendung nicht initialisierter Variablen}

\subsection*{Beschreibung:}\\
Lokale Variablen müssen initialisiert werden, bevor sie gelesen werden.
Anders als Instanzvariablen erhalten lokale Variablen keinen Standardwert.

\subsection*{Beispiele:}\\
\begin{lstlisting}
// Beispiel 1: Einfache nicht initialisierte Variable
public static void main(String[] args) {
    int x;
    int y = x + 5;  // FEHLER: x wurde nicht initialisiert
}

// Beispiel 2: Bedingte Initialisierung
int value;
if (Math.random() > 0.5) {
    value = 10;
}
System.out.println(value);  // FEHLER: value koennte uninitialisiert sein

// Beispiel 3: Variable in Schleife
int sum;
for (int i = 0; i < 10; i++) {
    sum += i;  // FEHLER: sum wurde nicht initialisiert
}

// Beispiel 4: String ohne Initialisierung
String message;
message = message.toUpperCase();  // FEHLER: message hat keinen Wert

// Beispiel 5: Array-Element
int[] numbers = new int[5];
int first = numbers[0];  // OK: Arrays werden mit 0 initialisiert
int standalone;
int test = standalone;  // FEHLER: lokale Variable nicht initialisiert
\end{lstlisting}

\begin{achtung}
	\begin{itemize}
		\item Schreibe \texttt{int x = 0;} statt nur \texttt{int x;}
		\item Die Variable braucht einen Startwert
		\item Java gibt dir keinen automatischen Wert
		\item Bei if/else: beide Wege müssen einen Wert geben
	\end{itemize}
\end{achtung}

% ============================================================
\newpage


\section*{7. Zugriff auf nicht-statische Variablen aus statischem Kontext}
\addcontentsline{toc}{section}{7. Zugriff auf nicht-statische Variablen aus statischem Kontext}

\subsection*{Beschreibung:}\\
Statische Methoden (wie main) können nicht direkt auf nicht-statische (Instanz-) Variablen zugreifen, da diese zu einem Objekt gehören, das möglicherweise nicht existiert.

\subsection*{Beispiele:}\\
\begin{lstlisting}
// Beispiel 1: Direkte Verwendung in main
public class Test {
    int instanceVar = 10;
    
    public static void main(String[] args) {
        System.out.println(instanceVar);  // FEHLER: nicht-static in static
    }
}

// Beispiel 2: In statischer Methode
public class Calculator {
    double result = 0;
    
    public static void reset() {
        result = 0;  // FEHLER: result ist nicht static
    }
}

// Beispiel 3: String-Variable
public class Program {
    String name = "Test";
    
    public static void printInfo() {
        System.out.println(name);  // FEHLER: name ist nicht static
    }
}

// Beispiel 4: Array-Variable
public class Data {
    int[] numbers = {1, 2, 3};
    
    public static void process() {
        int first = numbers[0];  // FEHLER: numbers ist nicht static
    }
}

// Beispiel 5: boolean-Variable
public class App {
    boolean isActive = true;
    
    public static void check() {
        if (isActive) { }  // FEHLER: isActive ist nicht static
    }
}
\end{lstlisting}

\begin{achtung}
	\begin{itemize}
		\item \texttt{static} bedeutet: gehört zur Klasse selbst
		\item Ohne \texttt{static}: gehört zu einem einzelnen Objekt
		\item In \texttt{main} kannst du nur \texttt{static} Sachen direkt benutzen
		\item Lösung: Schreibe \texttt{static} vor die Variable
	\end{itemize}
\end{achtung}

% ============================================================
\newpage


\section*{8. Zugriff auf lokale Variablen außerhalb ihres Gültigkeitsbereichs}
\addcontentsline{toc}{section}{8. Zugriff auf lokale Variablen außerhalb ihres Gültigkeitsbereichs}

\subsection*{Beschreibung:}\\
Variablen, die in einem Block (z.B. if, while, for) deklariert werden, existieren nur innerhalb dieses Blocks.
Nach der schließenden Klammer sind sie nicht mehr verfügbar.

\subsection*{Beispiele:}\\
\begin{lstlisting}
// Beispiel 1: if-Block
public static void main(String[] args) {
    if (true) {
        int x = 5;
    }
    System.out.println(x);  // FEHLER: x existiert hier nicht
}

// Beispiel 2: for-Schleife
public static void main(String[] args) {
    for (int i = 0; i < 10; i++) {
        int sum = i * 2;
    }
    System.out.println(sum);  // FEHLER: sum existiert hier nicht
}

// Beispiel 3: while-Schleife
public static void main(String[] args) {
    while (true) {
        String message = "Loop";
        break;
    }
    System.out.println(message);  // FEHLER: message existiert hier nicht
}

// Beispiel 4: Verschachtelte Bloecke
public static void main(String[] args) {
    {
        int temp = 100;
    }
    int value = temp;  // FEHLER: temp existiert hier nicht
}

// Beispiel 5: Switch-Case
public static void main(String[] args) {
    int choice = 1;
    switch (choice) {
        case 1:
            String result = "One";
            break;
    }
    System.out.println(result);  // FEHLER: result existiert hier nicht
}
\end{lstlisting}

\begin{achtung}
	\begin{itemize}
		\item Variablen leben nur zwischen \texttt{\{} und \texttt{\}}
		\item Nach der \texttt{\}} ist die Variable weg
		\item Wenn du sie später brauchst: Schreibe sie vor die \texttt{\{}
		\item Schleifenvariablen wie \texttt{i} existieren nur in der Schleife
	\end{itemize}
\end{achtung}
