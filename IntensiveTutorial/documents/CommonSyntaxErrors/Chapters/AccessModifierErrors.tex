{\let\cleardoublepage\relax \chapter*{Kategorie 5: Zugriffsmodifikator-Fehler}}

\section*{19. Zugriff auf private Variablen von außen}

\subsection*{Beschreibung:}\\
Private Variablen und Methoden können nur innerhalb derselben Klasse verwendet werden.
Von außen (andere Klassen) sind sie nicht sichtbar.

\subsection*{Beispiele:}
\begin{lstlisting}
// Beispiel 1: Einfacher Zugriff von aussen
class MyClass {
    private int secret = 42;
}

public class Test {
    public static void main(String[] args) {
        MyClass obj = new MyClass();
        System.out.println(obj.secret);  // FEHLER: secret ist private
    }
}

// Beispiel 2: Private String
class Person {
    private String password = "1234";
}

public class App {
    public static void main(String[] args) {
        Person p = new Person();
        String pw = p.password;  // FEHLER: password ist private
    }
}

// Beispiel 3: Private Methode
class Calculator {
    private int multiply(int a, int b) {
        return a * b;
    }
}

public class Main {
    public static void main(String[] args) {
        Calculator calc = new Calculator();
        int result = calc.multiply(5, 3);  // FEHLER: multiply ist private
    }
}

// Beispiel 4: Private Array
class DataHolder {
    private int[] data = {1, 2, 3};
}

public class Processor {
    public static void main(String[] args) {
        DataHolder holder = new DataHolder();
        int first = holder.data[0];  // FEHLER: data ist private
    }
}

// Beispiel 5: Aenderung von private Variable
class Counter {
    private int count = 0;
}

public class Program {
    public static void main(String[] args) {
        Counter c = new Counter();
        c.count = 10;  // FEHLER: count ist private
    }
}
\end{lstlisting}

\begin{achtung}
	\begin{itemize}
		\item \texttt{private} = nur in der eigenen Klasse nutzbar
		\item Von außen nicht sichtbar
		\item Lösung: Getter/Setter Methoden benutzen
		\item Oder: \texttt{private} durch \texttt{public} ersetzen
	\end{itemize}
\end{achtung}

% ============================================================
\newpage
\section*{20. Zugriff auf nicht-statische Elemente aus statischem Kontext}

\subsection*{Beschreibung:}\\
Dies ist ein Sonderfall von \#7 und \#9, aber so häufig, dass er nochmal erwähnt wird: Statische Methoden können nicht direkt auf Instanzvariablen oder -methoden zugreifen.

\subsection*{Beispiele:}
\begin{lstlisting}
// Beispiel 1: main greift auf Instanzvariable zu
public class Test {
    int number = 10;
    
    public static void main(String[] args) {
        System.out.println(number);  // FEHLER: number ist nicht static
    }
}

// Beispiel 2: Static Methode ruft Instanzmethode auf
public class Helper {
    public void help() {
        System.out.println("Helping...");
    }
    
    public static void doWork() {
        help();  // FEHLER: help ist nicht static
    }
}

// Beispiel 3: String-Variable in main
public class App {
    String message = "Hello";
    
    public static void main(String[] args) {
        System.out.println(message);  // FEHLER: message ist nicht static
    }
}

// Beispiel 4: Instanzvariable in statischer Hilfsmethode
public class Calculator {
    double result = 0;
    
    public static void reset() {
        result = 0;  // FEHLER: result ist nicht static
    }
    
    public static void main(String[] args) {
        reset();
    }
}

// Beispiel 5: Array-Zugriff
public class Data {
    int[] values = {1, 2, 3, 4, 5};
    
    public static void printFirst() {
        System.out.println(values[0]);  // FEHLER: values ist nicht static
    }
}
\end{lstlisting}

\begin{achtung}
	\begin{itemize}
		\item Ist in \texttt{main}? Dann nur \texttt{static} Sachen benutzen
		\item Lösung: Schreibe \texttt{static} vor die Variable
		\item Oder: Erstelle ein Objekt mit \texttt{new}
		\item \texttt{main} ist immer static - häufigster Fehler!
	\end{itemize}
\end{achtung}
