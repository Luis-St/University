{\let\cleardoublepage\relax \chapter*{Kategorie 7: Klammer und Syntax-Fehler}}

\section*{23. Fehlende geschweifte Klammern}

\subsection*{Beschreibung:}\\
Jede öffnende geschweifte Klammer \{ braucht eine schließende \}.
Dies betrifft Klassen, Methoden, Schleifen und Bedingungen.

\subsection*{Beispiele:}
\begin{lstlisting}
// Beispiel 1: Methode nicht geschlossen
public class Test {
    public static void main(String[] args) {
        System.out.println("Hello");
    // FEHLER: } fehlt fuer main
    
    public static void other() {
        System.out.println("Other");
    }
}

// Beispiel 2: if-Block nicht geschlossen
public static void check(int x) {
    if (x > 0) {
        System.out.println("Positive");
        // FEHLER: } fehlt fuer if
}

// Beispiel 3: Verschachtelte Bloecke
public static void complex() {
    for (int i = 0; i < 10; i++) {
        if (i % 2 == 0) {
            System.out.println(i);
        // FEHLER: } fehlt fuer if
    }
}

// Beispiel 4: Klasse nicht geschlossen
public class MyClass {
    private int value;
    
    public void setValue(int v) {
        this.value = v;
    }
    // FEHLER: } fehlt fuer Klasse

// Beispiel 5: while-Schleife
public static void loop() {
    int i = 0;
    while (i < 5) {
        System.out.println(i);
        i++;
        // FEHLER: } fehlt
}
\end{lstlisting}

\begin{achtung}
	\begin{itemize}
		\item Jede \texttt{\{} braucht eine passende \texttt{\}}
		\item Zähle die Klammern: gleich viele auf und zu?
		\item Bei Verwirrung: jede Methode einzeln prüfen
		\item IDEs zeigen oft an, wo eine \texttt{\}} fehlt
	\end{itemize}
\end{achtung}

% ============================================================
\newpage
\section*{24. Fehlende Semikolons}

\subsection*{Beschreibung:}\\
Jede Anweisung (Statement) in Java muss mit einem Semikolon ; enden.
Ausnahmen sind Klassen-, Methoden- und Block-Deklarationen.

\subsection*{Beispiele:}
\begin{lstlisting}
// Beispiel 1: Variable ohne Semikolon
public static void main(String[] args) {
    int x = 5  // FEHLER: ; fehlt
    System.out.println(x);
}

// Beispiel 2: Methodenaufruf
public static void test() {
    System.out.println("Hello")  // FEHLER: ; fehlt
    System.out.println("World");
}

// Beispiel 3: Return-Statement
public static int getValue() {
    return 42  // FEHLER: ; fehlt
}

// Beispiel 4: Zuweisung
public static void calculate() {
    int a = 10;
    int b = 20;
    int sum = a + b  // FEHLER: ; fehlt
    System.out.println(sum);
}

// Beispiel 5: Mehrere Anweisungen
public static void demo() {
    int x = 1  // FEHLER: ; fehlt
    int y = 2  // FEHLER: ; fehlt
    int z = x + y  // FEHLER: ; fehlt
}
\end{lstlisting}

\begin{achtung}
	\begin{itemize}
		\item Fast jede Zeile mit Code braucht \texttt{;} am Ende
		\item Ausnahme: \texttt{\{}, \texttt{\}}, \texttt{if}, \texttt{while} etc.
		\item Nach \texttt{=} kommt erst der Wert, dann \texttt{;}
		\item Vergiss nie das \texttt{;} am Zeilenende!
	\end{itemize}
\end{achtung}

% ============================================================
\newpage
\section*{25. Fehlende runde Klammern}

\subsection*{Beschreibung:}\\
Kontrollstrukturen (if, while, for) und Methodenaufrufe benötigen runde Klammern.
Die Bedingung oder Parameter müssen in ( ) stehen.

\subsection*{Beispiele:}
\begin{lstlisting}
// Beispiel 1: if ohne Klammern
public static void check(boolean flag) {
    if flag {  // FEHLER: () fehlen
        System.out.println("True");
    }
}

// Korrekt:
if (flag) {
    System.out.println("True");
}

// Beispiel 2: while ohne Klammern
public static void loop() {
    int i = 0;
    while i < 10 {  // FEHLER: () fehlen
        i++;
    }
}

// Beispiel 3: for-Schleife
public static void iterate() {
    for int i = 0; i < 5; i++ {  // FEHLER: () fehlen um gesamte for-Deklaration
        System.out.println(i);
    }
}

// Beispiel 4: Methodenaufruf
public static void greet() {
    System.out.println;  // FEHLER: () fehlen
}

// Beispiel 5: Verschachtelte Bedingung
public static void complex(int x, int y) {
    if x > 0 && y > 0 {  // FEHLER: () fehlen
        System.out.println("Both positive");
    }
}
\end{lstlisting}

\begin{achtung}
	\begin{itemize}
		\item \texttt{if}, \texttt{while}, \texttt{for}: Bedingung in \texttt{()} setzen
		\item Methoden aufrufen: immer \texttt{()} dahinter
		\item Auch ohne Parameter: \texttt{println()} nicht \texttt{println}
		\item Beide Klammern nicht vergessen: \texttt{(} und \texttt{)}
	\end{itemize}
\end{achtung}
