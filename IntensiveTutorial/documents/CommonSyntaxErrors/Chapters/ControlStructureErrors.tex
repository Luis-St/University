\begingroup
\let\cleardoublepage\relax


\chapter*{Kategorie 4: Kontrollstruktur-Fehler}
\label{ch:kategorie-4:-kontrollstruktur-fehler}
addcontentsline{toc}{chapter}{Kategorie 4: Kontrollstruktur-Fehler}
\endgroup


\section*{17. Verwendung von nicht-boolean Ausdrücken in if-Bedingungen}
\addcontentsline{toc}{section}{17. Verwendung von nicht-boolean Ausdrücken in if-Bedingungen}

\subsection*{Beschreibung:}\\
Die Bedingung in einer if-Anweisung muss immer einen boolean-Wert (true oder false) ergeben.
Zahlen oder andere Typen sind nicht erlaubt.

\subsection*{Beispiele:}
\begin{lstlisting}
// Beispiel 1: int statt boolean
int x = 5;
if (x) {  // FEHLER: x ist int, nicht boolean
    System.out.println("true");
}

// Beispiel 2: String-Bedingung
String name = "Max";
if (name) {  // FEHLER: String ist nicht boolean
    System.out.println("Name exists");
}

// Beispiel 3: Zuweisung statt Vergleich
int value = 10;
if (value = 5) {  // FEHLER: = ist Zuweisung, nicht Vergleich (sollte == sein)
    System.out.println("Five");
}

// Beispiel 4: Null-Wert
Object obj = null;
if (obj) {  // FEHLER: Object ist nicht boolean
    System.out.println("Exists");
}

// Beispiel 5: Arithmetischer Ausdruck
int a = 10, b = 20;
if (a + b) {  // FEHLER: a + b ist int (30), nicht boolean
    System.out.println("Sum");
}
\end{lstlisting}

\begin{achtung}
	\begin{itemize}
		\item Nach \texttt{if} muss etwas mit true/false kommen
		\item Schreibe \texttt{if (x > 0)} statt \texttt{if (x)}
		\item Ein \texttt{=} setzt einen Wert, \texttt{==} vergleicht
		\item Zahlen sind nicht automatisch true oder false
	\end{itemize}
\end{achtung}

% ============================================================
\newpage


\section*{18. Verwendung von nicht-boolean Ausdrücken in while-Schleifen}
\addcontentsline{toc}{section}{18. Verwendung von nicht-boolean Ausdrücken in while-Schleifen}

\subsection*{Beschreibung:}\\
Genau wie bei if-Anweisungen muss die Bedingung einer while-Schleife einen boolean-Wert ergeben.

\subsection*{Beispiele:}
\begin{lstlisting}
// Beispiel 1: int statt boolean
int count = 5;
while (count) {  // FEHLER: count ist int, nicht boolean
    System.out.println(count);
    count--;
}

// Beispiel 2: Zahl als Bedingung
while (10) {  // FEHLER: 10 ist kein boolean
    System.out.println("Loop");
}

// Beispiel 3: String-Bedingung
String text = "running";
while (text) {  // FEHLER: String ist nicht boolean
    System.out.println(text);
}

// Beispiel 4: Zuweisung statt Vergleich
int x = 0;
while (x = 5) {  // FEHLER: = statt == (und dann ist es auch noch int)
    x++;
}

// Beispiel 5: Array als Bedingung
int[] numbers = {1, 2, 3};
while (numbers) {  // FEHLER: Array ist nicht boolean
    System.out.println("Array");
}
\end{lstlisting}

\begin{achtung}
	\begin{itemize}
		\item Gleiche Regel wie bei \texttt{if}
		\item Schreibe \texttt{while (count > 0)} statt \texttt{while (count)}
		\item \texttt{while (true)} für Endlosschleife ist richtig
		\item Achte auf \texttt{=} (setzen) vs \texttt{==} (vergleichen)
	\end{itemize}
\end{achtung}
