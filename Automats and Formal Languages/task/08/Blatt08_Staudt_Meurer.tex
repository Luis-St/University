\documentclass[
	fontsize=12pt,          % default font size 12pt
	paper=a4,               % DIN A4 page format
	numbers=noenddot,       % remove dots behind chapter numbers (e.g. 1.5 not 1.5.)
	listof=totoc,           % add list of figures, tables, etc. to ToC
	listof=entryprefix,     % add entry name to figures, tables, etc.
	listof=nochaptergap,    % no chapter gap for figures, tables, etc.
	bibliography=totoc,     % add bibliography to ToC but without a chapter number
	parskip=half            % half line spacing between paragraphs
	openany                 % chapters can start on any page
]{scrbook}

% ##################################################
% ENCODING
% ##################################################
\usepackage{cmap}               % PDF character encoding

%\usepackage[T1]{fontenc}        % 8-bit font encoding
%\usepackage[utf8]{inputenc}     % UTF-8 input encoding
%\usepackage[german]{babel} %%%%% Sprache festlegen

\usepackage[utf8]{inputenc}
\usepackage[T1]{fontenc}
\usepackage{lmodern}
\usepackage{ngerman}


% ##################################################
% GENERAL
% ##################################################
\usepackage{scrhack}            % better KOMA adaptions
\usepackage[table]{xcolor}      % color support
\usepackage{chngcntr}           % for renumbering stuff


\usepackage{amsthm}


\usepackage{lipsum}             % lorem ipsum generator (used for example content)


\usepackage{mathtools}


% ##################################################
% PDF SETTINGS
% ##################################################
\usepackage[
	colorlinks=false,
	linkcolor=black,
	citecolor=black,
	filecolor=black,
	urlcolor=black,
	bookmarks=true,
	bookmarksopen=true,
	bookmarksopenlevel=3,
	bookmarksnumbered,
	plainpages=false,
	pdfpagelabels=true,
	hyperfootnotes
]{hyperref}


% ##################################################
% FONTS AND SPACING
% ##################################################
\renewcommand{\familydefault}{\sfdefault}   % default font
\usepackage[onehalfspacing]{setspace}       % default 1.5 line spacing
\raggedbottom   % don't stretch spacing to fit page length

\usepackage{anyfontsize} % use any font size

% font sizes and styles
\addtokomafont{chapter}{\sffamily\large\bfseries}  % chapter heading 
\addtokomafont{section}{\sffamily\normalsize\bfseries}
\addtokomafont{subsection}{\sffamily\normalsize\mdseries}
\addtokomafont{caption}{\sffamily\normalsize\mdseries}

% url font style
\usepackage{relsize}
\renewcommand*{\UrlFont}{\ttfamily\smaller\relax}


% ##################################################
% PAGE FORMATTING
% ##################################################
% Page layout / Seitenränder
\usepackage[
	bindingoffset=0cm,
	inner=2.5cm,
	outer=2.5cm,
	top=3cm,
	bottom=2cm
]{geometry}

% Page header
\usepackage[
	headsepline,        % seperator line beneath page header on normal pages
	plainheadsepline    % seperator line beneath page header on pages like ToC
]{scrlayer-scrpage}
\clearpairofpagestyles                  % clear default settings
\addtokomafont{pagehead}{\normalfont}   % use normal font for page header
\ohead*{\thepage}                       % page number
\ihead*{\leftmark}                      % chapter name


% ##################################################
% IMAGES AND FIGURES
% ##################################################
\usepackage{graphicx}       % support for including images
\graphicspath{{pictures/}}  % default path
\usepackage{float}          % better control over float positions

% simple numbering without chapter
\renewcommand{\thefigure}{\arabic{figure}}
\counterwithout{figure}{chapter}

% wrap text around figures
\usepackage{wrapfig}


% ##################################################
% TABLES
% ##################################################
% multi row and multi column table functionality
\usepackage{booktabs}   % beautiful table style
\usepackage{multirow}

% simple numbering without chapter
\renewcommand{\thetable}{\arabic{table}}
\counterwithout{table}{chapter}


% ##################################################
% SOURCE CODE LISTINGS
% ##################################################
\usepackage{listings}
\usepackage{beramono}   % use a typewriter font which supports bold characters

\renewcommand{\lstlistlistingname}{List of Code Listings}   % 
\renewcommand{\lstlistingname}{Code Listing}
\newcommand{\listoflolentryname}{\lstlistingname}   % prefix for List of Code Listings

% define colors for source code highlighting
\definecolor{codegreen}{rgb}{0,0.6,0}
\definecolor{codegray}{rgb}{0.5,0.5,0.5}
\definecolor{codepurple}{rgb}{0.5,0,0.33}
\definecolor{codepurblue}{rgb}{0.16,0.0,1.0}
\definecolor{backcolour}{rgb}{0.95,0.95,0.92}

% ##################################################
% TABLE OF CONTENTS
% ##################################################
\KOMAoptions{toc=chapterentrydotfill}       % dotted lines for chapters
\addtokomafont{chapterentry}{\normalfont}   % use normal font for chapter entries
\setuptoc{toc}{totoc}                       % add ToC to ToC

% spacing
\DeclareTOCStyleEntry[beforeskip=0cm]{chapter}{chapter}
\DeclareTOCStyleEntry[beforeskip=0cm]{section}{section}
\DeclareTOCStyleEntry[beforeskip=0cm]{default}{subsection}

% colons after entry names
\BeforeStartingTOC[lof]{\def\autodot{:}}
\BeforeStartingTOC[lot]{\def\autodot{:}}
\BeforeStartingTOC[lol]{\def\autodot{:}}


% ##################################################
% BIBLIOGRAPHY
% ##################################################
\iffalse
\usepackage{csquotes} % context sensitive quotation
\setlength\bibitemsep{.5\baselineskip} % increase spacing between entries
\setcounter{biburlnumpenalty}{9000} % break URLs on numbers
\setcounter{biburllcpenalty}{9000}  % break URLs on lower case letters
\setcounter{biburlucpenalty}{9000}  % break URLs on upper case letters

\fi

% ##################################################
% ABBREVIATIONS
% ##################################################
\usepackage[printonlyused]{acronym}


% ##################################################
% APPENDIX
% ##################################################
\usepackage[title,titletoc]{appendix}

% appendix chapter
\newcommand{\appendixchapter}[1]{
\cleardoublepage
\pagenumbering{arabic}
\renewcommand{\thepage}{\thechapter-\arabic{page}}
\chapter{#1}
}

% insert monthly report pdf as picture in order to keep page header
\newcommand{\monthlyreport}[2]{
\section{#1}
\centering
\includegraphics[trim=55 35 55 35,clip,width=1\textwidth]{#2}
\clearpage
}


% ##################################################
% Theoreme
% ##################################################

% Umgebung fuer Beispiele
\newtheorem{beispiel}{Beispiel}

% Umgebung fuer These
\newtheorem{these}{These}

% Umgebung fuer Definitionen
\newtheorem{definition}{Definition}


% ##################################################
% MISC
% ##################################################
% better referencing of images, tables, etc.
\usepackage[nameinlink, noabbrev]{cleveref}


% ############################################################################
% WIE MACH ICH DAS HIER?
% 
% 1. Schreibe im "titlepage" Abschnitt den Titel in das element mit "\fontsize{22}{22}"
% 2. Aus obsidian raus kopieren mit "Copy to LaTeX"
% 3. Zwischen "CONTENT STARTS HERE" und "CONTENT ENDS HERE" den Inhalt einfügen
% 4. Sections anpassen. Damit man ne gescheite Chapter > Section > Subsection Struktur hat
% 5. Leere Seite nach Titelseite am Anfang mit {\let\cleardoublepage\relax \chapter{Erstes Kapitel}} verhindern
%
% TABELLEN
% Müssen extra gemacht werden, da obsidian das nicht unterstützt
% Kann ich nur empfehlen: https://tableconvert.com/markdown-to-latex
%
% BILDER
% Muss man vermutlich auch extra machen, hab ich aber noch nicht probiert
% ############################################################################

\begin{document}

\begin{titlepage}
	\pagestyle{empty}
	
	% HFU Logo
	\begin{flushright}
		\begin{figure}[ht]
			\flushright
			\includegraphics[height=2cm]{../../for_latex/hfu.jpg}
		\end{figure}
	\end{flushright}
	
	\begin{center}
		\vspace{3cm}
		
		{\fontsize{22}{22} \selectfont \textbf{Blatt 8}}\\[5mm]
		{\fontsize{18}{18} \selectfont Automaten und formale Sprachen Praktikum}
		
		\vspace{12cm}
		
		\begin{tabular}{ll}
			% Teamname:       & SPInkompetent  \\\\
			Teammitglieder: & Luis Staudt \\ & Dominik Meurer
		\end{tabular}
	\end{center}
\end{titlepage}

% ############################################################################
% CONTENT STARTS HERE
% ############################################################################

{\let\cleardoublepage\relax \chapter{}}

\section{Multiplikation}

Basisfall: $\text{mult}(1, n) = n$\\
Rekursionsfall: $\text{mult}(m + 1, n) = \text{mult}(m, n) + n$

\section{Differenz}

Basisfall: $\text{diff}(m, 0) = m$\\
Rekursionsfall: $\text{diff}(m, n + 1) = \text{diff}(m, n) - 1$

\section{Absolute Differenz}

Rekursionsfall:  $\text{absdiff}(m, n) = \text{diff}(m, n) + \text{diff}(n,m)$

\section{Ist ungerade}

Basisfall: $\text{isOdd}(0) = 0$\\
Rekursionsfall: $\text{isOdd}(n + 1) = 1 - \text{isOdd}(n)$

\section{minimum}

\subsection{Verwendet sign()}

$$
\text{sign}(n) = \begin{cases*}
	1 & n > 0 \\
	0 & n = 0
\end{cases*}
$$
\\
Basisfall: $\text{sign}(0) = 0$\\
Rekursionsfall: $\text{sign}(n + 1) = (\text{sign}(n) - 1) + 1$

\subsection{Lösung}

Rekursionsfall: $\text{min}(m, n) = \text{sign}(\text{diff}(m, n)) * n + \text{sign}(\text{diff}(n,m)) * m$

\chapter{}

\begin{lstlisting}[language=java]
public class Ackermann {
  public static int ack(int x, int y){
    if (x == 0) {
      return y + 1;
    } else if (y == 0) {
      return ack(x - 1, 1);
    } else {
      return ack(x - 1, ack(x, y - 1));
    }
  }

  public static void main(String[] args){
    System.out.println(ack(3,2));
    System.out.println(ack(3,3));
    System.out.println(ack(4,1));
  }
}
\end{lstlisting}

\newpage
{\let\cleardoublepage\relax \chapter{}}

\section{$f(m, n) = (9 \ominus m^2) \times (n^2 \ominus 4)$}

\begin{table}[h!]
	\centering
	\begin{tabular}{|c|c|c|c|c|c|c|c|c|}
		\hline
		m | n & 0 & 1 & 2 & 3 & 4 & .. \\
		\hline
		0 & 9 * 0 = 0 & 9 * 0 = 0 & 9 * 0 = 0 & 9 * 2 = 2 & 9 * 4 = 36 & .. \\
		\hline
		1 & 8 * 0 = 0 & 8 * 0 = 0 & 8 * 0 = 0 & 8 * 2 = 18 & 8 * 4 = 32 & .. \\
		\hline
		2 & 5 * 0 = 0 & 5 * 0 = 0 & 5 * 0 = 0 & 5 * 2 = 10 & 5 * 4 = 20 & .. \\
		\hline
		3 & 0 * 0 = 0 & 0 * 0 = 0 & 0 * 0 = 0 & 0 * 2 = 0 & 0 * 4 = 0 & .. \\
		\hline
		.. & .. & .. & .. & .. & .. & .. \\
		\hline
	\end{tabular}
\end{table}
$f(m, n)$ is null für $\{0 <= n <= 2\}$\\
daher:\\
$$
\mu f(n) = \begin{cases*}
	0 & für 0 <= n <= 2 \\
	3 & sonst
\end{cases*}
$$

\section{$f(k, m, n) = (m \ominus k) + (n \ominus k)$}

$f(k, m, n) = 0$, wenn:
\begin{enumerate}
	\item $k >= m$
	\item $k >= n$
\end{enumerate}
somit ist $f(k, m, n) = 0$, wenn $k >= \text{max}(m, n)$\\
daher:\\
$\mu f(m, n) = \text{max}(m, n)$

\newpage
\section{$f(m, n) = \text{absdiff}(m, n)$}


\begin{table}[!ht]
	\centering
	\begin{tabular}{|c|c|c|c|c|c|c|c|}
		\hline
		m | n & 0 & 1 & 2 & 3 & 4 & 5 & 6 \\
		\hline
		0 & 0 & 1 & 2 & 3 & 4 & 5 & 6 \\
		\hline
		1 & 1 & 0 & 1 & 2 & 3 & 4 & 5 \\
		\hline
		2 & 2 & 1 & 0 & 1 & 2 & 3 & 4 \\
		\hline
		3 & 3 & 2 & 1 & 0 & 1 & 2 & 3 \\
		\hline
		4 & 4 & 3 & 2 & 1 & 0 & 1 & 2 \\
		\hline
		5 & 5 & 4 & 3 & 2 & 1 & 0 & 1 \\
		\hline
		6 & 6 & 5 & 4 & 3 & 2 & 1 & 0 \\
		\hline
	\end{tabular}
\end{table}

$f(m, n) = 0$, wenn $n = m$\\
daher:\\
$\mu f(n) = n$

\newpage

{\let\cleardoublepage\relax \chapter{}}

\section{$\mu f(n) = 3 \times n$}

$f(n, m) = \text{absdiff}(n \times 3, m)$

\begin{table}[!ht]
	\centering
	\begin{tabular}{|c|c|c|c|c|c|c|c|}
		\hline
		m | n & 0 & 1 & 2 & 3 & 4 & 5 & 6 \\
		\hline
		0 & 0 & 3 & 6 & 9 & 12 & 15 & 18 \\
		\hline
		1 & 1 & 2 & 5 & 8 & 11 & 14 & 17 \\
		\hline
		2 & 2 & 1 & 4 & 7 & 10 & 13 & 16 \\
		\hline
		3 & 3 & 0 & 3 & 6 & 9 & 12 & 15 \\
		\hline
		4 & 4 & 1 & 2 & 5 & 8 & 11 & 14 \\
		\hline
		5 & 5 & 2 & 1 & 4 & 7 & 10 & 13 \\
		\hline
		6 & 6 & 3 & 0 & 3 & 6 & 9 & 12 \\
		\hline
	\end{tabular}
\end{table}

\section{$\mu f(n) = \{0 \text{ für } 0 <= n <= 3 \text{ sonst } \bot\}$}

$f(m, n) = (n^2 \ominus 6)$

\begin{table}[h!]
	\centering
	\begin{tabular}{|c|c|c|c|c|c|c|c|c|}
		\hline
		m | n & 0 & 1 & 2 & 3 & 4 & .. \\
		\hline
		0 & 0 & 0 & 0 & 3 & 6 & .. \\
		\hline
		1 & 0 & 0 & 0 & 3 & 6 & .. \\
		\hline
		2 & 0 & 0 & 0 & 3 & 6 & .. \\
		\hline
		3 & 0 & 0 & 0 & 3 & 6 & .. \\
		\hline
		.. & .. & .. & .. & .. & .. & .. \\
		\hline
	\end{tabular}
\end{table}

% ############################################################################
% CONTENT ENDS HERE
% ############################################################################

\end{document}
