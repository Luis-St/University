\documentclass[
    fontsize=12pt,          % default font size 12pt
    paper=a4,               % DIN A4 page format
    numbers=noenddot,       % remove dots behind chapter numbers (e.g. 1.5 not 1.5.)
    listof=totoc,           % add list of figures, tables, etc. to ToC
    listof=entryprefix,     % add entry name to figures, tables, etc.
    listof=nochaptergap,    % no chapter gap for figures, tables, etc.
    bibliography=totoc,     % add bibliography to ToC but without a chapter number
    parskip=half            % half line spacing between paragraphs
    openany                 % chapters can start on any page
]{scrbook}
\usepackage{amsmath}

% ##################################################
% ENCODING
% ##################################################
\usepackage{cmap}               % PDF character encoding

%\usepackage[T1]{fontenc}        % 8-bit font encoding
%\usepackage[utf8]{inputenc}     % UTF-8 input encoding
%\usepackage[german]{babel} %%%%% Sprache festlegen

\usepackage[utf8]{inputenc}
\usepackage[T1]{fontenc}
\usepackage{lmodern}
\usepackage{ngerman}


% ##################################################
% GENERAL
% ##################################################
\usepackage{scrhack}            % better KOMA adaptions
\usepackage[table]{xcolor}      % color support
\usepackage{chngcntr}           % for renumbering stuff


\usepackage{amsthm}


\usepackage{lipsum}             % lorem ipsum generator (used for example content)


\usepackage{mathtools}


% ##################################################
% PDF SETTINGS
% ##################################################
\usepackage[
	colorlinks=false,
	linkcolor=black,
	citecolor=black,
	filecolor=black,
	urlcolor=black,
	bookmarks=true,
	bookmarksopen=true,
	bookmarksopenlevel=3,
	bookmarksnumbered,
	plainpages=false,
	pdfpagelabels=true,
	hyperfootnotes
]{hyperref}


% ##################################################
% FONTS AND SPACING
% ##################################################
\renewcommand{\familydefault}{\sfdefault}   % default font
\usepackage[onehalfspacing]{setspace}       % default 1.5 line spacing
\raggedbottom   % don't stretch spacing to fit page length

\usepackage{anyfontsize} % use any font size

% font sizes and styles
\addtokomafont{chapter}{\sffamily\large\bfseries}  % chapter heading 
\addtokomafont{section}{\sffamily\normalsize\bfseries}
\addtokomafont{subsection}{\sffamily\normalsize\mdseries}
\addtokomafont{caption}{\sffamily\normalsize\mdseries}

% url font style
\usepackage{relsize}
\renewcommand*{\UrlFont}{\ttfamily\smaller\relax}


% ##################################################
% PAGE FORMATTING
% ##################################################
% Page layout / Seitenränder
\usepackage[
	bindingoffset=0cm,
	inner=2.5cm,
	outer=2.5cm,
	top=3cm,
	bottom=2cm
]{geometry}

% Page header
\usepackage[
	headsepline,        % seperator line beneath page header on normal pages
	plainheadsepline    % seperator line beneath page header on pages like ToC
]{scrlayer-scrpage}
\clearpairofpagestyles                  % clear default settings
\addtokomafont{pagehead}{\normalfont}   % use normal font for page header
\ohead*{\thepage}                       % page number
\ihead*{\leftmark}                      % chapter name


% ##################################################
% IMAGES AND FIGURES
% ##################################################
\usepackage{graphicx}       % support for including images
\graphicspath{{pictures/}}  % default path
\usepackage{float}          % better control over float positions

% simple numbering without chapter
\renewcommand{\thefigure}{\arabic{figure}}
\counterwithout{figure}{chapter}

% wrap text around figures
\usepackage{wrapfig}


% ##################################################
% TABLES
% ##################################################
% multi row and multi column table functionality
\usepackage{booktabs}   % beautiful table style
\usepackage{multirow}

% simple numbering without chapter
\renewcommand{\thetable}{\arabic{table}}
\counterwithout{table}{chapter}


% ##################################################
% SOURCE CODE LISTINGS
% ##################################################
\usepackage{listings}
\usepackage{beramono}   % use a typewriter font which supports bold characters

\renewcommand{\lstlistlistingname}{List of Code Listings}   % 
\renewcommand{\lstlistingname}{Code Listing}
\newcommand{\listoflolentryname}{\lstlistingname}   % prefix for List of Code Listings

% define colors for source code highlighting
\definecolor{codegreen}{rgb}{0,0.6,0}
\definecolor{codegray}{rgb}{0.5,0.5,0.5}
\definecolor{codepurple}{rgb}{0.5,0,0.33}
\definecolor{codepurblue}{rgb}{0.16,0.0,1.0}
\definecolor{backcolour}{rgb}{0.95,0.95,0.92}

% ##################################################
% TABLE OF CONTENTS
% ##################################################
\KOMAoptions{toc=chapterentrydotfill}       % dotted lines for chapters
\addtokomafont{chapterentry}{\normalfont}   % use normal font for chapter entries
\setuptoc{toc}{totoc}                       % add ToC to ToC

% spacing
\DeclareTOCStyleEntry[beforeskip=0cm]{chapter}{chapter}
\DeclareTOCStyleEntry[beforeskip=0cm]{section}{section}
\DeclareTOCStyleEntry[beforeskip=0cm]{default}{subsection}

% colons after entry names
\BeforeStartingTOC[lof]{\def\autodot{:}}
\BeforeStartingTOC[lot]{\def\autodot{:}}
\BeforeStartingTOC[lol]{\def\autodot{:}}


% ##################################################
% BIBLIOGRAPHY
% ##################################################
\iffalse
\usepackage{csquotes} % context sensitive quotation
\setlength\bibitemsep{.5\baselineskip} % increase spacing between entries
\setcounter{biburlnumpenalty}{9000} % break URLs on numbers
\setcounter{biburllcpenalty}{9000}  % break URLs on lower case letters
\setcounter{biburlucpenalty}{9000}  % break URLs on upper case letters

\fi

% ##################################################
% ABBREVIATIONS
% ##################################################
\usepackage[printonlyused]{acronym}


% ##################################################
% APPENDIX
% ##################################################
\usepackage[title,titletoc]{appendix}

% appendix chapter
\newcommand{\appendixchapter}[1]{
\cleardoublepage
\pagenumbering{arabic}
\renewcommand{\thepage}{\thechapter-\arabic{page}}
\chapter{#1}
}

% insert monthly report pdf as picture in order to keep page header
\newcommand{\monthlyreport}[2]{
\section{#1}
\centering
\includegraphics[trim=55 35 55 35,clip,width=1\textwidth]{#2}
\clearpage
}


% ##################################################
% Theoreme
% ##################################################

% Umgebung fuer Beispiele
\newtheorem{beispiel}{Beispiel}

% Umgebung fuer These
\newtheorem{these}{These}

% Umgebung fuer Definitionen
\newtheorem{definition}{Definition}


% ##################################################
% MISC
% ##################################################
% better referencing of images, tables, etc.
\usepackage[nameinlink, noabbrev]{cleveref}



% ############################################################################
% WIE MACH ICH DAS HIER?
% 
% 1. Schreibe im "titlepage" Abschnitt den Titel in das element mit "\fontsize{22}{22}"
% 2. Aus obsidian raus kopieren mit "Copy to LaTeX"
% 3. Zwischen "CONTENT STARTS HERE" und "CONTENT ENDS HERE" den Inhalt einfügen
% 4. Sections anpassen. Damit man ne gescheite Chapter > Section > Subsection Struktur hat
% 5. Leere Seite nach Titelseite am Anfang mit {\let\cleardoublepage\relax \chapter{Erstes Kapitel}} verhindern
%
% TABELLEN
% Müssen extra gemacht werden, da obsidian das nicht unterstützt
% Kann ich nur empfehlen: https://tableconvert.com/markdown-to-latex
%
% BILDER
% Muss man vermutlich auch extra machen, hab ich aber noch nicht probiert
% ############################################################################

\begin{document}

\begin{titlepage}
	\pagestyle{empty}

	% HFU Logo
	\begin{flushright}
		\begin{figure}[ht]
			\flushright
			\includegraphics[height=2cm]{../../for_latex/hfu.jpg}
		\end{figure}
	\end{flushright}

	\begin{center}
		\vspace{3cm}

		{\fontsize{22}{22} \selectfont \textbf{Blatt 5}}\\[5mm]
		{\fontsize{18}{18} \selectfont Automaten und formale Sprachen Praktikum}

		\vspace{12cm}

		\begin{tabular}{ll}
			% Teamname:       & SPInkompetent  \\\\
			Teammitglieder: & Luis Staudt \\ & Dominik Meurer
		\end{tabular}
	\end{center}
\end{titlepage}

% ############################################################################
% CONTENT STARTS HERE
% ############################################################################

{\let\cleardoublepage\relax \chapter{}}

$L(G)=\{ab,aba,abaa,abaaa,abaaaa,...\}$

{\let\cleardoublepage\relax \chapter{}}

\begin{table}[!ht]
    \centering
    \begin{tabular}{|l|l|l|}
    \hline
        G & Zwischen & G' \\ \hline
        $S \rightarrow aB$ & $B \rightarrow Sa$ & $B \rightarrow Sa$ \\ \hline
        $B \rightarrow bA$ & $A \rightarrow Bb$ & $A \rightarrow Bb$ \\ \hline
        $A \rightarrow aA$ & $A \rightarrow Aa$ & $A \rightarrow Aa$ \\ \hline
        $A\rightarrow \epsilon$ & $X \rightarrow A$ & ~ \\ \hline
        ~ & ~ & $X \rightarrow Bb$ \\ \hline
        ~ & ~ & $X \rightarrow Aa$ \\ \hline
        ~ & ~ & $S \rightarrow \epsilon$ \\ \hline
    \end{tabular}
\end{table}

{\let\cleardoublepage\relax \chapter{}}

Als regulärer Ausdruck: $a*b+$

Dann ist $L(G)=\{b,ab,aab,aaab,bb,bbb,abb,abbb,aabb,...\}$

{\let\cleardoublepage\relax \chapter{}}

Weil $S \rightarrow aa$ nur Terminale hat.

$P=\{S\rightarrow aB, B\rightarrow aC, C\rightarrow \epsilon, S \rightarrow Sab\}$

{\let\cleardoublepage\relax \chapter{}}

$P=\{S\rightarrow aA, S\rightarrow bA, A\rightarrow aB, A\rightarrow bB, B\rightarrow aC, B\rightarrow bC, C\rightarrow \epsilon, C\rightarrow aA, C\rightarrow bA\}$

{\let\cleardoublepage\relax \chapter{}}

{\let\cleardoublepage\relax \section{}}

$P=\{S\rightarrow aA, A\rightarrow aB, A\rightarrow bB, B\rightarrow bB, B\rightarrow aC, C\rightarrow bB, C\rightarrow aC, C\rightarrow \epsilon\}$

{\let\cleardoublepage\relax \section{}}

$P=\{S\rightarrow aA, A\rightarrow bB, B\rightarrow aC, C\rightarrow aA, C\rightarrow \epsilon\}$

{\let\cleardoublepage\relax \chapter{}}

$P=\{S\rightarrow (S), S\rightarrow SS, S\rightarrow \epsilon\}$

\begin{figure}[!htbp]
    \centering
    \includegraphics[width=0.8\textwidth]{./images/Task05_7.png}
\end{figure}

{\let\cleardoublepage\relax \chapter{}}

\begin{align*}
    P=\{\\
    ex &::= (ex) | ex op ex | str | num,\\
    op &::= + | - | * | /,\\
    str &::= \{char\},\\
    num &::= 0 | numA \{ numB \},\\
    numA &::= 1 | 2 | 3 | 4 | 5 | 6 | 7 | 8 | 9,\\
    numB &::= 0 |numA,\\
    char &::= a | b | c | d | e | f | g | h | i | j | k | l | m | n | o | p | q | r | s | t | u | v | w | x | y | z,\\
    \}\\
\end{align*}

{\let\cleardoublepage\relax \chapter{}}

{\let\cleardoublepage\relax \section{}}

\begin{itemize}
	\item $ba$
	\item $ab$
	\item $baba$
	\item $abab$
	\item $abba$
	\item $baab$
\end{itemize}

{\let\cleardoublepage\relax \section{}}

\begin{figure}[!htbp]
    \centering
    \includegraphics[width=0.4\textwidth]{./images/Task05_9_2_1.png}
    \caption{Automat baabbbaa}
    \label{fig:baabbbaa}
\end{figure}


\begin{figure}[!htbp]
    \centering
    \includegraphics[width=0.4\textwidth]{./images/9_2_2.png}
    \caption{Automat ababaabb}
    \label{fig:ababaabb}
\end{figure}

{\let\cleardoublepage\relax \section{}}


$S \rightarrow bA$\\
$S \rightarrow aB$

$A \rightarrow a$\\
$A \rightarrow aS$\\
$A \rightarrow bAA$

$B \rightarrow b$\\
$B \rightarrow bS$\\
$B \rightarrow aBB$

S
aB
aaBB
aabB
aabaBB
aababB
aababb

S
aB
aaBB
aaBb
aabSb
aabaBb
aababb

{\let\cleardoublepage\relax \section{}}

Es erzeugt nur wörter mit gleich vielen a wie b.



% ############################################################################
% CONTENT ENDS HERE
% ############################################################################

\end{document}
