\documentclass[
	fontsize=12pt,          % default font size 12pt
	paper=a4,               % DIN A4 page format
	numbers=noenddot,       % remove dots behind chapter numbers (e.g. 1.5 not 1.5.)
	listof=totoc,           % add list of figures, tables, etc. to ToC
	listof=entryprefix,     % add entry name to figures, tables, etc.
	listof=nochaptergap,    % no chapter gap for figures, tables, etc.
	bibliography=totoc,     % add bibliography to ToC but without a chapter number
	parskip=half            % half line spacing between paragraphs
	openany                 % chapters can start on any page
]{scrbook}
\usepackage{amsmath}
\usepackage{amssymb}

% ##################################################
% ENCODING
% ##################################################
\usepackage{cmap}               % PDF character encoding

%\usepackage[T1]{fontenc}        % 8-bit font encoding
%\usepackage[utf8]{inputenc}     % UTF-8 input encoding
%\usepackage[german]{babel} %%%%% Sprache festlegen

\usepackage[utf8]{inputenc}
\usepackage[T1]{fontenc}
\usepackage{lmodern}
\usepackage{ngerman}


% ##################################################
% GENERAL
% ##################################################
\usepackage{scrhack}            % better KOMA adaptions
\usepackage[table]{xcolor}      % color support
\usepackage{chngcntr}           % for renumbering stuff


\usepackage{amsthm}


\usepackage{lipsum}             % lorem ipsum generator (used for example content)


\usepackage{mathtools}


% ##################################################
% PDF SETTINGS
% ##################################################
\usepackage[
	colorlinks=false,
	linkcolor=black,
	citecolor=black,
	filecolor=black,
	urlcolor=black,
	bookmarks=true,
	bookmarksopen=true,
	bookmarksopenlevel=3,
	bookmarksnumbered,
	plainpages=false,
	pdfpagelabels=true,
	hyperfootnotes
]{hyperref}


% ##################################################
% FONTS AND SPACING
% ##################################################
\renewcommand{\familydefault}{\sfdefault}   % default font
\usepackage[onehalfspacing]{setspace}       % default 1.5 line spacing
\raggedbottom   % don't stretch spacing to fit page length

\usepackage{anyfontsize} % use any font size

% font sizes and styles
\addtokomafont{chapter}{\sffamily\large\bfseries}  % chapter heading 
\addtokomafont{section}{\sffamily\normalsize\bfseries}
\addtokomafont{subsection}{\sffamily\normalsize\mdseries}
\addtokomafont{caption}{\sffamily\normalsize\mdseries}

% url font style
\usepackage{relsize}
\renewcommand*{\UrlFont}{\ttfamily\smaller\relax}


% ##################################################
% PAGE FORMATTING
% ##################################################
% Page layout / Seitenränder
\usepackage[
	bindingoffset=0cm,
	inner=2.5cm,
	outer=2.5cm,
	top=3cm,
	bottom=2cm
]{geometry}

% Page header
\usepackage[
	headsepline,        % seperator line beneath page header on normal pages
	plainheadsepline    % seperator line beneath page header on pages like ToC
]{scrlayer-scrpage}
\clearpairofpagestyles                  % clear default settings
\addtokomafont{pagehead}{\normalfont}   % use normal font for page header
\ohead*{\thepage}                       % page number
\ihead*{\leftmark}                      % chapter name


% ##################################################
% IMAGES AND FIGURES
% ##################################################
\usepackage{graphicx}       % support for including images
\graphicspath{{pictures/}}  % default path
\usepackage{float}          % better control over float positions

% simple numbering without chapter
\renewcommand{\thefigure}{\arabic{figure}}
\counterwithout{figure}{chapter}

% wrap text around figures
\usepackage{wrapfig}


% ##################################################
% TABLES
% ##################################################
% multi row and multi column table functionality
\usepackage{booktabs}   % beautiful table style
\usepackage{multirow}

% simple numbering without chapter
\renewcommand{\thetable}{\arabic{table}}
\counterwithout{table}{chapter}


% ##################################################
% SOURCE CODE LISTINGS
% ##################################################
\usepackage{listings}
\usepackage{beramono}   % use a typewriter font which supports bold characters

\renewcommand{\lstlistlistingname}{List of Code Listings}   % 
\renewcommand{\lstlistingname}{Code Listing}
\newcommand{\listoflolentryname}{\lstlistingname}   % prefix for List of Code Listings

% define colors for source code highlighting
\definecolor{codegreen}{rgb}{0,0.6,0}
\definecolor{codegray}{rgb}{0.5,0.5,0.5}
\definecolor{codepurple}{rgb}{0.5,0,0.33}
\definecolor{codepurblue}{rgb}{0.16,0.0,1.0}
\definecolor{backcolour}{rgb}{0.95,0.95,0.92}

% ##################################################
% TABLE OF CONTENTS
% ##################################################
\KOMAoptions{toc=chapterentrydotfill}       % dotted lines for chapters
\addtokomafont{chapterentry}{\normalfont}   % use normal font for chapter entries
\setuptoc{toc}{totoc}                       % add ToC to ToC

% spacing
\DeclareTOCStyleEntry[beforeskip=0cm]{chapter}{chapter}
\DeclareTOCStyleEntry[beforeskip=0cm]{section}{section}
\DeclareTOCStyleEntry[beforeskip=0cm]{default}{subsection}

% colons after entry names
\BeforeStartingTOC[lof]{\def\autodot{:}}
\BeforeStartingTOC[lot]{\def\autodot{:}}
\BeforeStartingTOC[lol]{\def\autodot{:}}


% ##################################################
% BIBLIOGRAPHY
% ##################################################
\iffalse
\usepackage{csquotes} % context sensitive quotation
\setlength\bibitemsep{.5\baselineskip} % increase spacing between entries
\setcounter{biburlnumpenalty}{9000} % break URLs on numbers
\setcounter{biburllcpenalty}{9000}  % break URLs on lower case letters
\setcounter{biburlucpenalty}{9000}  % break URLs on upper case letters

\fi

% ##################################################
% ABBREVIATIONS
% ##################################################
\usepackage[printonlyused]{acronym}


% ##################################################
% APPENDIX
% ##################################################
\usepackage[title,titletoc]{appendix}

% appendix chapter
\newcommand{\appendixchapter}[1]{
\cleardoublepage
\pagenumbering{arabic}
\renewcommand{\thepage}{\thechapter-\arabic{page}}
\chapter{#1}
}

% insert monthly report pdf as picture in order to keep page header
\newcommand{\monthlyreport}[2]{
\section{#1}
\centering
\includegraphics[trim=55 35 55 35,clip,width=1\textwidth]{#2}
\clearpage
}


% ##################################################
% Theoreme
% ##################################################

% Umgebung fuer Beispiele
\newtheorem{beispiel}{Beispiel}

% Umgebung fuer These
\newtheorem{these}{These}

% Umgebung fuer Definitionen
\newtheorem{definition}{Definition}


% ##################################################
% MISC
% ##################################################
% better referencing of images, tables, etc.
\usepackage[nameinlink, noabbrev]{cleveref}



% ############################################################################
% WIE MACH ICH DAS HIER?
% 
% 1. Schreibe im "titlepage" Abschnitt den Titel in das element mit "\fontsize{22}{22}"
% 2. Aus obsidian raus kopieren mit "Copy to LaTeX"
% 3. Zwischen "CONTENT STARTS HERE" und "CONTENT ENDS HERE" den Inhalt einfügen
% 4. Sections anpassen. Damit man ne gescheite Chapter > Section > Subsection Struktur hat
% 5. Leere Seite nach Titelseite am Anfang mit {\let\cleardoublepage\relax \chapter{Erstes Kapitel}} verhindern
%
% TABELLEN
% Müssen extra gemacht werden, da obsidian das nicht unterstützt
% Kann ich nur empfehlen: https://tableconvert.com/markdown-to-latex
%
% BILDER
% Muss man vermutlich auch extra machen, hab ich aber noch nicht probiert
% ############################################################################

\begin{document}

\begin{titlepage}
	\pagestyle{empty}
	
	% HFU Logo
	\begin{flushright}
		\begin{figure}[ht]
			\flushright
			\includegraphics[height=2cm]{../../for_latex/hfu.jpg}
		\end{figure}
	\end{flushright}
	
	\begin{center}
		\vspace{3cm}
		
		{\fontsize{22}{22} \selectfont \textbf{Blatt 6}}\\[5mm]
		{\fontsize{18}{18} \selectfont Automaten und formale Sprachen Praktikum}
		
		\vspace{12cm}
		
		\begin{tabular}{ll}
			% Teamname:       & SPInkompetent  \\\\
			Teammitglieder: & Luis Staudt \\ & Dominik Meurer
		\end{tabular}
	\end{center}
\end{titlepage}

% ############################################################################
% CONTENT STARTS HERE
% ############################################################################

{\let\cleardoublepage\relax \chapter{}}

{\let\cleardoublepage\relax \section{}}

% funktion, um Q auf N abzubilden

\begin{table}[!ht]
    \centering
    \begin{tabular}{|l|l|l|l|l|l|l|}
    \hline
        \textbf{} & \textbf{0} & \textbf{1} & \textbf{2} & \textbf{3} & \textbf{4} & \textbf{...} \\ \hline
        \textbf{0} & 0 & 1 & 3 & 6 & 10 & ~ \\ \hline
        \textbf{1} & 2 & 4 & 7 & 11 & 16 & ~ \\ \hline
        \textbf{2} & 5 & 8 & 12 & 17 & 23 & ~ \\ \hline
        \textbf{3} & 9 & 13 & 18 & 24 & 31 & ~ \\ \hline
        \textbf{4} & 14 & 19 & 25 & 32 & 39 & ~ \\ \hline
        \textbf{5} & 20 & 26 & 33 & 40 & 47 & ~ \\ \hline
        \textbf{6} & 27 & 34 & 41 & 48 & 55 & ~ \\ \hline
        \textbf{...} & ~ & ~ & ~ & ~ & ~ & ~ \\ \hline
    \end{tabular}
\end{table}

Horizontale $0$ Zeile

$f_0(n)=\frac{n(n+1)}{2}$

{\let\cleardoublepage\relax \subsection{Gesamte Funktion}}

\begin{align*}
	f(n,m)=\frac{(n+m)(n+m+1)}{2}+m
\end{align*}

{\let\cleardoublepage\relax \section{}}


Es kann wie ein Zahlensystem mit 3 Zeichen gesehen werden, das in das Dezimalsystem übersetzt wird.

\begin{align*}
	a &:= 1\\
	b &:= 2\\
	c &:= 3\\
	\Sigma^* &= c_n*3^n + c_{n-1}*3^{n-1} + ... + c_1*3^1 + c_0*3^0\\
	aaabbbaaac &= 1*3^9 + 1*3^8 + 1*3^7 + 2*3^6 + 2*3^5 + 2*3^4 + 1*3^3 + 1*3^2 + 1*3^1 + 3*3^0\\
	&= 19683 + 6561 + 2187 + 1458 + 486 + 162 + 27 + 9 + 3 + 3\\
	&= 30579
\end{align*}

\begin{align*}
	\Sigma^* &\rightarrow \mathbb{N}\\
	a &\rightarrow 1\\
	b &\rightarrow 2\\
	c &\rightarrow 3\\
	aa &\rightarrow 4\\
	ab &\rightarrow 5\\
	ac &\rightarrow 6\\
	ba &\rightarrow 7\\
	\end{align*}

{\let\cleardoublepage\relax \section{}}

\begin{table}[!ht]
    \centering
    \begin{tabular}{|l|l|l|l|l|l|l|l|l|}
    \hline
        ~ & 0 & 1 & 2 & 3 & 4 & ... & ~ & ~ \\ \hline
        0 & $f_0(0)$ & $f_0(1)$ & $f_0(2)$ & $f_0(3)$ & $f_0(4)$ & ~ & ~ & gerade is 0 \\ \hline
        1 & $f_1(0)$ & $f_1(1)$ & $f_1(2)$ & $f_1(3)$ & $f_1(4)$ & ~ & ~ & ungerade ist 0 \\ \hline
        2 & $f_2(0)$ & $f_2(1)$ & $f_2(2)$ & $f_2(3)$ & $f_2(4)$ & ~ & ~ & durch 5 teilbar ist 0 \\ \hline
        3 & $f_3(0)$ & $f_3(1)$ & $f_3(2)$ & $f_3(3)$ & $f_3(4)$ & ~ & ~ & ... \\ \hline
        4 & $f_4(0)$ & $f_4(1)$ & $f_4(2)$ & $f_4(3)$ & $f_4(4)$ & ~ & ~ & ~ \\ \hline
        5 & $f_5(0)$ & $f_5(1)$ & $f_5(2)$ & $f_5(3)$ & $f_5(4)$ & ~ & ~ & ~ \\ \hline
        6 & $f_6(0)$ & $f_6(1)$ & $f_6(2)$ & $f_6(3)$ & $f_6(4)$ & ~ & ~ & ~ \\ \hline
        ... & ~ & ~ & ~ & ~ & ~ & ~ & ~ & ~ \\ \hline
        ~ & ~ & ~ & ~ & ~ & ~ & ~ & ~ & ~ \\ \hline
        $g$ & $f_0(0)+1$ & $f_1(1)+1$ & $f_2(2)+1$ & $f_3(3)+1$ & $f_4(4)+1$ & ~ & ~ & hier gilt $1+1 = 0$ \\ \hline
    \end{tabular}
\end{table}

die funktion $g$ ist per definition nicht oben enthalten. Daher kann die Tabelle nicht vollständig sein.
Aus dem Wiederspruch folgt, dass die Menge aller Funktionen nicht abzählbar ist.

{\let\cleardoublepage\relax \chapter{}}

{\let\cleardoublepage\relax \section{}}

\begin{lstlisting}[caption=Inkrementieren (A \= 5)]
whilenot iszero(A) do # Setzt A auf 0
	pred(A);
od;

# Inkrementiert A auf 5
succ(A);
succ(A);
succ(A);
succ(A);
succ(A);
\end{lstlisting}

{\let\cleardoublepage\relax \section{}}

\begin{lstlisting}[caption=Addition (A + B)]
whilenot iszero(B) do # Dekrementiert B auf 0 und A hoch (A = A + B)
	succ(A);
	pred(B);
od;
\end{lstlisting}

{\let\cleardoublepage\relax \section{}}

\begin{lstlisting}[caption=Kopieren (A nach B)]
whilenot iszero(C) do # Setzt C auf 0
	pred(C);
od;

whilenot iszero(A) do # Addiert A auf B und kopiert A nach C
	succ(B);
	pred(A);
	succ(C);
od;

whilenot iszero(C) do # Restauriert A aus C
	pred(C);
	succ(A);
od;
\end{lstlisting}

{\let\cleardoublepage\relax \section{}}

\begin{lstlisting}[caption=Signum-(Vorzeichen-)Test]
whilenot iszero(B) do # Setzt B auf 0
	pred(B);
od;

whilenot iszero(A) do # Dekrementiert A auf 0, B ist nach jedem Schritt 1
	pred(A);
	pred(B);
	succ(B);
od;
\end{lstlisting}

{\let\cleardoublepage\relax \section{}}

\begin{lstlisting}[caption=Vergleichstest (A > B)]
whilenot iszero(C) do # Setzt C auf 0
	pred(C);
od;

whilenot iszero(B) do # Dekrementiert B auf 0 und dekrementiert A
	pred(A);
	pred(B);
od;

whilenot iszero(A) do # Dekrementiert A auf 0, C ist nach jedem Schritt 1
	pred(A);
	pred(C);
	succ(C);
od;
\end{lstlisting}

{\let\cleardoublepage\relax \chapter{}}

{\let\cleardoublepage\relax \section{}}

\begin{lstlisting}[caption=Subtraktion (A - B)]
whilenot iszero(B) do # Dekrementiert B auf 0 und dekrementiert A (A = A - B)
	pred(A);
	pred(B);
od;
\end{lstlisting}

{\let\cleardoublepage\relax \section{}}

\begin{lstlisting}[caption=Multiplikation (A * B)]
whilenot iszero(C) do # Setzt C auf 0
	pred(C);
od;

whilenot iszero(B) do # Dekrementiert B auf 0 und addiert A zu C in jedem Schritt
	whilenot iszero(A) do # Addiert A zu C und inkrementiert Temp
		succ(C);
		pred(A);
		succ(Temp);
	od;
	
	# A ist hier 0
 
	whilenot iszero(Temp) do # Restauriert A aus Temp
		succ(A);
		pred(Temp);
	od;

	pred(B); # B wird um 1 dekrementiert nach jedem Schritt
od;
\end{lstlisting}

\newpage

{\let\cleardoublepage\relax \section{}}

{\let\cleardoublepage\relax \subsection{}}

\begin{lstlisting}[caption=Testen auf Gleicheit]
whilenot iszero(Temp) do # Setzt Temp auf 0
	pred(Temp);
od;

succ(Temp); # Temp auf 1 setzen
whilenot iszero(Temp) do # Dekrementiert A und B, bis eines von beiden 0 ist
	if iszero(A) then
		pred(Temp);
	fi;
	if iszero(B) then
		pred(Temp);
	fi;
	
	pred(A);
	pred(B);
od;

# Wenn A und B 0 sind, dann sind sie gleich
if iszero(A) then
	if iszero(B) then
		succ(C);
	fi;
fi;
\end{lstlisting}

\newpage

{\let\cleardoublepage\relax \subsection{}}

\begin{lstlisting}[caption=Testen auf Gleicheit -> Komplex aber erster Versuch]
copy(R1, R2) # Kopiert R1 nach R2
	whilenot iszero(R_Copy) do # Setzt R_Copy auf 0
		pred(R_Copy);
	od;
	
	whilenot iszero(R1) do # Kopiert R1 nach R2 und R_Copy
		succ(R2);
		pred(R1);
		succ(R_Copy);
	od;
	
	whilenot iszero(R_Copy) do # Restauriert R1 aus R_Copy
		pred(R_Copy);
		succ(R1);
	od;

whilenot iszero(C) do
	pred(C);
od;

copy(A, A_Copy);
copy(B, B_Copy);

whilenot iszero(A) do # Dekrementiert A auf 0 (A wird 0, B unbekannt)
	pred(A);
	pred(B);
od;

whilenot iszero(B_Copy) do # Dekrementiert B auf 0 (A unbekannt, B_Copy wird 0)
	pred(B_Copy);
	pred(A_Copy);
od;

# Wenn A und B_Copy 0 sind, dann sind A und B gleich
if iszero(B) then
	if iszero(A_Copy) then
		succ(C);
	fi;
fi;
\end{lstlisting}

\newpage

{\let\cleardoublepage\relax \section{}}

\begin{lstlisting}[caption=Abstand]
whilenot iszero(Temp) do # Setzt Temp auf 0
	pred(Temp);
od;

succ(Temp); # Setzt Temp auf 1
whilenot iszero(Temp) do # Dekrementiert A und B, bis eines von beiden 0 ist
		if iszero(A) then # Temp wird 0, da A 0 ist
		pred(Temp);
	fi;
	if iszero(B) then # Temp wird 0, da B 0 ist
		pred(Temp);
	fi;
	
	pred(A);
	pred(B);
od;

if iszero(A) then # Wenn A 0 ist, dann ist B der Abstand
	whilenot iszero(B) do # Dekrementiert B und C, bis B 0 ist
		pred(B);
		succ(C);
	od;
else # Wenn B 0 ist, dann ist A der Abstand
	whilenot iszero(A) do # Dekrementiert A und C, bis A 0 ist
		pred(A);
		succ(C);
	od;
fi;
\end{lstlisting}

% ############################################################################
% CONTENT ENDS HERE
% ############################################################################

\end{document}
