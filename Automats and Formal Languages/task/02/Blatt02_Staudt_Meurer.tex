\documentclass[
	fontsize=12pt,          % default font size 12pt
	paper=a4,               % DIN A4 page format
	numbers=noenddot,       % remove dots behind chapter numbers (e.g. 1.5 not 1.5.)
	listof=totoc,           % add list of figures, tables, etc. to ToC
	listof=entryprefix,     % add entry name to figures, tables, etc.
	listof=nochaptergap,    % no chapter gap for figures, tables, etc.
	bibliography=totoc,     % add bibliography to ToC but without a chapter number
	parskip=half            % half line spacing between paragraphs
	openany                 % chapters can start on any page
]{scrbook}

% ##################################################
% ENCODING
% ##################################################
\usepackage{cmap}               % PDF character encoding

%\usepackage[T1]{fontenc}        % 8-bit font encoding
%\usepackage[utf8]{inputenc}     % UTF-8 input encoding
%\usepackage[german]{babel} %%%%% Sprache festlegen

\usepackage[utf8]{inputenc}
\usepackage[T1]{fontenc}
\usepackage{lmodern}
\usepackage{ngerman}


% ##################################################
% GENERAL
% ##################################################
\usepackage{scrhack}            % better KOMA adaptions
\usepackage[table]{xcolor}      % color support
\usepackage{chngcntr}           % for renumbering stuff


\usepackage{amsthm}


\usepackage{lipsum}             % lorem ipsum generator (used for example content)


\usepackage{mathtools}


% ##################################################
% PDF SETTINGS
% ##################################################
\usepackage[
	colorlinks=false,
	linkcolor=black,
	citecolor=black,
	filecolor=black,
	urlcolor=black,
	bookmarks=true,
	bookmarksopen=true,
	bookmarksopenlevel=3,
	bookmarksnumbered,
	plainpages=false,
	pdfpagelabels=true,
	hyperfootnotes
]{hyperref}


% ##################################################
% FONTS AND SPACING
% ##################################################
\renewcommand{\familydefault}{\sfdefault}   % default font
\usepackage[onehalfspacing]{setspace}       % default 1.5 line spacing
\raggedbottom   % don't stretch spacing to fit page length

\usepackage{anyfontsize} % use any font size

% font sizes and styles
\addtokomafont{chapter}{\sffamily\large\bfseries}  % chapter heading 
\addtokomafont{section}{\sffamily\normalsize\bfseries}
\addtokomafont{subsection}{\sffamily\normalsize\mdseries}
\addtokomafont{caption}{\sffamily\normalsize\mdseries}

% url font style
\usepackage{relsize}
\renewcommand*{\UrlFont}{\ttfamily\smaller\relax}


% ##################################################
% PAGE FORMATTING
% ##################################################
% Page layout / Seitenränder
\usepackage[
	bindingoffset=0cm,
	inner=2.5cm,
	outer=2.5cm,
	top=3cm,
	bottom=2cm
]{geometry}

% Page header
\usepackage[
	headsepline,        % seperator line beneath page header on normal pages
	plainheadsepline    % seperator line beneath page header on pages like ToC
]{scrlayer-scrpage}
\clearpairofpagestyles                  % clear default settings
\addtokomafont{pagehead}{\normalfont}   % use normal font for page header
\ohead*{\thepage}                       % page number
\ihead*{\leftmark}                      % chapter name


% ##################################################
% IMAGES AND FIGURES
% ##################################################
\usepackage{graphicx}       % support for including images
\graphicspath{{pictures/}}  % default path
\usepackage{float}          % better control over float positions

% simple numbering without chapter
\renewcommand{\thefigure}{\arabic{figure}}
\counterwithout{figure}{chapter}

% wrap text around figures
\usepackage{wrapfig}


% ##################################################
% TABLES
% ##################################################
% multi row and multi column table functionality
\usepackage{booktabs}   % beautiful table style
\usepackage{multirow}

% simple numbering without chapter
\renewcommand{\thetable}{\arabic{table}}
\counterwithout{table}{chapter}


% ##################################################
% SOURCE CODE LISTINGS
% ##################################################
\usepackage{listings}
\usepackage{beramono}   % use a typewriter font which supports bold characters

\renewcommand{\lstlistlistingname}{List of Code Listings}   % 
\renewcommand{\lstlistingname}{Code Listing}
\newcommand{\listoflolentryname}{\lstlistingname}   % prefix for List of Code Listings

% define colors for source code highlighting
\definecolor{codegreen}{rgb}{0,0.6,0}
\definecolor{codegray}{rgb}{0.5,0.5,0.5}
\definecolor{codepurple}{rgb}{0.5,0,0.33}
\definecolor{codepurblue}{rgb}{0.16,0.0,1.0}
\definecolor{backcolour}{rgb}{0.95,0.95,0.92}

% ##################################################
% TABLE OF CONTENTS
% ##################################################
\KOMAoptions{toc=chapterentrydotfill}       % dotted lines for chapters
\addtokomafont{chapterentry}{\normalfont}   % use normal font for chapter entries
\setuptoc{toc}{totoc}                       % add ToC to ToC

% spacing
\DeclareTOCStyleEntry[beforeskip=0cm]{chapter}{chapter}
\DeclareTOCStyleEntry[beforeskip=0cm]{section}{section}
\DeclareTOCStyleEntry[beforeskip=0cm]{default}{subsection}

% colons after entry names
\BeforeStartingTOC[lof]{\def\autodot{:}}
\BeforeStartingTOC[lot]{\def\autodot{:}}
\BeforeStartingTOC[lol]{\def\autodot{:}}


% ##################################################
% BIBLIOGRAPHY
% ##################################################
\iffalse
\usepackage{csquotes} % context sensitive quotation
\setlength\bibitemsep{.5\baselineskip} % increase spacing between entries
\setcounter{biburlnumpenalty}{9000} % break URLs on numbers
\setcounter{biburllcpenalty}{9000}  % break URLs on lower case letters
\setcounter{biburlucpenalty}{9000}  % break URLs on upper case letters

\fi

% ##################################################
% ABBREVIATIONS
% ##################################################
\usepackage[printonlyused]{acronym}


% ##################################################
% APPENDIX
% ##################################################
\usepackage[title,titletoc]{appendix}

% appendix chapter
\newcommand{\appendixchapter}[1]{
\cleardoublepage
\pagenumbering{arabic}
\renewcommand{\thepage}{\thechapter-\arabic{page}}
\chapter{#1}
}

% insert monthly report pdf as picture in order to keep page header
\newcommand{\monthlyreport}[2]{
\section{#1}
\centering
\includegraphics[trim=55 35 55 35,clip,width=1\textwidth]{#2}
\clearpage
}


% ##################################################
% Theoreme
% ##################################################

% Umgebung fuer Beispiele
\newtheorem{beispiel}{Beispiel}

% Umgebung fuer These
\newtheorem{these}{These}

% Umgebung fuer Definitionen
\newtheorem{definition}{Definition}


% ##################################################
% MISC
% ##################################################
% better referencing of images, tables, etc.
\usepackage[nameinlink, noabbrev]{cleveref}


% ############################################################################
% WIE MACH ICH DAS HIER?
% 
% 1. Schreibe im "titlepage" Abschnitt den Titel in das element mit "\fontsize{22}{22}"
% 2. Aus obsidian raus kopieren mit "Copy to LaTeX"
% 3. Zwischen "CONTENT STARTS HERE" und "CONTENT ENDS HERE" den Inhalt einfügen
% 4. Sections anpassen. Damit man ne gescheite Chapter > Section > Subsection Struktur hat
% 5. Leere Seite nach Titelseite am Anfang mit {\let\cleardoublepage\relax \chapter{Erstes Kapitel}} verhindern
%
% TABELLEN
% Müssen extra gemacht werden, da obsidian das nicht unterstützt
% Kann ich nur empfehlen: https://tableconvert.com/markdown-to-latex
%
% BILDER
% Muss man vermutlich auch extra machen, hab ich aber noch nicht probiert
% ############################################################################

\begin{document}

\begin{titlepage}
	\pagestyle{empty}
	
	% HFU Logo
	\begin{flushright}
		\begin{figure}[ht]
			\flushright
			\includegraphics[height=2cm]{../../for_latex/hfu.jpg}
		\end{figure}
	\end{flushright}
	
	\begin{center}
		\vspace{3cm}
		
		{\fontsize{22}{22} \selectfont \textbf{Blatt 2}}\\[5mm]
		{\fontsize{18}{18} \selectfont Automaten und formale Sprachen Praktikum}
		
		\vspace{12cm}
		
		\begin{tabular}{ll}
			% Teamname:       & SPInkompetent  \\\\
			Teammitglieder: & Luis Staudt \\ & Dominik Meurer
		\end{tabular}
	\end{center}
\end{titlepage}

% ############################################################################
% CONTENT STARTS HERE
% ############################################################################

{\let\cleardoublepage\relax \chapter{Aufgabe 1}}

\begin{enumerate}
	\item $\{\epsilon, r, rr, rrr, ...\}$
	\item $\{\epsilon, rt, rtrt, rtrtrt, ...\}$
	\item $\{\epsilon, r, t, rt, rrtt, rrrttt, ...\}$
	\item $\{\epsilon, tr, trtr, trtrtr, ...\}$
	\item $\{\epsilon, r, t, rt, tr, rtr, trt, trr, ...\}$
	\item $\{\epsilon, r, t, rr, tt, rrr, ttt, ...\}$
	\item $\{\epsilon, rt, rtrt, rtrtrt, ...\}$
	\item $\{\epsilon, 0, 1, 00, 01, 10, 11, 001, 010, 011, ...\}$
	\item $\{\epsilon, 0, 1, 00, 01, 10, 11, 001, 010, 011, ...\}$
\end{enumerate}


\begin{lstlisting}[language=java, caption=Implementierung]
package task2;

import org.junit.jupiter.api.Test;

import java.util.regex.Pattern;

import static org.junit.jupiter.api.Assertions.*;

public class RegexTest {
    
    @Test
    public void testGleitkommazahlen() {
        String gleitkomma = "[+-]?(([123456789]\\d*)(\\.\\d+)?|0(\\.\\d+)?)([eE]\\d+)?";
        
        assertTrue(Pattern.matches(gleitkomma, "22"));
        assertTrue(Pattern.matches(gleitkomma, "+2"));
        assertTrue(Pattern.matches(gleitkomma, "-10000000"));
        assertTrue(Pattern.matches(gleitkomma, "2244444444"));
        assertTrue(Pattern.matches(gleitkomma, "+2.2"));
        assertTrue(Pattern.matches(gleitkomma, "-23.211111"));
        assertTrue(Pattern.matches(gleitkomma, "-23e1"));
        assertTrue(Pattern.matches(gleitkomma, "23E1"));
        assertTrue(Pattern.matches(gleitkomma, "-23E123456634"));
        assertTrue(Pattern.matches(gleitkomma, "-211113.124566E123456634"));
        assertTrue(Pattern.matches(gleitkomma, "0"));
        assertTrue(Pattern.matches(gleitkomma, "0.0"));
        
        assertFalse(Pattern.matches(gleitkomma, "00"));
        assertFalse(Pattern.matches(gleitkomma, "1."));
        assertFalse(Pattern.matches(gleitkomma, "|1"));
        assertFalse(Pattern.matches(gleitkomma, "|245"));
        assertFalse(Pattern.matches(gleitkomma, "23e"));
        assertFalse(Pattern.matches(gleitkomma, "23f1"));
        assertFalse(Pattern.matches(gleitkomma, ""));
        assertFalse(Pattern.matches(gleitkomma, ".0"));
        assertFalse(Pattern.matches(gleitkomma, "7..0"));
        assertFalse(Pattern.matches(gleitkomma, "e4"));
        assertFalse(Pattern.matches(gleitkomma, "++e4"));
        assertFalse(Pattern.matches(gleitkomma, "+-12"));
        assertFalse(Pattern.matches(gleitkomma, "+ 12"));
        assertFalse(Pattern.matches(gleitkomma, "1:0"));
        assertFalse(Pattern.matches(gleitkomma, "1.1.0"));
        
    }
    
    @Test
    public void testZeitpunkt() {
        String zeitpunkt = "\\d{4}-(0[123456789]|1[012])-([012]\\d|3[01])T(0?\\d|1\\d|2[01234]):\\d{2}:\\d{2}:\\d{3}";
        
        assertTrue(Pattern.matches(zeitpunkt, "2012-09-30T23:28:51:544"));
        assertTrue(Pattern.matches(zeitpunkt, "2012-12-30T12:28:51:544"));
        assertTrue(Pattern.matches(zeitpunkt, "0000-01-01T00:00:00:000"));
        assertTrue(Pattern.matches(zeitpunkt, "2020-12-31T23:59:59:999"));
        
        assertFalse(Pattern.matches(zeitpunkt, "    -  -  T  :  :  :   "));
        assertFalse(Pattern.matches(zeitpunkt, "2012-20-30T23:28:51:544"));
        assertFalse(Pattern.matches(zeitpunkt, "2012-10-32T23:28:51:544"));
        assertFalse(Pattern.matches(zeitpunkt, "012-12-30T22:28:51:544"));
        assertFalse(Pattern.matches(zeitpunkt, " 012-12-30T22:28:51:544"));
        assertFalse(Pattern.matches(zeitpunkt, "2012-00-30T22:28:51:544"));
        assertFalse(Pattern.matches(zeitpunkt, "2012-12-30t5:28:51:544"));
        assertFalse(Pattern.matches(zeitpunkt, "2012-12-30T30:28:51:544"));
        assertFalse(Pattern.matches(zeitpunkt, "2012-12-30T00:8:51:544"));
        assertFalse(Pattern.matches(zeitpunkt, "2012-12-30T21:28:1:544"));
        assertFalse(Pattern.matches(zeitpunkt, "2012-12-30T21:28:511:544"));
        assertFalse(Pattern.matches(zeitpunkt, "2012-12-30T21:28:51:5440"));
    }
    
    @Test
    public void testKFZKennzeichen() {
        String kennzeichen = "([a-zA-Z]){1,3} [A-Za-z]{1,2} ([1-9]\\d{0,2} [HE]|[1-9]\\d{0,3})$";
        
        assertTrue(Pattern.matches(kennzeichen, "VS HH 1000"));
        assertTrue(Pattern.matches(kennzeichen, "HH HH 1"));
        assertTrue(Pattern.matches(kennzeichen, "H HH 1"));
        assertTrue(Pattern.matches(kennzeichen, "HH HH 1 H"));
        assertTrue(Pattern.matches(kennzeichen, "T# HH 1 H")); // LaTeX findet das Zeichen nicht. Ich bin zu faul fuer den Fix, also gibts das.
        
        assertFalse(Pattern.matches(kennzeichen, "HH HH 1 "));
        assertFalse(Pattern.matches(kennzeichen, "VS VS 0"));
        assertFalse(Pattern.matches(kennzeichen, "VS VS1 1"));
        assertFalse(Pattern.matches(kennzeichen, "VS 1"));
        assertFalse(Pattern.matches(kennzeichen, "VS ABC 1"));
        assertFalse(Pattern.matches(kennzeichen, "VS HH 0"));
        assertFalse(Pattern.matches(kennzeichen, "VS HH 10000"));
        assertFalse(Pattern.matches(kennzeichen, "VS HH 99 EH"));
        assertFalse(Pattern.matches(kennzeichen, "HVSA HH 1 "));
        assertFalse(Pattern.matches(kennzeichen, "HH HH 01"));
    }
}
\end{lstlisting}


% ############################################################################
% CONTENT ENDS HERE
% ############################################################################

\end{document}
