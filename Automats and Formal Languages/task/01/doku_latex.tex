\documentclass[
	fontsize=12pt,          % default font size 12pt
	paper=a4,               % DIN A4 page format
	numbers=noenddot,       % remove dots behind chapter numbers (e.g. 1.5 not 1.5.)
	listof=totoc,           % add list of figures, tables, etc. to ToC
	listof=entryprefix,     % add entry name to figures, tables, etc.
	listof=nochaptergap,    % no chapter gap for figures, tables, etc.
	bibliography=totoc,     % add bibliography to ToC but without a chapter number
	parskip=half            % half line spacing between paragraphs
	openany                 % chapters can start on any page
]{scrbook}

% ##################################################
% ENCODING
% ##################################################
\usepackage{cmap}               % PDF character encoding

%\usepackage[T1]{fontenc}        % 8-bit font encoding
%\usepackage[utf8]{inputenc}     % UTF-8 input encoding
%\usepackage[german]{babel} %%%%% Sprache festlegen

\usepackage[utf8]{inputenc}
\usepackage[T1]{fontenc}
\usepackage{lmodern}
\usepackage{ngerman}


% ##################################################
% GENERAL
% ##################################################
\usepackage{scrhack}            % better KOMA adaptions
\usepackage[table]{xcolor}      % color support
\usepackage{chngcntr}           % for renumbering stuff


\usepackage{amsthm}


\usepackage{lipsum}             % lorem ipsum generator (used for example content)


\usepackage{mathtools}


% ##################################################
% PDF SETTINGS
% ##################################################
\usepackage[
	colorlinks=false,
	linkcolor=black,
	citecolor=black,
	filecolor=black,
	urlcolor=black,
	bookmarks=true,
	bookmarksopen=true,
	bookmarksopenlevel=3,
	bookmarksnumbered,
	plainpages=false,
	pdfpagelabels=true,
	hyperfootnotes
]{hyperref}


% ##################################################
% FONTS AND SPACING
% ##################################################
\renewcommand{\familydefault}{\sfdefault}   % default font
\usepackage[onehalfspacing]{setspace}       % default 1.5 line spacing
\raggedbottom   % don't stretch spacing to fit page length

\usepackage{anyfontsize} % use any font size

% font sizes and styles
\addtokomafont{chapter}{\sffamily\large\bfseries}  % chapter heading 
\addtokomafont{section}{\sffamily\normalsize\bfseries}
\addtokomafont{subsection}{\sffamily\normalsize\mdseries}
\addtokomafont{caption}{\sffamily\normalsize\mdseries}

% url font style
\usepackage{relsize}
\renewcommand*{\UrlFont}{\ttfamily\smaller\relax}


% ##################################################
% PAGE FORMATTING
% ##################################################
% Page layout / Seitenränder
\usepackage[
	bindingoffset=0cm,
	inner=2.5cm,
	outer=2.5cm,
	top=3cm,
	bottom=2cm
]{geometry}

% Page header
\usepackage[
	headsepline,        % seperator line beneath page header on normal pages
	plainheadsepline    % seperator line beneath page header on pages like ToC
]{scrlayer-scrpage}
\clearpairofpagestyles                  % clear default settings
\addtokomafont{pagehead}{\normalfont}   % use normal font for page header
\ohead*{\thepage}                       % page number
\ihead*{\leftmark}                      % chapter name


% ##################################################
% IMAGES AND FIGURES
% ##################################################
\usepackage{graphicx}       % support for including images
\graphicspath{{pictures/}}  % default path
\usepackage{float}          % better control over float positions

% simple numbering without chapter
\renewcommand{\thefigure}{\arabic{figure}}
\counterwithout{figure}{chapter}

% wrap text around figures
\usepackage{wrapfig}


% ##################################################
% TABLES
% ##################################################
% multi row and multi column table functionality
\usepackage{booktabs}   % beautiful table style
\usepackage{multirow}

% simple numbering without chapter
\renewcommand{\thetable}{\arabic{table}}
\counterwithout{table}{chapter}


% ##################################################
% SOURCE CODE LISTINGS
% ##################################################
\usepackage{listings}
\usepackage{beramono}   % use a typewriter font which supports bold characters

\renewcommand{\lstlistlistingname}{List of Code Listings}   % 
\renewcommand{\lstlistingname}{Code Listing}
\newcommand{\listoflolentryname}{\lstlistingname}   % prefix for List of Code Listings

% define colors for source code highlighting
\definecolor{codegreen}{rgb}{0,0.6,0}
\definecolor{codegray}{rgb}{0.5,0.5,0.5}
\definecolor{codepurple}{rgb}{0.5,0,0.33}
\definecolor{codepurblue}{rgb}{0.16,0.0,1.0}
\definecolor{backcolour}{rgb}{0.95,0.95,0.92}

% ##################################################
% TABLE OF CONTENTS
% ##################################################
\KOMAoptions{toc=chapterentrydotfill}       % dotted lines for chapters
\addtokomafont{chapterentry}{\normalfont}   % use normal font for chapter entries
\setuptoc{toc}{totoc}                       % add ToC to ToC

% spacing
\DeclareTOCStyleEntry[beforeskip=0cm]{chapter}{chapter}
\DeclareTOCStyleEntry[beforeskip=0cm]{section}{section}
\DeclareTOCStyleEntry[beforeskip=0cm]{default}{subsection}

% colons after entry names
\BeforeStartingTOC[lof]{\def\autodot{:}}
\BeforeStartingTOC[lot]{\def\autodot{:}}
\BeforeStartingTOC[lol]{\def\autodot{:}}


% ##################################################
% BIBLIOGRAPHY
% ##################################################
\iffalse
\usepackage{csquotes} % context sensitive quotation
\setlength\bibitemsep{.5\baselineskip} % increase spacing between entries
\setcounter{biburlnumpenalty}{9000} % break URLs on numbers
\setcounter{biburllcpenalty}{9000}  % break URLs on lower case letters
\setcounter{biburlucpenalty}{9000}  % break URLs on upper case letters

\fi

% ##################################################
% ABBREVIATIONS
% ##################################################
\usepackage[printonlyused]{acronym}


% ##################################################
% APPENDIX
% ##################################################
\usepackage[title,titletoc]{appendix}

% appendix chapter
\newcommand{\appendixchapter}[1]{
\cleardoublepage
\pagenumbering{arabic}
\renewcommand{\thepage}{\thechapter-\arabic{page}}
\chapter{#1}
}

% insert monthly report pdf as picture in order to keep page header
\newcommand{\monthlyreport}[2]{
\section{#1}
\centering
\includegraphics[trim=55 35 55 35,clip,width=1\textwidth]{#2}
\clearpage
}


% ##################################################
% Theoreme
% ##################################################

% Umgebung fuer Beispiele
\newtheorem{beispiel}{Beispiel}

% Umgebung fuer These
\newtheorem{these}{These}

% Umgebung fuer Definitionen
\newtheorem{definition}{Definition}


% ##################################################
% MISC
% ##################################################
% better referencing of images, tables, etc.
\usepackage[nameinlink, noabbrev]{cleveref}


% ############################################################################
% WIE MACH ICH DAS HIER?
% 
% 1. Schreibe im "titlepage" Abschnitt den Titel in das element mit "\fontsize{22}{22}"
% 2. Aus obsidian raus kopieren mit "Copy to LaTeX"
% 3. Zwischen "CONTENT STARTS HERE" und "CONTENT ENDS HERE" den Inhalt einfügen
% 4. Sections anpassen. Damit man ne gescheite Chapter > Section > Subsection Struktur hat
% 5. Leere Seite nach Titelseite am Anfang mit {\let\cleardoublepage\relax \chapter{Erstes Kapitel}} verhindern
%
% TABELLEN
% Müssen extra gemacht werden, da obsidian das nicht unterstützt
% Kann ich nur empfehlen: https://tableconvert.com/markdown-to-latex
%
% BILDER
% Muss man vermutlich auch extra machen, hab ich aber noch nicht probiert
% ############################################################################

\begin{document}

\begin{titlepage}
	\pagestyle{empty}
	
	% HFU Logo
	\begin{flushright}
		\begin{figure}[ht]
			\flushright
			\includegraphics[height=2cm]{../../for_latex/hfu.jpg}
		\end{figure}
	\end{flushright}
	
	\begin{center}
		\vspace{3cm}
		
		{\fontsize{22}{22} \selectfont \textbf{Blatt 1}}\\[5mm]
		{\fontsize{18}{18} \selectfont Automaten und formale Sprachen Praktikum}
		
		\vspace{12cm}
		
		\begin{tabular}{ll}
			% Teamname:       & SPInkompetent  \\\\
			Teammitglieder: & Luis Staudt \\ & Dominik Meurer
		\end{tabular}
	\end{center}
\end{titlepage}

% ############################################################################
% CONTENT STARTS HERE
% ############################################################################

{\let\cleardoublepage\relax \chapter{Wortmengen durch Aufzählen aller Element}}

\begin{enumerate}
	\item $\{a,0\}^{*}_{1} = \{a;0\}$
	\item $\{a,0\}^{*}_{2} = \{aa;a0;0a;00\}$
	\item $\{a,b,c\}^{+}_{3} = \\\{aaa;aab;aac;aba;abb;abc;aca;acb;acc;\\baa;bab;bac;bba;bbb;bbc;bcb;bcc;\\caa;cab;cac;cba;cbb;cca;ccb;ccc\}$
	\item $\{a\}^{*} = \{a;aa;aaa;aaaa;...\}$
	\item $\{b\}^{*} = \{b;bb;bbb;bbbb;...\}$
	\item $\{a,b\}^{*} = \{\epsilon ;a;b;aa;bb;ab;ba;aab;...\}$
\end{enumerate}

{\let\cleardoublepage\relax \chapter{Richtig oder falsch?}}

\begin{enumerate}
	\item $\{a\}^{*} \cup \{b\}^{*} = \{a,b\}^{*}$ Das ist falsch. Links wäre es $\{a,aa,...,b,bb,...\}$ und rechts  $\{a,b,aa,ab,ba,bb,...\}$
	\item $\{a\}^{*} \cup \{b\}^{*} \subseteq \{a,b\}^{*}$ Das ist richtig, da auch jedes $e \in \{a,aa,...,b,bb,...\}$ auch in $\{a,b,aa,ab,ba,bb,...\}$ vorkommt.
	\item $\{a\}^{*} \cap \{b\}^{*} \cap \{c\}^{*} = \{\}$ Das stimmt nicht, die richtige Lösung wäre $\{\epsilon\}$
	\item $\{a,b\}^{*} \cap \{b,c\}^{*} = \{b\}^{*}$ Das stimmt, da nur diese Elemente übrig bleiben.
\end{enumerate}

{\let\cleardoublepage\relax \chapter{Behauptungen überprüfen}}

\begin{enumerate}
	\item Nein, gilt nicht für $v=`abbbb`$, $w=`aaa`$
	\item Nein, gilt nicht für $v=`aab`$, $w=`abb`$
	\item Ja, die Wörter werden aneinander gehängt, und somit die Länge um die
	entsprechende Zahl erhöht.
	\item Ja, macht nur Sinn. Wenn $`aa`$ ein Teilwort von $`aaa`$ ist und das wiederum
	von $`aaab`$, dann muss $`aa`$ auch ein Teilwort von $`aaab`$ sein.
	\item Nein, denn $(ab)^4 = abababab$.
	\item Ja, wenn $u$ mehr `a` hat, dann wäre es kein Teilwort.
	\item Ja, da sie beide echte Teilwörter oder gleich sind. Da sie nicht beide
	gleichzeitig echte Teilwörter von einander sein können, müssen sie gleich sein.
	\item Ja, wenn das Wort $\epsilon$. Das ist das neutrale Element.
	\item Ja, in ihrer Vorlesung hieß es, die Operation sei assoziativ.
	\item Nein, bei $v=`a`$ und $u=`b`$ wäre das Ergebnis $ab = ba$, was nicht stimmt.
\end{enumerate}

{\let\cleardoublepage\relax \chapter{Programmieraufgabe}}

\begin{figure}[h]
	\includegraphics[width=13cm]{./tests.png}
	\caption{Screenshot Testprotokoll}
\end{figure}

Unten gibts noch den Code dazu:

\begin{lstlisting}[language=java, caption=Implementierung]
package task1;

import java.util.Objects;

public class WortImpl implements Wort {
	
	private final String internal;
	
	public WortImpl(char[] internal) {
		this.internal = new String(internal);
	}
	
	public WortImpl(String internal) {
		this.internal = internal;
	}
	
	@Override
	public char position(int k) {
		if (k <= 0) {
			throw new IllegalArgumentException("Index must not be greater than 0");
		}
		if (k > this.internal.length()) {
			throw new IndexOutOfBoundsException("Index must not be greater than the length of the word");
		}
		return this.internal.charAt(k - 1);
	}
	
	@Override
	public int laenge() {
		return this.internal.length();
	}
	
	@Override
	public Wort concat(Wort w) {
		return new WortImpl(this.internal + w.toString());
	}
	
	@Override
	public int anzahl(char c) {
		return (int) this.internal.chars().filter(ch -> ch == c).count();
	}
	
	@Override
	public Wort tausche(char c1, char c2) {
		return new WortImpl(this.internal.replace(c1, c2));
	}
	
	@Override
	public int istTeilwortVon(Wort w) {
		return this.internal.indexOf(w.toString()) + 1;
	}
	
	@Override
	public Wort teilwort(int start, int laenge) {
		return new WortImpl(this.internal.substring(start - 1, start - 1 + laenge));
	}
	
	@Override
	public Wort ersetze(Wort w1, Wort w2) {
		return new WortImpl(this.internal.replace(w1.toString(), w2.toString()));
	}
	
	@Override
	public boolean equals(Object o) {
		if (this == o) return true;
		if (!(o instanceof WortImpl wort)) return false;
		
		return this.internal.equals(wort.internal);
	}
	
	@Override
	public int hashCode() {
		return Objects.hash(this.internal);
	}
	
	@Override
	public String toString() {
		return this.internal;
	}
}
\end{lstlisting}

% ############################################################################
% CONTENT ENDS HERE
% ############################################################################

\end{document}
